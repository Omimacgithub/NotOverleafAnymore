% Presentation guidelines: https://www.dc.fi.udc.es/ai/~cabalar/docencia.html

\documentclass{beamer}
\usepackage{xcolor}
\usetheme{Madrid}
%\usecolortheme{beaver}
%Information to be included in the title page:
\AtBeginSection[]
{
  \begin{frame}
    \frametitle{Índice}
    \tableofcontents[currentsection]
  \end{frame}
}
\definecolor{MyOrange}{RGB}{230, 85, 10}

% --- PORTADA ---
\title{Reconocimiento facial con aprendizaje máquina en arquitecturas embebidas de altas prestaciones}
\author{Omar Montenegro Macía \and \\\vspace{1em} Dirección: Carlos Vázquez Regueiro}%\\}
%\author{
%    Dirección: Carlos Vázquez Regueiro}
\institute{Universidade da Coruña}
%\institute{Universidade de Santiago de Compostela}
\date{2026}

\begin{document}

\frame{
    \begin{figure}[]
        \centering
        %   \begin{subfigure}[t]{0.3\textwidth}

        \includegraphics[width=0.3\linewidth]{../imagenes/udc.jpg}
        %   \end{subfigure}
        %  \hspace{2cm}
        % \begin{subfigure}[t]{0.3\textwidth}
        %        \includegraphics[width=0.1\linewidth]{../imagenes/usc.jpg}
        %\end{subfigure}
    \end{figure}

    \titlepage
}

% USAR FRASES CORTAS, DESTACAR PALABRAS EN CADA ITEM.

%\section{Motivación}
\begin{frame}{Motivación}
    Poner foto robot móvil con una Jetson integrada.
\end{frame}

\begin{frame}
    \frametitle{Índice}
    \tableofcontents
\end{frame}

\section{Demo}
\begin{frame}{Demos}
    GIFs con los componentes construidos funcionando
\end{frame}

\section{(Opcional?) Trabajo relacionado}
\begin{frame}{Trabajo relacionado}
    This is some text in the first frame. This is some text in the first frame. This is some text in the first frame.
\end{frame}

\section{Conclusiones}
\begin{frame}{Conclusiones}
    Seguir mismo esquema de la memoria:
    \begin{enumerate}
        \item Resumen
        \item Conclusiones positivas y negativas
        \item Trabajo futuro
    \end{enumerate}
\end{frame}

\begin{frame}[c]
    \centering
    \Large ¡Gracias por su atención!%\textcolor{MyOrange}{¡Gracias por su atención!}
\end{frame}

\end{document}
