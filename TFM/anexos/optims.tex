\chapter{Pequeñas optimizaciones al código del sistema}
\label{chap:optimus}

\lettrine{E}{n} este apéndice se exponen algunas optimizaciones realizadas al código del sistema en términos de reserva de memoria y rendimiento.

\section{Optimizaciones en la reserva de memoria}

\begin{lstlisting}[language=Python, float=t, label=coud:slots, caption=Uso de \emph{\_\_slots\_\_} para evitar la generación de diccionarios dinámicos]
class CameraNode(parent):
__slots__ = ["face_detector", "face_recognizer", "person_detector", "person_recognizer", "camera", "identity", "publisher", "image_pub"]

    def __init__(self, camera):
        self.face_detector = importer.load_model(config.FACE_DETECTION_MODEL)
        self.face_recognizer = importer.load_model(config.FACE_RECOGNITION_MODEL)
        self.person_detector = importer.load_model(config.PERSON_DETECTION_MODEL)
        self.person_recognizer = importer.load_model(config.PERSON_RECOGNITION_MODEL)
        ...
\end{lstlisting}

Debido a las limitaciones de memoria de la Jetson Orin Nano durante la ejecución del sistema, se ha optado por modificar el código a modo de hacerlo más eficiente en términos de reserva de memoria. Los cambios realizados son los siguientes:
\begin{itemize}
    \item Sustituir \textit{imports} de librerías "pesadas" (ejemplo: OpenCV o ultralytics, en torno a cientos de \acrshort{mb}) por otras más ligeras (ejemplo: numpy o scipy, en torno a decenas de \acrshort{mb}) con funciones alternativas, sin comprometer en exceso la precisión y latencia de las inferencias (ejemplo: sustituir cv2.copyMakeBorder por np.pad, o cv2.imread por PIL.Image).
    \item Uso de \textit{slots}: la variable \textit{\textunderscore \textunderscore slots\_\_} evita la creación del diccionario dinámico de Python (\textit{\_\_dict\_\_}), que almacena las variables de una \textbf{clase}. Los \textit{\_\_slots\_\_} son más rápidos y eficientes que los diccionarios dinámicos de Python, aunque también más extrictos (no se pueden crear más atributos que los especificados en \textit{\_\_slots\_\_}). Un ejemplo de uso se muestra en el código \ref{coud:slots}.
\end{itemize}

\begin{table}[]
    \begin{tabular}{c|c|c|c}
        \hline
                                                                         & Input size & \#Params & Latency (ms)                                \\ \hline
        \begin{tabular}[c]{@{}c@{}}YuNet \cite{wu2023yunet}\end{tabular} & 640x480x3  & 75856    & 9.6\footnote{tested on Intel i7-12700k CPU} \\ \cline{1-1}
        ArcFace                                                          & 112x112x3  &          &                                             \\ \cline{1-1}
        YOLO11n                                                          & 640x640x3  &          &                                             \\ \cline{1-1}
        OSNet                                                            & 258x126x3  &          &                                             \\ \hline
    \end{tabular}
    \caption{Propiedades de los modelos utilizados}
    \label{tab:modelspecs}
\end{table}

El \textbf{tamaño de un modelo} viene determinado principalmente por el tipo de \textbf{precisión} utilizado (ejemplo: capas en \acrshort{fp}16 o \acrshort{fp}32) y el número de \textbf{parámetros}. En la tabla \ref{tab:modelspecs} se muestra el tamaño de la entrada y el número de parámetros de cada modelo (extraídos directamente del modelo \acrshort{onnx}). Cada valor de la columna \textit{input size} se compone de la resolución de la imagen de entrada (ejemplo: 112x112) multiplicado por el número de canales de entrada (en este caso 3, que se corresponde con los canales \textit{Red, Green} y \textit{Blue} de una imagen \acrshort{rgb}). El modelo \acrshort{onnx} de YuNet pesa únicamente \textbf{0.2 \acrshort{mb}s}, esto se refleja en el reducido número de parámetros que lo compone. Le siguen los modelos YOLO y OSNet con 2 millones de parámetros y que pesan en torno a los \textbf{8 \acrshort{mb}s}. Finalmente, el modelo ArcFace (montado sobre el modelo ResNet100) es el que más parámetros contiene con amplia diferencia, que también se refleja en su tamaño de \textbf{152 \acrshort{mb}s}.

La versión optimizada en TensorRT del modelo ArcFace pesa \textbf{79 \acrshort{mb}}, recortando casi a la mitad el peso del modelo \acrshort{onnx}, que en mayor medida se debe al uso de la precisión \acrshort{fp}16. Aun así, utilizar otro modelo o una versión de ArcFace más reducida podría reducir en torno a \textbf{cientos de \acrshort{mb}} la ejecución del sistema.

\section{Modificaciones en el pre y postprocesado de modelos}
\label{chap:prepost}

Los pipelines de pre y postprocesado de los modelos suponen uno de los principales retos de los dispositivos Jetson, debido a limitada capacidad de procesamiento de las \acrshort{cpu} \acrshort{arm}, en este capítulo se exponen las soluciones aplicadas para mitigar el impacto computacional de dichos procesos en los modelos YOLO y OSNet.

\subsection{TODO: Postprocesado YOLO}
\label{sec:postyolo}

Respecto al postprocesado, se ha optimizado para aprovechar los arrays y operaciones de NumPy, de esta forma ha sido posible recortar varios milisegundos en todos los equipos.

\subsection{Preprocesado OSNet}
\label{sec:preosnet}

Gracias a que se disponía del código de PyTorch del modelo, se ha podido integrar el pipeline de preprocesado de OSNet. El preprocesado de OSNet lleva a cabo las siguientes operaciones aplicadas a la imagen de entrada:
\begin{itemize}
    \item Intercambiar el orden de los canales de color (\acrshort{bgr} -> \acrshort{rgb}), ya que OpenCV trabaja con imágenes en \acrshort{bgr} y TensorRT con imágenes en \acrshort{rgb}.
    \item Reescalar la imagen a la resolución de entrada del modelo.
    \item Normalizar la imagen (dividir sus valores por 255, restar la media y dividir el resultado por la desviación típica).
    \item Intercambiar el orden de las dimensiones de la imagen: HWC (Height, Width, Channels) -> CHW, que es el modo que recomienda TensorRT para presentar los datos.
    \item Expandir una dimensión: CHW -> NCHW (donde N es el tamaño del \gls{batch}).
\end{itemize}

\begin{lstlisting}[language=Python, float=t, label=coud:preosnet, caption=Preprocesado de OSNet integrado en su estructura, basicstyle=\footnotesize]
pixel_mean = [0.485, 0.456, 0.406]
pixel_std = [0.229, 0.224, 0.225]
mean = torch.tensor(pixel_mean).view(1, 3, 1, 1)
std = torch.tensor(pixel_std).view(1, 3, 1, 1)

def forward(self, x, return_featuremaps=False):
    x = x.permute(2, 0, 1) # Swap dimension order
    x = x.unsqueeze(0) # Expand dims
    x = x[:,[2, 1, 0], :, :]  # BGR to RGB
    x = x / 255.0 # Normalization
    x = (x - mean) / std # Normalization
    x = self.featuremaps(x)
    ...

\end{lstlisting}

El código \ref{coud:preosnet} muestra un extracto de la función \textit{forward} del modelo en el que se muestran las operaciones del preprocesado. Se han tenido que intercambiar funciones de OpenCV por alternativas integradas de los arrays de NumPy (ejemplo: cvtColor por permute), el reescalado de la imagen se ha dejado en el código de la inferencia.

Finalmente, se crea la instancia del modelo en PyTorch y se exporta mediante la función \textbf{torch.onnx.export}.

Es \textbf{muy importante} nombrar de forma única los \glspl{tensor} que se declaren, ya que especialmente en las últimas versiones de \gls{onnx} esto puede causar errores de \textbf{grafos acíclicos} en los modelos exportados.

\section{Memoria fijada mapeada (\textit{pinned mapped memory})}
\label{sec:pmm}

Los dispositivos con \acrshort{gpu}s integradas, como en el caso de las Jetson, disponen de la \textbf{memoria fijada mapeada} (o \textit{pinned mapped memory}) de CUDA. La memoria fijada es una región de memoria que no puede ser paginada de vuelta a disco, de forma que siempre se mantiene en \acrshort{dram}, lo que acelera las transferencias de datos. Si la \acrshort{gpu} es integrada, entonces el acceso a la memoria entre el \textit{host} (o \acrshort{cpu}) y el \textit{device} (o \acrshort{gpu}) es \textbf{directo} (se pueden mapear directamente las direcciones de memoria ente el \textit{host} y el \textit{device}), de este modo, se \textbf{evitan operaciones de copias de memoria} (cudaMemcpy), lo que permite recortar la latencia asociada a estas operaciones. Es importante mencionar que dicho tipo de memoria también se puede aplicar a \textbf{\acrshort{gpu}s discretas}, aunque su uso es ventajoso solo en ciertas situaciones \cite{leimao}.

\begin{figure}[tbp]
    \centering
    \includegraphics[width=0.75\linewidth]{imagenes/memcpy.png}
    \caption{Arquitectura de interconexión de una \acrshort{gpu} discreta y una \acrshort{gpu} integrada, figura extraída de \cite{zerocpy}}
    \label{fig:memcpy}
\end{figure}

La figura \ref{fig:memcpy} muestra un esquema de interconexión de una \acrshort{gpu} discreta (familia Intel) y una \acrshort{gpu} integrada (familia Jetson), en la que la primera requiere un cudaMemcpy para poder disponer de los datos en la \acrshort{gpu} y la segunda evita realizar transferencias debido a que se utiliza la misma \acrshort{dram} (\textbf{zero-copy}).

A pesar de los beneficios en la reducción de operaciones \textbf{cudaMemcpy} a procesar en los \glspl{stream} de CUDA, la mejora apenas ha supuesto una reducción de \textbf{1 milisegundo} en las pruebas del capítulo \ref{chap:cnn} y, debido a la falta de tiempo y a que ya se encontraban realizadas las pruebas de los modelos y del sistema, finalmente \textbf{no se ha aplicado} dicha mejora en los resultados de esta memoria.