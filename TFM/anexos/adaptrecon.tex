\chapter{Reconocimiento adaptativo}
\label{chap:adaptrecon}

\lettrine{E}{n} este apéndice se presentan los detalles de la capacidad de reconocimiento con adaptación a los cambios (o aprendizaje incremental) propuesta en \cite{Erik, CESAR}.

\section{Diseño}

\begin{figure}[tbp]
    \centering
    \includegraphics[width=1\linewidth]{imagenes/ADAPTSYS.jpg}
    \caption{Diseño del módulo de reconocimiento adaptativo}
    \label{fig:cloneADAPTSYS}
\end{figure}

En la figura \ref{fig:cloneADAPTSYS} se muestra la arquitectura del módulo de reconocimiento adaptativo ya introducido en la sección \ref{sec:archrecon}. A continuación, se comenta en profundidad todos sus componentes.

\subsection{Módulo de valoración}
\label{sec:val}

Es el módulo (función EDF en la figura \ref{fig:cloneADAPTSYS}) encargado de devolver las \textbf{puntuaciones mínimas} de cada comité (individuo) a la secuencia de entrada, que serán aplicadas en la decisión de reconocimiento. La puntuación de un comité es a su vez un valor consensuado entre los resultados de las predicciones de las \acrshort{svm} que lo conforman, el criterio de consenso (o de fusión) se basa en un \textbf{percentil} (generalmente la mediana). Aplicar percentiles ayuda a tolerar los resultados de las \acrshort{svm} \textbf{dañinas} de un comité (ejemplo: corresponden a otra persona). Para cada comité, se ejecutan las siguientes funciones:
\begin{description}
    \item[FDF (Frame Decision Function)] Se encarga de calcular las puntuaciones de cada \acrshort{svm} contra \textbf{un frame} de la secuencia. Las puntuaciones (o salidas) de cada \acrshort{svm} se \textbf{fusionan} en una única salida, que representa la puntuación mínima del comité para dicho frame. Antes de la fusión, se aplica la normalización Euclidiana a cada puntuación con el fin de hacerlas comparables. Siendo $x_{i}$ la salida de una \acrshort{svm}, se obtiene el vector normalizado $s_{i}$ como sigue en la ecuación \ref{eq:l2norm}.
    \item[SDF (Sequence Decision Function)] Se encarga de fusionar todas las puntuaciones de la anterior función para obtener un único resultado que representa a la \textbf{secuencia}.
\end{description}

\begin{equation}
    s_{i} = \frac{x_{i}}{\left\| x_{i}\right\|_{2}} \label{eq:l2norm}
\end{equation}

Finalmente, se ordenan las puntuaciones de todos los comités de menor a mayor para su uso en el módulo de reconocimiento.

\subsection{Módulo de reconocimiento}
\label{sec:reckon}

Este módulo determina, dado el conjunto de puntuaciones del módulo de valoración, si una entidad es \textbf{desconocida} o no. Para tomar dicha decisión, se ha implementado un método que compone una \textbf{distribución de Weibull} (sección \ref{sec:incrlearning}) junto a un valor de umbral, que establece una frontera entre lo conocido y lo desconocido.

\begin{figure}
    \centering
    \includegraphics[width=0.8\linewidth]{imagenes/weibullshapescale.png}
    \caption[Impacto de los parámetros \emph{shape} y \emph{scale} en la distribución de Weibull]{Impacto de los parámetros \emph{shape} y \emph{scale} en la distribución de Weibull, figura extraída de \cite{hawkins2019modeling}}
    \label{fig:weibullshapescale}
\end{figure}

La primera puntuación (la mejor) se corresponde \textit{a priori} con el comité del individuo, mientras que el resto de puntuaciones conforman la distribución de puntuaciones \textbf{no coincidentes} (denominada a partir de ahora como \textit{f(x)}), que siguen una \textbf{distribución de Weibull}. Dicha distribución se modela a partir los parámetros \textbf{\emph{shape} y \emph{scale}}, en la figura \ref{fig:weibullshapescale} se muestra el impacto al cambiar dichos parámetros.

\begin{figure}
    \centering
    \includegraphics[width=1\linewidth]{imagenes/UNK.jpg}
    \caption{Ejemplos de distribuciones de Weibull y como el umbral (\emph{Tw}) distingue entre un desconocido (\emph{unknown}) y un conocido (\emph{drift}).}
    \label{fig:FITS}
\end{figure}

A partir de la distribución \textit{f(x)}, se hallan los parámetros \emph{shape} y \emph{scale} (mediante un método de estimación de parámetros) que modelan la distribución de Weibull. Finalmente, se calcula la probabilidad de pertenencia a la distribución \textit{f(x)} de la \textbf{mejor puntuación}, es decir, la salida de la \acrfull{pdf} de Weibull. Si la probabilidad se encuentra por debajo del umbral (\textit{Tw} en la figura \ref{fig:cloneADAPTSYS}), entonces se trata de un \textbf{conocido}, en caso contrario de un \textbf{desconocido}. En la figura \ref{fig:FITS} se muestran dos ejemplos de decisiones de reconocimiento, el de la izquierda representado a un desconocido y el de la derecha a un conocido.

El método expuesto otorga un conocimiento robusto, ya que convierte puntuaciones (o scores) concretos de una \acrshort{svm} en probabilidades que siguen una teoría estadística. Esto permite la \textbf{fusión de datos de diferentes fuentes} como nuevas \acrshort{cnn} de reconocimiento o nuevas cámaras diferentes a las Kinect en el sistema \cite{scorenorm}.

\subsection{Módulo de actualización}
Es el módulo que implementa la actualización de los comités tras el reconocimiento realizado en la anterior función. Su funcionamiento se expone en la sección \ref{sec:archrecon}.

Con el fin de evitar comités \textbf{muy específicos} (ejemplo: todas sus \acrshort{svm} responden igual), se comprueba si las muestras a añadir son lo \textbf{suficientemente representativas}. Las muestras con puntuaciones dentro de la clase positiva no contribuyen a la \textbf{diversidad}, a diferencia de las que se encuentran dentro de los límites negativos, pero a costa de incluir muestras con ruido (ejemplo: borrosas), que pueden \textbf{confundir} al comité en futuros reconocimientos. Las \textbf{muestras en el borde} (cercanas al 0) añaden cierto nivel de características nuevas al comité, mientras se mantiene un riesgo bajo de mezclar muestras de otros comités \cite{Erik, CESAR}.

Como precondición de este módulo para individuos conocidos, se comprueba si la mediana de las puntuaciones de la secuencia se encuentran por debajo de 0.1, lo que indica que la secuencia contiene muestras en el borde. El módulo selecciona las muestras más cercanas al 0, que compondrán el conjunto de positivos de la nueva \acrshort{svm} del comité.

\subsection{Módulo de limitación}
\label{seq:limmod}

El módulo de limitación se activa cuando un comité excede el máximo prefijado de \acrshort{svm}s. En este caso, se elimina una de las \acrshort{svm} con base a los siguientes criterios.

\subsubsection{Diversidad}

Este criterio devuelve un valor que representa el grado de diferenciación de una \acrshort{svm} respecto a su comité. Se escoge un conjunto aleatorio de \textit{\glspl{embedding}} de entre todos los comités y se calcula la puntuación de cada \acrshort{svm}. Posteriormente, se acumula el producto de los signos entre puntuaciones y se multiplica por -1, de esta forma, las \acrshort{svm}s más discordantes respecto a la mayoría obtendrán un mayor valor de este criterio y viceversa. El valor de diversidad de cada \acrshort{svm} se calcula como en \cite{Erik}. Dado un conjunto de \glspl{embedding} \{$x_{0}$, $x_{1}$, ..., $x_{Q-1}$\} y un comité $e^{k}$ con N \acrshort{svm}s \{$h^{k}_{0}$, $h^{k}_{1}$, ..., $h^{k}_{N-1}$\}, $D(h^{k}_{i})$ se calcula como sigue en la ecuación \ref{eq:diversity}, donde \textit{sgn} es la función \textit{sign}, que devuelve 1 o -1 dependiendo del signo del número real.

\begin{align}
    D(h_{i}^{k})= \sum_{j=0;j\neq i}^{N-1} d(h_{i}^{k}, h_{j}^{k}) \label{eq:diversity} \\
    d(h_{i}^{k}, h_{j}^{k}) = -\frac{1}{Q}\sum_{q=0}^{Q-1} sgn(h_{i}^{k}(x_{q})) \cdot sgn(h_{j}^{k}(x_{q}))
\end{align}

\subsubsection{Coherencia}

Este criterio viene a determinar la precisión de una \acrshort{svm} en el reconocimiento respecto a la salida de su comité. El valor de coherencia determina cuantas veces una \acrshort{svm} devuelve el mismo signo que el resultado consensuado del comité para la secuencia de entrada. Si los signos coinciden, se suma 1 al valor de coherencia, en caso contrario, se resta 1 a dicho valor. En el momento de creación de una \acrshort{svm}, este valor se inicializa a 0 y se va acumulando, de forma que un valor alto se corresponde con una buena precisión. Dado un comité ganador $e^{k}$ de N \acrshort{svm}s \{$h^{k}_{0}$, $h^{k}_{1}$, ..., $h^{k}_{N-1}$\} y la secuencia de \textit{\glspl{embedding}} de entrada \{$x_{0}$, $x_{1}$, ..., $x_{Q-1}$\}, $C_{k,i}$ se calcula como sigue en \cite{CESAR} (ecuación \ref{eq:coherence}), donde $1[\cdot]$ representa a un condicional que devuelve 1 si la condición se cumple o 0 en caso contrario.

\begin{align}
    C_{k,i} = 1[sgn(e^{k}) == sgn(\bar{y})] - 1 [sgn(e^{k}) \neq sgn(\bar{y})] \label{eq:coherence} \\
    \bar{y} = \frac{1}{Q} \sum_{q=1}^{Q-1} h_{n}^{k} (x_{q})
\end{align}

\subsubsection{Favorabilidad}

Finalmente, los dos criterios anteriores se fusionan en un valor llamado \textbf{índice de favorabilidad}. La fórmula para calcular dicho índice es la misma que en \cite{CESAR}. Dado el valor de coherencia $C_{k,i}$ de la \acrshort{svm} $h^{i}$ del comité ganador $e^{k}$ y el valor de diversidad $D(h^{m})$, la favorabilidad se calcula como sigue en la ecuación \ref{eq:fav}, donde son $\alpha$ y $\gamma$ constantes que ajustan el peso de la coherencia y la diversidad respectivamente.

\begin{equation}
    F(h_{i}^{k}) = \alpha C_{k,i} + \gamma D(h_{i}^{k}) \label{eq:fav}
\end{equation}

La \acrshort{svm} con el menor valor en el índice de favorabilidad \textbf{es eliminada del comité}.

\section{Implementación}
\label{sec:adaptimpel}

En esta sección se explican algunos de los detalles de implementación de los componentes que conforman esta funcionalidad. Para la mayoría de componentes se ha tomado como base el siguiente repositorio \cite{src}.

\subsection{Módulo de valoración}

La implementación sigue el funcionamiento descrito en la sección \ref{sec:val}. Para calcular los percentiles y la mediana en las funciones FDF y SDF se utilizan las funciones \textit{percentile} y \textit{median} respectivamente de la librería NumPy.

\subsection{Módulo de reconocimiento}

\begin{lstlisting}[language=Python, float=t, label=coud:weib, caption=Módulo de reconocimiento, basicstyle=\footnotesize]
import numpy as np
from scipy import stats

def weib(x,scale,shape):
    if x < 0:
      return 0
    return (shape / scale) * (x / scale) ** (shape - 1) * np.exp(- (x / scale) ** shape)

def RDF(R, sorted_ens_scr, Tw):
  #R: original set of ensemble scores
  #sorted_ens_scr: sorted set of ensemble scores
  #Tw: Weibull threshold
  c = sorted_ens_scr[0]
  set_of_mins = sorted_ens_scr[1:int(len(sorted_ens_scr))]
  m = np.median(sorted_ens_scr[1:])
  v = np.abs(set_of_mins - m) #Non-match distribution
  cand_scr = np.abs(c - m)  #Best score
  #Obtain shape and scale Weibull parameters
  shape,loc,scale = stats.weibull_min.fit(v, floc=0)  
  pdf = weib(cand_scr, scale, shape)  #Calculate PDF
  # Decision
  if (pdf<Tw):
    decision = np.where( R == c )[0][0] # Drift
  else:
    decision = -1 # Unknown
  return decision
\end{lstlisting}

El código \ref{coud:weib} muestra la implementación del módulo de reconocimiento. Para que el reconocimiento pueda resolver casos en los que hay disputa entre comités (ejemplo: secuencia con muestras en el borde), se calcula para cada punto la distancia respecto a la mediana, de forma que los puntos de la distribución se distancian de la mejor puntuación. Se obtienen los parámetros \textit{shape} y \textit{scale} de la distribución de Weibull de puntuaciones no coincidentes (función \emph{scipy.stats.weibull\_min.fit}) y se calcula la probabilidad de pertenencia (\acrshort{pdf}) de la mejor puntuación a la distribución (función \emph{weib}). Si la probabilidad es inferior al umbral (\textit{Tw}), se reconoce el caso como un extremo y se asigna la identidad correspondiente al sujeto (\textit{drift}), en caso contrario se devuelve como desconocido (\textit{unknown}).

Un cambio importante respecto a la implementación de \cite{Erik}, es que se toman \textbf{todas las puntuaciones} de los comités (excepto el comité presuntamente ganador) en vez de \textbf{la mitad de las puntuaciones} para formar la distribución de puntuaciones no coincidentes. La principal razón de este cambio es la \textbf{escasa disponibilidad} de individuos para hacer las pruebas (únicamente 10), que limita la correcta definición de la distribución de Weibull.

\subsection{Módulo de limitación}

Cuando se excede el límite de \acrshort{svm}s de un comité, se invoca a la función que calcula el criterio de favorabilidad, que otorga un valor para cada \acrshort{svm} a partir de la suma de los criterios de diversidad y coherencia. Se almacenan los resultados en un array de NumPy y se ordenan mediante la función \textit{argsort} de NumPy, por lo que la \acrshort{svm} a eliminar se corresponde con la primera entrada del vector (es decir, la que tenga el valor de favorabilidad más bajo).
Los valores de coherencia se calculan sobre el valor acumulado de cada \acrshort{svm} cada vez que se invoca a este módulo. Por simpleza, se ha optado por reiniciar este valor en todo el comité a 0 cada vez que se reemplaza una \acrshort{svm}, de forma que se mantiene una comparación justa respecto a las \acrshort{svm} recientes.