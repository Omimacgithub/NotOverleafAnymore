%%%%%%%%%%%%%%%%%%%%%%%%%%%%%%%%%%%%%%%%%%%%%%%%%%%%%%%%%%%%%%%%%%%%%%%%%%%%%%%%

\begin{abstract}
    \thispagestyle{empty}
    El auge del Deep Learning (DL) y las redes neuronales profundas han impulsado numerosos avances en aplicaciones como la visión artificial. Sin embargo, su implementación en sistemas embebidos en tiempo real (conocido como TinyML) sigue representando un reto debido a los elevados recursos computacionales que requieren. La optimización de dichas redes para su ejecución en sistemas embebidos es clave para proporcionar el internet de las cosas (IoT) y otras aplicaciones relacionadas.

    En el ámbito del reconocimiento de personas, una característica deseable sería que las redes neuronales fueran capaces de reconocer a nuevos individuos y guardar una etiqueta para referenciarlos en un futuro. Sin embargo, dichas redes son fuertemente dependientes del conjunto de datos de entrenamiento, lo que acaba en una pérdida significativa de precisión a la hora de
    manejar datos diferentes a los del entrenamiento. La perspectiva del
    aprendizaje logra superar esta limitación, al otorgar la capacidad de añadir nuevos datos durante el tiempo de operación (\textit{Open-World}), el sistema es capaz de evolucionar y extenderse a nuevos contextos sin necesidad de reentrenar las redes neuronales, lo que puede ser un proceso extremadamente costoso.

    Este proyecto propone la implementación de un método de aprendizaje máquina para la inclusión de nuevos individuos (\textit{Open-World}). Dicho método se ha integrado en un sistema de detección y reconocimiento de personas, diseñado para el robot móvil Summit\_XL, que cuenta con un sensor LiDAR y múltiples cámaras RGBD. El sistema está implementado en la arquitectura ROS, que permite ejecutar procesos de forma distribuida en diferentes máquinas. Con el motivo del fin de soporte de ROS Noetic el 31 de Mayo de 2025, se ha realizado una migración a la distribución ROS 2 Humble, con el fin de mantener el sistema actualizado a las nuevas versiones de Linux y a los nuevos robots del mercado.

    El software se ha adaptado para funcionar en la familia de placas NVIDIA Jetson y otros dispositivos relacionados (GeForce GTX y RTX). Se ha analizado el rendimiento de los modelos de detección y reconocimiento en varios motores de inferencia como OpenVINO y TensorRT. Adicionalmente, se ha medido el rendimiento del sistema completo en tiempo real y su escalabilidad en función del número de cámaras y de nodos que procesan sus datos.

    Las pruebas se han realizado a partir de muestras (videos etiquetados) de un entorno de operación real, con movimientos bruscos de las cámaras y entrecruzado de los individuos.

    \vspace*{25pt}
    \begin{english}
        \centerline{\bfseries \abstractname}
        a
    \end{english}

    \vspace*{25pt}
    \begin{multicols}{2}
  \begin{description}
    \item [Palabras chave:] \mbox{} \\[-20pt]
          \begin{itemize}
            \item Aprendizaje incremental
            \item Cámaras RGBD
            \item Unidad de procesamiento gráfico (GPU)
            \item Máquinas de vectores de soporte (SVM)
            \item Aprendizaje por comités
            \item NVIDIA Jetson
            \item TensorRT
            \item Sistema operativo robótico (ROS)
          \end{itemize}
  \end{description}
  \begin{description}
    \item [Keywords:] \mbox{} \\[-20pt]
          \begin{itemize}
            \item Incremental learning
            \item RGBD cameras
            \item Graphics Processing Unit (GPU)
            \item Support Vector Machine (SVM)
            \item Ensemble learning
            \item NVIDIA Jetson
            \item TensorRT
            \item Robotic Operating System (ROS)
          \end{itemize}
  \end{description}
\end{multicols}
\end{abstract}

%%%%%%%%%%%%%%%%%%%%%%%%%%%%%%%%%%%%%%%%%%%%%%%%%%%%%%%%%%%%%%%%%%%%%%%%%%%%%%%%
