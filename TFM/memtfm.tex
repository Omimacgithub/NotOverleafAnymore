%%%%%%%%%%%%%%%%%%%%%%%%%%%%%%%%%%%%%%%%%%%%%%%%%%%%%%%%%%%%%%%%%%%%%%%%%%%%%%%%
% Preámbulo                                                                    %
%%%%%%%%%%%%%%%%%%%%%%%%%%%%%%%%%%%%%%%%%%%%%%%%%%%%%%%%%%%%%%%%%%%%%%%%%%%%%%%%

\documentclass[11pt,a4paper,titlepage,twoside,openright,openbib]{report}


\usepackage{styletfm}

\newcommand{\dataacceso}[1]{#1}
\usepackage{multirow}
\usepackage{tabularx}
\usepackage{pdflscape} % https://tex.stackexchange.com/questions/19017/how-to-place-a-table-on-a-new-page-with-landscape-orientation-without-clearing-t
\usepackage{capt-of}
\usepackage{subcaption} % permite a inclusión de varias subfiguras nunha figura

\hypersetup{pdfauthor={Omar Montenegro Macía},
            pdftitle={OMMTFM},
            pdfsubject={Traballo Fin de Máster},
            pdfkeywords={De-SVM, Open-World, Cámaras RGBD, reconocimiento de caras, aprendizaje semi-supervisado, Jetson, TinyML, Arquitecturas embebidas.}}

%%%%%%%%%%%%%%%%%%%%%%%%%%%%%%%%%%%%%%%%%%%%%%%%%%%%%%%%%%%%%%%%%%%%%%%%%%%%%%%%
% Cuerpo                                                                      %
%%%%%%%%%%%%%%%%%%%%%%%%%%%%%%%%%%%%%%%%%%%%%%%%%%%%%%%%%%%%%%%%%%%%%%%%%%%%%%%%

%\usepackage[acronym]{glossaries}
%\usepackage{glossaries}

%%\chapter{Glosario de términos}
%\label{chap:glosario-terminos}

%\begin{description}
% \item [Aceleración] Ratio que relaciona el tiempo de ejecución de un proceso antes y después de una optimización aplicada sobre dicho proceso. 
%\end{description}
\newglossaryentry{dataset}
{
    name=dataset,
    description={Conjunto de datos normalmente extenso, que contiene información veraz utilizada para entrenar, validar o evaluar modelos de inteligencia artificial.}
}
\newglossaryentry{gt}
{
    name=ground truth,
    description={Información veraz que se proporciona a un modelo para evaluar su precisión. En visión artificial, puede referirse a una imagen etiquetada con la información del objeto que se muestra.}
}
\newglossaryentry{embsystem}
{
    name=sistema embebido,
    description={Circuito integrado capaz de realizar tareas computacionales. Puede llegar a incluir desde un simple microcontrolador hasta una arquitectura multiprocesador con dispositivos periféricos y red \cite{embsys}.}
}
\newglossaryentry{bbox}
{
    name=bounding box,
    description={Área que encierra a un objeto identificado por un modelo de detección y que otorga el tamaño y la posición de dicho objeto. Sirve para evaluar la precisión de modelos de detección, implementar algoritmos de seguimiento, entre otras funciones.}
}

\newglossaryentry{pdf}
{
    name=Probability Density Function,
    description={TODO.}
}

\newglossaryentry{wrapper}
{
    name=wrapper,
    description={Interfaz que invoca al código que realiza las computaciones.}
}

\newglossaryentry{profiling}
{
    name=profiling,
    description={Fase del software en el que se analiza el rendimiento de un programa durante su ejecución. Los \textit{profilers} son las herramientas diseñadas para este fin.}
}
\newglossaryentry{batch}
{
    name=batch,
    description={Conjunto de datos que suponen la entrada de la inferencia en un modelo de inteligencia artificial.}
}
\newglossaryentry{pcl}
{
    name=nube de puntos,
    description={Mapa que alberga puntos tridimensionales que definen la forma o superficie de un objeto en el entorno. Los puntos contienen la información de las coordenadas en el formato (x, y, z), adicionalmente, puede contener información como el color o la intensidad.}
}
\newglossaryentry{timestamp}
{
    name=timestamp,
    description={Marca temporal que puede asociarse a un mensaje para indicar el instante en el que se generó o para cualquier otro evento.}
}

%%\chapter{Glosario de acrónimos}
%\label{chap:acronimos}

%%%%%%%%%%%%%%%%%%%%%%%%%%%%%%%%%%%%%%%%%%%%%%%%%%%%%%%%%%%%%%%%%%%%%%%%%%%%%%%%

%\newacronym{gcd}{GCD}{Greatest Common Divisor}

%\begin{description}
%	\item [IoI] \emph{Individual of Interest}.
%\end{description}
%\newacronym{ioi}{IoI}{Individual of Interest}
%\newacronym{sota}{SOTA}{State-of-the-art}

%%%%%%%%%%%%%%%%%%%%%%%%%%%%%%%%%%%%%%%%%%%%%%%%%%%%%%%%%%%%%%%%%%%%%%%%%%%%%%%%


\begin{document}

%%%%%%%%%%%%%%%%%%%%%%%%%%%%%%%%%%%%%%%%
% Preliminares del documento           %
%%%%%%%%%%%%%%%%%%%%%%%%%%%%%%%%%%%%%%%%

\renewcommand{\lstlistingname}{Código}
\renewcommand{\lstlistlistingname}{Índice de \lstlistingname s}
\renewcommand{\tablename}{Tabla}
\renewcommand{\listtablename}{Índice de \tablename s}

\begin{titlepage}

  \hspace*{128pt}
  \textcolor{udcgray}{{\fontencoding{T1}\fontfamily{phv}\selectfont Departamento de Ingeniería de Computadores}}\\[-2pt]
  \hspace*{145pt}
  \textcolor{udcpink}{{\fontencoding{T1}\fontfamily{phv}\selectfont Facultad de Informática de A Coruña}}\\[-32pt]
  \textcolor{udcpink}{{\fontencoding{T1}\fontfamily{phv}\selectfont Escuela Técnica Superior de Ingeniería}}\\[-22pt]

  \begin{center}
    \includegraphics[scale=0.3]{imagenes/udc}\\[35pt]
    %\includegraphics[scale=0.1]{imagenes/usc.jpg}\\[35pt]

    {\large TRABAJO FIN DE MÁSTER \\
    MÁSTER INTERUNIVERSITARIO EN  \\
    COMPUTACIÓN DE ALTAS PRESTACIONES \\
    }

    \begin{huge}
      \bfseries Reconocimiento facial \\[7pt]
      con aprendizaje máquina \\[7pt]
      en arquitecturas embebidas \\[7pt]
      de altas prestaciones \\[7pt]
    \end{huge}
  \end{center}

  \vfill

  \begin{flushright}
    {\large
      \begin{tabular}{ll}
        {\bf Estudiante:}       & Omar Montenegro Macía   \\
        {\bf Director/a/es/as:} & Carlos Vázquez Regueiro \\
      \end{tabular}}
  \end{flushright}
  \rightline{A Coruña, \today.}
\end{titlepage}

\paginaenblanco
\dedicatoria{A la mama, pase lo que pase}
\paginaenblanco
\paginaenblanco
\begin{agradecimientos}
    En primer lugar, quisiera agradecer a Carlos Vázquez Regueiro, ya que sin su apoyo constante el proyecto no habría seguido el rumbo que debería. También doy las gracias al CITIC (Centro de Investigación en Tecnoloxías da Información e das Comunicacións) por ofrecernos el robot móvil Summit\_XL y los dispositivos Jetson Orin Nano, Jetson AGX Orin y Jetson AGX Thor para la realización de este proyecto (la Xavier NX es de Marga?).
\end{agradecimientos}
\paginaenblanco
%%%%%%%%%%%%%%%%%%%%%%%%%%%%%%%%%%%%%%%%%%%%%%%%%%%%%%%%%%%%%%%%%%%%%%%%%%%%%%%%

\begin{abstract}
    \thispagestyle{empty}
    El auge del Deep Learning (DL) ha impulsado numerosos avances en aplicaciones como la visión artificial. No obstante, estos modelos dependen en gran medida de los datos de entrenamiento, lo que limita su capacidad de generalización a nuevos contextos. El aprendizaje incremental aborda esta problemática al permitir el reconocimiento de datos desconocidos (\textit{Open-Set}) y su inclusión durante el tiempo de operación (\textit{Open-World}) sin necesidad de reentrenar el modelo desde cero. En aplicaciones de redes neuronales en tiempo real con enfoque en sistemas embebidos, se requiere además de un esfuerzo de optimización debido a las restricciones computacionales existentes.

    Este proyecto propone la integración de un método de aprendizaje incremental en un sistema de detección y reconocimiento de personas, diseñado para el robot móvil Summit\_XL, equipado con múltiples cámaras RGBD. Con el fin de garantizar la eficacia del enfoque propuesto, el sistema se ha extendido al procesamiento de vídeo, permitiendo una recolección de datos más eficiente durante la operación. Se analizan las capacidades del aprendizaje implementado en escenarios con conocimiento parcial de los datos (\textit{semisupervised learning}) y sin conocimiento previo (\textit{unsupervised learning}). Dicho sistema está implementado en la arquitectura ROS, que permite la interacción con el robot por medio de una red distribuida. Debido al fin de soporte de ROS 1 el 31 de mayo de 2025, se ha realizado una migración a ROS 2 para garantizar su continuidad.

    El sistema ha sido adaptado para aprovechar las capacidades de la familia de placas NVIDIA Jetson y otros dispositivos relacionados (GeForce GTX y RTX). Se evalúa el rendimiento de los modelos de redes neuronales en distintas arquitecturas de GPU y se analiza la escalabilidad del sistema completo en tiempo real en función del número de cámaras.

    Todas las pruebas se han realizado a partir de muestras (videos etiquetados) obtenidas en un entorno real de operación, con movimientos de las cámaras y entrecruzado de los individuos.

    \vspace*{25pt}
    \begin{english}
        \centerline{\bfseries \abstractname}
        The rise of Deep Learning (DL) has led to significant advances in applications such as computer vision. However, these models heavily depend on the training data, which limits their ability to generalize to new contexts. Incremental learning addresses this limitation by enabling the recognition of previously unseen data (Open-Set) and their incorporation during the operational phase (Open-World), without the need to retrain the model from scratch. In real-time neural network applications targeting embedded systems, additional optimization efforts are required due to existing computational constraints.

        This project proposes the integration of an incremental learning method into a person detection and recognition system designed for the mobile robot Summit\_XL, equipped with multiple RGBD cameras. To ensure the effectiveness of the proposed approach, the system has been extended to process video streams, enabling more efficient data collection during operation. The capabilities of the implemented learning strategy are analyzed under scenarios with partial prior knowledge of the data (\textit{semisupervised learning}) and with no prior knowledge (\textit{unsupervised learning}). The system is implemented within the ROS architecture, which enables interaction with the robot through a distributed network. Due to the end of support for ROS 1 on May 31, 2025, the system has been migrated to ROS 2 to ensure its long-term maintainability.

        The system has been adapted to exploit the capabilities of the NVIDIA Jetson family of embedded platforms, as well as related devices (GeForce GTX and RTX). The performance of the neural network models is evaluated across different GPU architectures, and the real-time scalability of the complete system is analyzed according to the number of cameras.

        All experiments have been conducted using labeled video samples acquired in a real-world operational environment, characterized by camera movements and frequent inter-person occlusions.
    \end{english}

    \vspace*{25pt}
    \begin{multicols}{2}
  \begin{description}
    \item [Palabras chave:] \mbox{} \\[-20pt]
          \begin{itemize}
            \item Aprendizaje incremental
            \item Cámaras RGBD
            \item Unidad de procesamiento gráfico (GPU)
            \item Máquinas de vectores de soporte (SVM)
            \item Aprendizaje por comités
            \item NVIDIA Jetson
            \item TensorRT
            \item Sistema operativo robótico (ROS)
          \end{itemize}
  \end{description}
  \begin{description}
    \item [Keywords:] \mbox{} \\[-20pt]
          \begin{itemize}
            \item Incremental learning
            \item RGBD cameras
            \item Graphics Processing Unit (GPU)
            \item Support Vector Machine (SVM)
            \item Ensemble learning
            \item NVIDIA Jetson
            \item TensorRT
            \item Robotic Operating System (ROS)
          \end{itemize}
  \end{description}
\end{multicols}
\end{abstract}

%%%%%%%%%%%%%%%%%%%%%%%%%%%%%%%%%%%%%%%%%%%%%%%%%%%%%%%%%%%%%%%%%%%%%%%%%%%%%%%%

\paginaenblanco

\pagenumbering{roman}
\setcounter{page}{1}

\tableofcontents
\listoffigures
\listoftables
\lstlistoflistings

\pagenumbering{arabic}
\setcounter{page}{1}

%%%%%%%%%%%%%%%%%%%%%%%%%%%%%%%%%%%%%%%%
% Capítulos                            %
%%%%%%%%%%%%%%%%%%%%%%%%%%%%%%%%%%%%%%%%

\chapter{Introducción}
\label{chap:introduccion}
\lettrine{E}{n} este capítulo se expone la motivación y los objetivos de este proyecto. Adicionalmente, se comenta la estructura por capítulos de esta memoria y brevemente su contenido.

\section{Motivación}
Con la aparición del Deep Learning y las redes neuronales profundas, se han logrado notables avances en términos de precisión en áreas como la visión artificial. Sin embargo, debido a sus elevados requerimientos computacionales, los modelos de Deep Learning suelen ejecutarse en entornos en la nube \cite{MITTAL2019428}, lo que no es una solución viable en aplicaciones con ejecución en tiempo real debido a la latencia en las respuestas. En este contexto, surge el concepto de TinyML \cite{OLIVEIRA2024101153}, que consiste en la implementación de modelos compactos de forma eficiente en \glspl{embsystem} como microcontroladores (Low-end TinyML, ejemplo: Arduino) u ordenadores de placa única (High-end TinyML, ejemplo: NVIDIA Jetson), más potentes que los microcontroladores, pero menos económicos.

Dado que el caso de uso de este proyecto es el reconocimiento de personas en tiempo real a partir de los datos de múltiples cámaras, resultaría muy beneficioso aplicar la \textbf{computación en el borde o Edge Computing}. Este paradigma consiste en procesar los datos desde su fuente de origen o desde un dispositivo muy cercano, de esta forma, se reduce la latencia de transmisión y se reparte el volumen de datos entre múltiples dispositivos de bajo consumo \cite{EDCOM}. Este proyecto propone la integración de un sistema compuesto por varias redes neuronales \cite{andrew} en varios dispositivos Jetson y su análisis en un entorno de operación real con múltiples cámaras. El sistema emplea el \textit{framework} ROS \cite{ROS}, estándar de facto en robótica, que permite la ejecución distribuida de procesos en diferentes máquinas y posee una amplia biblioteca de sensores y algoritmos.

Durante el desarrollo del sistema base, se contempló implementar la capacidad de detectar cuando un individuo es un \textbf{desconocido} (individuo no registrado previamente en el sistema) y en ese caso agregarlo al sistema (\textit{Open-World})\cite{andrew}. Esta funcionalidad es muy útil para implementar un robot móvil guía que asista a los usuarios proporcionando un trato cercano en todo momento (ejemplo: saludarlos por su nombre). Este proyecto propone la implementación de un método de aprendizaje máquina para detectar e incluir nuevos individuos basado en las ideas de una tesis doctoral \cite{Erik}. El método requiere de un conjunto reducido de datos para su inicialización (por ejemplo, los datos biométricos de 5 personas). Dado que la recopilación de imágenes faciales está fuertemente regulada por la ley de protección de datos, el uso de este enfoque resulta especialmente relevante para el presente proyecto \cite{Erik}.

\section{Objetivos}

El objetivo global del proyecto es la implementación y funcionamiento de un método de aprendizaje máquina bajo un sistema para el reconocimiento y seguimiento de personas en tiempo real en sistemas embebidos de altas prestaciones de la familia NVIDIA Jetson y dispositivos similares. En concreto, el proyecto pretende abordar los siguientes objetivos:

O1. Extender la funcionalidad de un sistema de reconocimiento y seguimiento de personas para robots móviles con el fin de detectar desconocidos e ingresarlos en el sistema.

O2. La ejecución eficiente del programa en sistemas embebidos y otros dispositivos.

TODO: comparan la AGX Xavier (+1000 leuros) con una GTX 1050 y los resultados son muy similares \cite{murthy2020investigations}

\section{Estructura de la memoria}

Esta memoria se divide en capítulos, cada uno con una explicación detallada del trabajo realizado. Se expone a continuación la estructura de esta memoria, explicando brevemente su contenido:

\begin{enumerate}
    \item \textbf{Introducción}: presente capítulo en el que se exponen los motivos de realización de este proyecto, los objetivos planteados y la estructura de la memoria resultante.
    \item \textbf{Fundamentos teóricos y tecnológicos}: capítulo en el que se presentan los conceptos clave del proyecto, así como las herramientas tanto hardware como software utilizadas.
    \item \textbf{Gestión del proyecto}: en este capítulo se presentan los requisitos, actores y casos de uso que definen el sistema propuesto, el plan de gestión de los riesgos, la metodología de desarrollo empleada y una vista detallada de la planificación del proyecto y su posterior seguimiento.
    \item \textbf{Integración y extensión del sistema base}: capítulo en el que se expone la arquitectura del sistema del proyecto, el procedimiento seguido para integrar dicho sistema en la familia de dispositivos NVIDIA Jetson y varias modificaciones del sistema necesarias para el correcto transcurso del proyecto.
    \item \textbf{Reconocimiento facial adaptativo}: capítulo en el que se detallan las características del método de aprendizaje máquina propuesto y su implementación.
    \item \textbf{Pruebas y resultados}: capítulo en el que se presenta la preparación de las pruebas y se arrojan resultados de todos los componentes implementados.
    \item \textbf{Protección de templates}: estudio de múltiples opciones para la ocultación de los datos biométricos y resultados de la opción más viable.
    \item \textbf{Conclusiones y trabajo futuro}: análisis de los resultados e hitos alcanzados en el proyecto y propuesta de varias líneas de investigación y desarrollo que podrían mejorar los resultados obtenidos.
\end{enumerate}
\chapter{Fundamentos teóricos}
\lettrine{E}{n} este capítulo se exponen los conceptos necesarios para comprender la memoria.

\section{Clasificadores}

Los clasificadores o algoritmos de clasificación resuelven la tarea de decidir la membresía de un nuevo dato a una clase a partir de un \gls{dataset}. Se suelen mencionar los siguientes algoritmos en el campo de la detección de objetos y en otros problemas de clasificación \cite{DREISEITL2002352}:

\begin{itemize}
    \item \textbf{\textit{k-Nearest neighbors}}: es un método de clasificación basado en las distancias de la muestra respecto a sus muestras más cercanas (vecinos). El parámetro k establece la cantidad de dichos vecinos, de forma que se asigna la clase predominante de los vecinos a la propia muestra \cite{peterson2009k}.
    \item \textbf{\textit{Decision tree}}: es un método de clasificación que sigue una estructura de árbol, en el que un nodo representa una característica y una rama un conjunto de características. La clasificación empieza desde el nodo raíz y termina en uno de los nodos hoja \cite{ali2012random}.
    \item \textbf{\textit{Artificial Neural Networks}}: es una arquitectura multicapa, compuesta por nodos (o neuronas), cada uno representando a un \textbf{perceptrón}. El perceptrón es un clasificador lineal
    \item \textbf{\textit{Support Vector Machines}}: trata de dibujar el hiperplano óptimo entre clases de datos. A diferencia de los anteriores clasificadores, las predicciones devueltas por las \acrshort{svm} son puramente \textbf{dicotómicas}, es decir, no devuelve probabilidades de pertenencia para cada clase (al menos de forma nativa) \cite{DREISEITL2002352}.
\end{itemize}

En este proyecto, se ha optado por el entrenamiento discriminativo, que consiste en resaltar las diferencias entre las muestras de la clase que se representa (positivos) respecto al resto del mundo (negativos). Debido a que las muestras negativas se pueden compartir entre diferentes clasificadores, se obtiene una gran capacidad de generalización partiendo de un conjunto reducido de datos.

Los clasificadores como KNN o las redes neuronales no poseen la capacidad de aprender de los datos negativos y, en cambio, necesitan grandes \glspl{dataset} para converger correctamente \cite{malisiewicz2011ensemble}. Las \acrshort{svm} sí que cuentan con dicho poder discriminatorio y, por este motivo, son relevantes para el presente proyecto.

\subsection{Support Vector Machine (SVM)}

\begin{figure}[tbp]
    \centering
    \includegraphics[width=0.5\linewidth]{imagenes/LINSVM.png}
    \caption{Clasificación en espacio \acrshort{2d} mediante el uso de las \acrshort{svm}, figura extraída de \cite{svn}.}
    \label{fig:linsvm}
\end{figure}

Es un algoritmo de \textbf{clasificación} que dibuja un hiperplano entre categorías de datos, buscando siempre el mayor margen (distancia entre los puntos que definen las fronteras de las categorías, denominados \textit{support vectors}), de modo que agrega cierta tolerancia al posible ruido producido. En la figura \ref{fig:linsvm} se muestra un ejemplo de clasificación en el espacio \acrshort{2d}, los cuadrados grises representan los \textit{support vectors}, que definen el margen máximo de separación entre clases.

Las \acrshort{svm} requieren de un conjunto inicial de datos para su entrenamiento (aprendizaje supervisado) y no son capaces de discernir entre clases fuera del conjunto de datos de entrenamiento (\textit{Closed-Set}), por lo que un dato desconocido se clasificaría erróneamente como una de las clases del entrenamiento \cite{rudd2017extreme}.

\subsection{Comités}

Los métodos basados en comités (o \textit{ensembles}) se han explorado como alternativa para reducir los errores en la generalización de los clasificadores \cite{zhang2002neural}. También se han estudiado como solución para evitar el borrado del conocimiento previo de los modelos (\textit{catastrophic forgetting}) \cite{Erik, ensembles0, ensembles3}, dentro del campo del aprendizaje incremental.

Dichos métodos se basan en un sistema de votación en el que un conjunto de clasificadores participan, lo que permite tolerar los fallos de un clasificador en particular \cite{zhang2002neural}. En el caso de este proyecto, el comité estaría formado por un conjunto de \acrshort{svm}s. Se ha demostrado que múltiples \acrshort{svm} simples (ejemplo: lineales) como conjunto \textbf{generalizan mejor} que una única \acrshort{svm} compleja (ejemplo: sigmoide) \cite{malisiewicz2011ensemble}.

A continuación se exponen algunos de los tipos más conocidos:

\begin{itemize}
    \item \textbf{\textit{Gradient Boosting}}: Learning rate, number of boosting rounds, maximum tree depth
    \item \textbf{\textit{Random Forest}}: es un conjunto de árboles de clasificación creados a partir de muestras escogidas aleatoriamente del conjunto de entrenamiento. Las predicciones se realizan mediante la votación por mayoría \cite{ali2012random}.
    \item \textbf{\textit{Median Fusion}}:
\end{itemize}

\section{Tipos de aprendizaje incremental}
Como ya se ha introducido en la sección \ref{sec:incrlearning}, el aprendizaje incremental permite añadir nuevo conocimiento al sistema durante su operación o fase de inferencia (\textit{online training}). Esta aproximación permite al sistema adaptarse a nuevos datos que no corresponden a los etiquetados en los \glspl{dataset} utilizados durante su fase de entrenamiento (\textit{offline training}). Dentro de este campo se encuentran los siguientes tipos de configuración:
\begin{itemize}
    \item \textbf{No supervisado}: el sistema se entrena con datos obtenidos durante su operación (muestras o datos \textit{online}), es decir, en \textbf{completa ausencia de datos etiquetados}. El sistema, de forma autónoma, recolecta los datos de los casos de prueba utilizando \textbf{métodos de seguimiento} de los objetos. Este tipo de aprendizaje debe ser capaz de manejar el \textbf{ruido inevitable} de las muestras \textit{online} \cite{sharma2012unsupervised}.
    \item \textbf{Semisupervisado}: el sistema parte de un conjunto \textbf{reducido} de datos etiquetados, que se expanden posteriormente con datos \textit{online}. A partir de los datos etiquetados se logra suprimir fácilmente el ruido en las muestras iniciales, que puede ser un factor crítico en el rendimiento del sistema \cite{andrew}.
\end{itemize}

No se menciona un tipo supervisado, ya que esto supondría no utilizar datos \textit{online} para el aprendizaje, lo que contradice el principio principal del aprendizaje incremental.

Dentro del aprendizaje incremental, los conceptos \textit{Closed-Set}, \textit{Open-Set} y \textit{Open-World} han surgido para etiquetar a los sistemas según el tipo de reconocimiento que implementan. A continuación, se explican los detalles de cada uno.

\subsection{\textit{Closed-Set}}
Un sistema \textit{Closed-Set} opera en un conjunto \textbf{cerrado} de objetos y no implementa ningún tipo de aprendizaje en absoluto. Se entrena puramente con datos etiquetados y está limitado a reconocer las clases con las que se inicializó. Este es el modo en el que trabaja el sistema base \cite{andrew}.

Asumiendo que el sistema no va a reconocer nuevas clases, se estrecha enormemente el universo de clases que este conoce, lo que permite implementar métodos de clasificación basados en la clase más probable \cite{bendale2015towards}.

\subsection{\textit{Open-Set}}
Extiende al modo \textit{Closed-Set}, de forma que puede reconocer nuevas clases nunca vistas en el entrenamiento, o clases \textbf{desconocidas}.

Al ampliar el universo de categorías a la categoría desconocida, deja de ser válida la clasificación por la clase más probable. El reconocimiento de nuevas clases sigue siendo un tópico activo en la investigación, donde se apuesta fuertemente por el teorema \acrfull{evt}.

\subsection{\textit{Open-World}}

Este modo a su vez extiende al modo \textit{Open-Set}. Atendiendo a la definición de \cite{rudd2017extreme}, un sistema de reconocimiento \textit{Open-World} debe de ser capaz de realizar las siguientes 4 tareas:
\begin{itemize}
    \item Detectar desconocidos (\textbf{\textit{Open-Set}}): identificar cuando una muestra de entrada no pertenece al conjunto de datos del entrenamiento.
    \item Escoger las muestras que puedan aportar información del desconocido al sistema.
    \item Etiquetar dichas muestras, por ejemplo, con un número o un pseudónimo.
    \item Actualizar el sistema.
\end{itemize}

En un principio puede resultar natural aplicar los algoritmos de reconocimiento válidos en \textit{Open-Set} al modo \textit{Open-World}. Sin embargo, no todos los algoritmos son adecuados para las actualizaciones incrementales o la escalabilidad en la cantidad de clases \cite{bendale2015towards}.

\chapter{Fundamentos tecnológicos}
\lettrine{E}{n} este capítulo se detallan las métricas de evaluación y herramientas hardware y software utilizadas, así como los conceptos necesarios para comprender el resto de la memoria.

\section{Hardware}
\label{sec:hw}

\subsection{Cámaras RGBD}
Para nutrir al sistema de imágenes se cuenta con las cámaras Kinect de Microsoft. Dichas cámaras otorgan una imagen \acrfull{rgb} más una imagen de profundidad (depth).

Para este proyecto, se cuenta con dos de estas cámaras, cada una con una cobertura de visión de 57º, lo que otorga un rango total de 117º, asumiendo que no existen solapes entre las cámaras.

\subsection{Summit\_XL}
\label{sec:summit}

El sistema que se expondrá se ha construido para funcionar en el robot móvil Summit\_XL. Dicho robot es de tipo diferencial (permite el movimiento en todas las direcciones excepto en los laterales) y cuenta con 4 ruedas de goma, de gran utilidad para moverse en entornos de exteriores. Entre sus componentes, tiene instalado un sensor Velodyne \acrshort{lidar} que otorga nubes de puntos \acrshort{3d} del entorno y dos cámaras Kinect instaladas en los laterales.

\subsection{Dispositivos de altas prestaciones}

\subsubsection{Arquitecturas de GPU}

Según la arquitectura de la \acrshort{gpu} de NVIDIA a tratar se agregan nuevas características que maximizan el rendimiento de la inferencia de los modelos (se ahondará en este tema en el capítulo \ref{chap:cnn}). Las arquitecturas que se muestran a continuación están ordenadas de menor a mayor generación:
\begin{itemize}
    \item Volta: se introducen los \textbf{tensor cores}, que son cores destinados para acelerar multiplicaciones y sumas de matrices (conocido también como MMA). La \textbf{Jetson Xavier NX} sigue esta arquitectura.
    \item Turing: . La \textbf{GeForce GTX 1650} (gráfica del PC lab) sigue esta arquitectura.
    \item Ampere: incluye las siguientes mejoras:
          \begin{itemize}
              \item Incorpora los tensor cores de tercera generación, que incluyen el tipo de dato \textbf{TF32}, que funciona como FP32 pero optimizado en \acrshort{gpu} sin realizar ningún cambio en el código \cite{ampere}
              \item Introduce la funcionalidad \acrfull{mig}, que permite la partición \textbf{física} de \acrshort{gpu}s para correr aplicaciones de CUDA. A la hora de asignar particiones, es necesario comprender los siguientes conceptos \cite{migconcepts}:
                    \begin{itemize}
                        \item \textbf{\acrshort{gpu} Engine}: es la parte de la \acrshort{gpu} que ejecuta el trabajo. Un ejemplo es el copy engine, que se encarga de realizar las operaciones relacionadas con la memoria (direct memory access, memcpy...).
                        \item \textbf{\acrshort{gpu} Memory Slice}: porción de los controladores de memoria y caché de la \acrshort{gpu}.
                        \item \textbf{\acrshort{gpu} SM Slice}: porción de los \glspl{sm} de la \acrshort{gpu}.
                        \item \textbf{\acrshort{gpu} Slice}: porción de la \acrshort{gpu} que combina un único \textbf{\acrshort{gpu} Memory Slice} y un único \textbf{\acrshort{gpu} SM Slice}.
                        \item \textbf{\acrshort{gpu} Instance}: conjunto de \textbf{\acrshort{gpu} slices} y \textbf{\acrshort{gpu} engines}.
                        \item \textbf{Compute Instance}: una o múltiples instancias que pertenecen a una \textbf{\acrshort{gpu} Instance}. Entre ellas se comparten las \textbf{\acrshort{gpu} Memory Slices} y los \textbf{\acrshort{gpu} Engines}, mientras que los \textbf{\acrshort{gpu} SM Slices} se \textbf{particionan}.
                    \end{itemize}
              \item Añade aceleración en las operaciones con matrices \textbf{dispersas} (\textit{sparsity}), que son matrices con una gran proporción de ceros entre sus valores.
          \end{itemize}
          La \textbf{Jetson Orin Nano}, \textbf{Jetson AGX Orin} y la \textbf{GeForce RTX 3050} (gráfica del portátil) siguen esta arquitectura.
    \item Blackwell: se introducen las siguientes mejoras:
          \begin{itemize}
              \item Incorpora los tensor cores de quinta generación, que permiten operar en precisión \textbf{FP4} y \textbf{FP6} (punto flotante de 4 y 6 bits respectivamente) sin apenas pérdida en la precisión, gracias a la incorporación de \textit{micro-tensor scaling}, que aplica diferentes factores de escalado a los elementos de un \gls{tensor} \cite{blackwell}.
          \end{itemize}
          La \textbf{Jetson AGX Thor} sigue esta arquitectura.
\end{itemize}

\begin{figure}[tbp]
    \centering
    \includegraphics[width=0.5\linewidth]{imagenes/mastel.png}
    \caption{Asignación de bits según el tipo de dato, figura extraída de \cite{tf32}}
    \label{fig:types}
\end{figure}

La figura \ref{fig:types} muestra algunas de las precisiones vistas y que se aplicarán para las inferencias de los modelos de este proyecto (a excepción del formato BF16). TODO:

\subsubsection{Familia Jetson}

%A continuación se expone el listado de dispositivos sujetos de las pruebas de este proyecto.

Es una familia de \textbf{\glspl{embsystem}} pensados para aplicaciones de \acrshort{ia}. Dichos dispositivos incluyen un kit de desarrollo llamado \textbf{JetPack}, diseñado para exprimir la potencia de la NVIDIA Jetson en aplicaciones como la robótica, \acrshort{ia} generativa y visión artificial \cite{jason}.

% Please add the following required packages to your document preamble:
% \usepackage{multirow}
\begin{table}[]
    \centering
    \resizebox{\textwidth}{!}{\begin{tabular}{cc|cccc}
            \hline
            \multicolumn{2}{l|}{\multirow{2}{*}{}}             & \multicolumn{4}{c}{\textit{Jetson family}}                                                                                                                                                                                                                                                                                                                                                                                                             \\
            \multicolumn{2}{l|}{}                              & \textbf{\begin{tabular}[c]{@{}c@{}}Xavier \\ NX \cite{jxavier}\end{tabular}} & \textbf{\begin{tabular}[c]{@{}c@{}}Orin \\ Nano \cite{jorin}\end{tabular}}  & \textbf{\begin{tabular}[c]{@{}c@{}}AGX \\ Orin \cite{jorin}\end{tabular}}     & \textbf{\begin{tabular}[c]{@{}c@{}}AGX \\ Thor \cite{jthor}\end{tabular}}                                                                                                                                 \\ \hline
            \multirow{3}{*}{\textit{CPU}}                      & \textit{Cores}                                                               & \begin{tabular}[c]{@{}c@{}}6-core \\ NVIDIA Carmel \\ Armv8.2\end{tabular}  & \begin{tabular}[c]{@{}c@{}}6-core\\ Arm Cortex-A78AE \\ v8.2\end{tabular}     & \begin{tabular}[c]{@{}c@{}}12-core\\ Arm Cortex-A78AE \\ v8.2\end{tabular}                                 & \begin{tabular}[c]{@{}c@{}}14-core \\ Arm Neoverse-V3AE\end{tabular}                         \\ \cline{2-2}
                                                               & \textit{Max freq}                                                            & 1.9 GHz                                                                     & 1.7 GHz                                                                       & 2.2 GHz                                                                                                    & 2.6 GHz                                                                                      \\ \cline{2-2}
                                                               & \textit{Cache}                                                               & \begin{tabular}[c]{@{}c@{}}L2: 6 MB\\ L3: 4MB\end{tabular}                  & \begin{tabular}[c]{@{}c@{}}L2: 1.5MB\\ L3: 4MB\end{tabular}                   & \begin{tabular}[c]{@{}c@{}}L2: 3MB\\ L3: 6MB\end{tabular}                                                  & \begin{tabular}[c]{@{}c@{}}L2: 1MB per core\\ L3: 16MB\end{tabular}                          \\ \hline
            \multirow{4}{*}{\textit{GPU}}                      & \textit{Arch}                                                                & Volta                                                                       & Ampere                                                                        & Ampere                                                                                                     & Blackwell                                                                                    \\ \cline{2-2}
                                                               & \textit{Cores}                                                               & \begin{tabular}[c]{@{}c@{}}384 CUDA cores\\ 48 Tensor cores\end{tabular}    & \begin{tabular}[c]{@{}c@{}}1024 CUDA cores\\ 32 Tensor cores\end{tabular}     & \begin{tabular}[c]{@{}c@{}}2048 CUDA cores\\ 64 Tensor cores\end{tabular}                                  & \begin{tabular}[c]{@{}c@{}}2560 CUDA cores\\ 96 Tensor cores\\ MIG with 10 TPCs\end{tabular} \\ \cline{2-2}
                                                               & \textit{Max freq}                                                            & 1.1 GHz                                                                     & 1.02 GHz                                                                      & 1.3 GHz                                                                                                    & 1.57 GHz                                                                                     \\ \cline{2-2}
                                                               & \textit{\begin{tabular}[c]{@{}c@{}}\#SM \footnotemark[1]\end{tabular}}       & 6                                                                           & 8                                                                             & 16                                                                                                         & 20                                                                                           \\ \hline
            \multicolumn{2}{c|}{\textit{Memory}}               & \begin{tabular}[c]{@{}c@{}}128-bit\\ 8 GB LPDDR4x \\ (59.7GB/s)\end{tabular} & \begin{tabular}[c]{@{}c@{}}128-bit\\ 8 GB LPDDR5 \\ (102 GB/s)\end{tabular} & \begin{tabular}[c]{@{}c@{}}256-bit\\ 64 GB LPDDR5 \\ (204.8GB/s)\end{tabular} & \begin{tabular}[c]{@{}c@{}}256-bit\\ 128 GB LPDDR5X \\ (273 GB/s)\end{tabular}                                                                                                                            \\ \hline
            \multicolumn{2}{c|}{\textit{Network}}              & \begin{tabular}[c]{@{}c@{}}1 RJ45 \\ Gigabit Ethernet\end{tabular}           & \begin{tabular}[c]{@{}c@{}}1 RJ45 \\ Gigabit Ethernet\end{tabular}          & \begin{tabular}[c]{@{}c@{}}1 RJ45 \\ 10 Gigabit Ethernet\end{tabular}         & \begin{tabular}[c]{@{}c@{}}1 RJ45 \\ 5 Gigabit Ethernet\\ 1 QSFP28 \\ 2 x 25 Gigabit Ethernet\end{tabular}                                                                                                \\ \hline
            \multicolumn{2}{c|}{\textit{Power}}                & 10-20 W                                                                      & 7-25 W                                                                      & 15-60 W                                                                       & 40-130 W                                                                                                                                                                                                  \\ \hline
            \multicolumn{2}{c|}{\textit{AI performance}}       & 21 TOPS                                                                      & 67 TOPS                                                                     & 275 TOPS                                                                      & \begin{tabular}[c]{@{}c@{}}2070 TFLOPS \\ (FP4—Sparse)\end{tabular}                                                                                                                                       \\ \hline
            \multicolumn{2}{c|}{\textit{Price (Jan 20, 2026)}} & 399\$                                                                        & 249\$                                                                       & 1999\$                                                                        & 3499\$                                                                                                                                                                                                    \\ \hline
        \end{tabular}}
    \caption{Tabla de dispositivos Jetson del proyecto}
    \label{tab:jetsons}
\end{table}
\footnotetext[1]{Extraído de la información del sistema}

Actualmente existe una amplia variedad en la potencia y precio de estos dispositivos \cite{jason}. Para este proyecto, se han empleado cuatro modelos diferentes que se exponen en la tabla \ref{tab:jetsons} (se encuentran ordenados de menores a mayores prestaciones).

La Jetson AGX Thor soporta \acrshort{mig}, que como ya se ha comentado, divide la \acrshort{gpu} en varias particiones físicas, de forma que se puede asignar un conjunto de recursos exclusivos a una aplicación de CUDA, sin interferencias de otras aplicaciones. La Jetson AGX Thor tiene la siguiente configuración (información extraída de los comandos \textbf{nvidia-smi mig}):
\begin{itemize}
    \item 1 \textbf{\acrshort{gpu} Instance}: denominada \textbf{3g.0gb}, donde 0gb indica el tamaño de la memoria compartida en \acrshort{gb}s (TODO: debido a que la memoria de la \acrshort{gpu} es compartida con la \acrshort{cpu}, la \acrshort{gpu} no la particiona), y 3g, que indica 3 \acrshort{gpu} SM Slices (TODO: que equivalen a 3 \gls{sm}). Está compuesta por las siguientes \textbf{Compute Instances}:
          \begin{itemize}
              \item \textbf{1c.3g.0gb}:
                    \begin{itemize}
                        \item Instancias disponibles: 2 (TODO: 2 instancias de 3 \glspl{sm} cada una)
                        \item \gls{sm} dedicados: 6
                    \end{itemize}
              \item \textbf{2c.3g.0gb}:
                    \begin{itemize}
                        \item Instancias disponibles: 1 (TODO: 1 instancia de 6 \glspl{sm} cada una)
                        \item \gls{sm} dedicados: 8
                    \end{itemize}
              \item \textbf{3c.3g.0gb}
          \end{itemize}
\end{itemize}

Desafortunadamente, el software que controla el \acrshort{mig} en la Jetson no funciona adecuadamente, ya que no permite la reserva de las \textbf{Compute Instances}. Este error está relacionado con Jetpack 7.0 \cite{notworking}, por lo que instalar Jetpack 7.1 \textbf{podría} habilitar el \acrshort{mig} en la Jetson AGX Thor.

\subsubsection{Familia Intel}

% Please add the following required packages to your document preamble:
% \usepackage{multirow}
% Please add the following required packages to your document preamble:
% \usepackage{multirow}
\begin{table}[]
    \resizebox{\textwidth}{!}{\begin{tabular}{cc|ccc}
            \hline
            \multicolumn{2}{c|}{\multirow{2}{*}{}} & \multicolumn{3}{c}{\textit{Desktop}}                                                                                                                                                                                                                                                                                                                                                                                        \\
            \multicolumn{2}{c|}{}                  & \textbf{PC Lab}                      & \textbf{Laptop \textbackslash{}cite\{asuspender\}}                                                                                  & \textbf{Reference \textbackslash{}cite\{andrew\}}                                                                                                                                                                                              \\ \hline
            \multirow{3}{*}{\textit{CPU}}          & \textit{Cores}                       & \begin{tabular}[c]{@{}c@{}}8 performance-cores\\ 4 efficient-cores\\ Intel Core i7-12700 \textbackslash{}cite\{12700\}\end{tabular} & \begin{tabular}[c]{@{}c@{}}6 performance-cores\\ 4 efficient-cores\\ Intel Core i7-12650H \textbackslash{}cite\{12650H\}\end{tabular} & \begin{tabular}[c]{@{}c@{}}4 cores \\ Intel Core i5-2500K\\ \textbackslash{}cite\{2500K\}\end{tabular} \\ \cline{2-2}
                                                   & \textit{Freq}                        & \begin{tabular}[c]{@{}c@{}}Performance: 4.9 GHz\\ Efficient: 3.6 GHz\end{tabular}                                                   & \begin{tabular}[c]{@{}c@{}}Performance: 4.7 GHz\\ Efficient: 3.6 GHz\end{tabular}                                                     & \begin{tabular}[c]{@{}c@{}}Turbo: 3.7 GHz\\ Base: 3.3 GHz\end{tabular}                                 \\ \cline{2-2}
                                                   & \textit{Cache}                       & \begin{tabular}[c]{@{}c@{}}L2 per core: 1.25 MB\\ L3 shared: 25 MB\end{tabular}                                                     & \begin{tabular}[c]{@{}c@{}}L2 per core: 1.25 MB\\ L3 shared: 24 MB\end{tabular}                                                       & \begin{tabular}[c]{@{}c@{}}L2 per core: 256 KB\\ L3 shared: 6 MB\end{tabular}                          \\ \cline{2-2}
                                                   & \textit{TDP}                         &                                                                                                                                     &                                                                                                                                       & 95 W                                                                                                   \\ \hline
            \textit{GPU}                           & \textit{Device}                      & GeForce GTX 1650 Laptop \textbackslash{}cite\{1650\}                                                                                & GeForce RTX 3050 Laptop \textbackslash{}cite\{3050\}                                                                                  & Intel HD Graphics 3000 \textbackslash{}cite\{IHG3000\}                                                 \\ \cline{2-2}
                                                   & \textit{Cores}                       & 896 CUDA cores                                                                                                                      & 2048 CUDA cores                                                                                                                       &                                                                                                        \\ \cline{2-2}
                                                   & \textit{Max Freq}                    &                                                                                                                                     & 1.74 GHz                                                                                                                              & 1.1 GHz                                                                                                \\ \cline{2-2}
                                                   & \textit{Mem}                         &                                                                                                                                     & 4 GB GDDR6                                                                                                                            & \begin{tabular}[c]{@{}c@{}}Shared with CPU\\ (integrated)\end{tabular}                                 \\ \hline
            \multicolumn{2}{c|}{\textit{Memory}}   & 33.4 GB                              & 16 GB DDR4                                                                                                                          & 16 GB DDR3                                                                                                                                                                                                                                     \\ \hline
            \multicolumn{2}{c|}{\textit{Network}}  & Gigabit Ethernet                     & Gigabit Ethernet                                                                                                                    &                                                                                                                                                                                                                                                \\ \hline
            \multicolumn{2}{c|}{\textit{OS}}       & Ubuntu 24.04                         & Linux Mint 21.3                                                                                                                     & Arch Linux 6.9.3-arch1-1                                                                                                                                                                                                                       \\ \hline
            \multicolumn{2}{c|}{\textit{Power}}    &                                      &                                                                                                                                     &                                                                                                                                                                                                                                                \\ \hline
            \multicolumn{2}{c|}{\textit{Price}}    &                                      & 1299€                                                                                                                               &                                                                                                                                                                                                                                                \\ \hline
        \end{tabular}}
    \caption{Tabla de equipos Intel}
    \label{tab:x86}
\end{table}

La tabla \ref{tab:x86} muestra los equipos con \acrshort{cpu}s Intel empleados en el proyecto. La columna \textit{Reference} muestra el equipo de pruebas utilizado en \cite{andrew} para realizar comparaciones de resultados.


\section{Software específico}
\label{sec:sw}

\subsection{ROS}
\label{sec:ros}

\acrfull{ros} es un middleware de \textbf{código abierto} que incorpora las herramientas necesarias para la interacción y desarrollo de software en robots. Cuenta con una amplia biblioteca de distribuciones y librerías.

Cualquier entorno de desarrollo de \acrshort{ros} está compuesto por \textbf{paquetes}, los paquetes son las unidades funcionales del entorno de \acrshort{ros} y estas contienen el código, archivos de configuración y de lanzamiento de los nodos, entre otros componentes.
\subsubsection{Arquitectura de la red}

\acrshort{ros} crea y mantiene una \textbf{red distribuida} en la que los nodos se ejecutan e intercambian mensajes. Sigue un patrón \textit{publish/subscribe}, en el que existen nodos que publican información por medio de \textbf{tópicos}, disponibles para cualquier nodo que se suscriba a dicho tópico. Según la versión, la arquitectura de la red varía en una de las siguientes:

\begin{figure}[tbp]
    \centering
    \includegraphics[width=0.5\linewidth]{imagenes/mastel.png}
    \caption{Arquitectura maestro-esclavo, figura extraída de \cite{mastel}}
    \label{fig:mastel}
\end{figure}

\begin{figure}[tbp]
    \centering
    \includegraphics[width=0.5\linewidth]{imagenes/dds.jpg}
    \caption{Topología bus de Data Distribution Service (DDS), figura extraída de \cite{bus}}
    \label{fig:mastel}
\end{figure}

\begin{itemize}
    \item Maestro-esclavo (\acrshort{ros} 1): el maestro (denominado \acrshort{ros}\_MASTER) registra a todos los publicadores/subscriptores de la red, de forma que si un nodo quiere suscribirse/publicar un tópico, este se comunica con el maestro para registrar/conocer la manera de conectarse a dicho tópico \cite{mastel}. El principal inconveniente es que el nodo central es un \textbf{punto único de fallo} (si el maestro falla, todo el sistema se cae). La figura \ref{fig:mastel} describe el comportamiento descrito.
    \item Data Distribution Service (\acrshort{ros} 2): es un \textit{middleware} que permite la comunicación de los nodos mediante una topología tipo bus (figura ). Los nodos notifican su presencia en la red y esperan mensajes de respuesta con información sobre el resto de nodos existentes \cite{dds}. De esta forma, todos los nodos conocen el estado de la red sin depender de un maestro que gestione todas las comunicaciones.
\end{itemize}

Existe una amplia variedad de distribuciones de este software según la versión del \acrshort{so}. Para este proyecto, se han empleado las distribuciones Noetic, Humble y Rolling, para Ubuntu 20.04, 22.04 y 24.04 respectivamente, el Summit\_XL tiene la distribución Melodic, ya que tiene la versión 18.04 de Ubuntu. Melodic y Noetic corresponden a ROS 1 (descontinuado), mientras que Humble y Rolling pertenecen a ROS 2.

\subsubsection{TODO: lanzamiento de los threads}

\subsubsection{Rosbag}

Existe una herramienta del ecosistema \acrshort{ros} que graba y reproduce los datos capturados de la ejecución de un sistema \acrshort{ros} en un momento concreto en el tiempo. Los \glspl{dataset} creados en \cite{andrew} se construyeron partiendo de un fichero rosbag, que contiene toda la información generada por los diferentes sensores. En este proyecto se emplearán dichos ficheros rosbag para reproducir las pruebas en tiempo real.

\subsection{Optimizadores y \textit{backends} de inferencias}

\textbf{TensorRT}
Es un software de \textbf{código abierto} desarrollado por NVIDIA para la optimización de inferencias de modelos de IA en aceleradoras de NVIDIA \cite{x36}. Permite ejecutar modelos entrenados en \textit{frameworks} como Pytorch o TensorFlow, aunque se aconseja exportarlos al formato \acrfull{onnx} previamente, para aprovechar al completo las funcionalidades de este software. Se ha empleado TensorRT para optimizar la mayoría de modelos \acrshort{cnn} tanto en equipos \acrshort{arm} como x86. TensorRT versión 10 está disponible para las aceleradoras de NVIDIA con un \textit{compute capability} igual o superior a 7.5. En el caso de querer usar el software con una \textit{compute capability} inferior, es necesario instalar versiones anteriores descontinuadas (ejemplo: versión 8). El código generado en la versión 8 de la API \textbf{no es compatible} con la versión 10.

\textbf{OpenVINO}
Es un software de \textbf{código abierto} desarrollado por Intel para la optimización de inferencias en \acrshort{cpu} (ARM,x86) y en aceleradoras de Intel (\acrshort{gpu},\acrshort{npu}) \cite{OpenVINO}. Permite la conversión directa de modelos entrenados en frameworks como Tensorflow y Pytorch a ficheros XML, que son versiones optimizadas de las \acrshort{cnn} y que se utilizan para realizar las inferencias. Se ha empleado para reducir la latencia de modelos a la hora de ejecutarlos en \acrshort{cpu}.
\subsection{Librerías de Python}
\textbf{PyCUDA}
Librería de Python de \textbf{código abierto} \cite{pycuda} que interacciona con el driver de CUDA por medio de un \gls{wrapper}. Se ha utilizado para programar los códigos de inferencia de los modelos que utilizan TensorRT.
\textbf{NumPy}
Librería de Python de \textbf{código abierto} usada en este proyecto para representar imágenes y la información generada por los modelos en forma de vectores y matrices multidimensionales. Estos elementos pueden manipularse por medio de una amplia cantidad de funciones matemáticas que dicha librería ofrece.

\textbf{Scipy}
Librería de Python de \textbf{código abierto} que implementa numerosos algoritmos relacionados con la computación científica. En este proyecto se ha usado para aplicar el método húngaro, hallar los parámetros que modelan una distribución de Weibull, calcular distancias coseno, entre muchas otras funcionalidades que dicho software ofrece.

\textbf{OpenCV}
La librería de \textbf{código abierto} por excelencia para visión por computadora de código abierto. Otorga herramientas para la obtención y procesado de las imágenes, además de un módulo de visión artificial (dnn), que incluye modelos ya entrenados de visión artificial. Dicha librería tiene implementación en CUDA (si se compila el código fuente \cite{OpenCVCUDA}.), lo que permite ejecutar operaciones y kernels de modelos en aceleradoras de NVIDIA (ejemplo: NVIDIA Jetson).

\subsection{Docker}
Es un software de virtualización de \textbf{código abierto}. Permite realizar despliegues automatizados mediante ficheros de configuración, el software se ejecuta en un entorno aislado del sistema operativo llamado \textbf{contenedor}, dichos contenedores proporcionan seguridad y portabilidad. Esta herramienta ha sido de gran utilidad para desplegar el software del proyecto en los diferentes dispositivos.

\section{Software general}
\subsection{VS Code}
\subsection{Git}
\subsection{Trello}
\subsection{Draw.io}
\chapter{Arquitectura del sistema propuesto}
\label{chap:sysarch}
\lettrine{E}{n} este capítulo se comenta la arquitectura  del sistema propuesto y de los nuevos componentes: procesamiento de vídeo, reconocimiento adaptativo y personas registradas.

\section{Arquitectura general}
\label{sec:partarch}

Para entender mejor la arquitectura general, se empezará por explicar la arquitectura base de la que se parte y del nodo cámara, el componente más importante. A continuación, se explicarán las mejoras de la arquitectura final extendida.

\subsection{Arquitectura base}
\label{sec:basearch}

\begin{figure}[tbp]
    \centering
    \includegraphics[width=0.45\linewidth]{imagenes/SYSARCH.jpg}
    \caption[Arquitectura del sistema base de partida]{Arquitectura del sistema base de partida, figura extraída de \cite{andrew}}
    \label{fig:sysarch}
\end{figure}

Como ya se ha comentado en la introducción, en este proyecto se parte de un sistema diseñado para el robot móvil Summit\_XL, compuesto por dos cámaras \acrshort{rgbd} y un sensor \acrshort{lidar}, que \textbf{detecta y reconoce personas} a partir de nodos que aplican los respectivos \textbf{modelos de redes neuronales convolucionales} (o \acrshort{cnn}).

En la figura \ref{fig:sysarch} se muestra la arquitectura planteada en \cite{andrew}, compuesta por un nodo que recibe \glspl{pcl} \acrshort{3d} del entorno y otorga las posiciones \acrshort{3d} de las personas detectadas mediante el respectivo modelo (nodo \acrshort{lidar}) y 1 o varios nodos que detectan y reconocen personas mediante los datos \acrshort{rgb} y obtienen su posición \acrshort{3d} mediante la imagen de distancias de las cámaras. La información generada por los nodos \textbf{se fusiona} y se devuelve en forma de una \textbf{lista final de todos los individuos reconocidos y su posición \acrshort{3d}} (nodo integración de sensores). El sistema opera \textbf{frame a frame}, es decir, devuelve la información de los individuos en cada frame capturado por las cámaras.

Toda la arquitectura se ejecuta en el \textbf{\textit{framework} \acrshort{ros}} \cite{ROS} (sección \ref{sec:ros}). \acrshort{ros} se encarga de crear los procesos para cada nodo, comprobar su estado, regular la frecuencia a la que trabajan, crear la red en la que dichos procesos intercambian mensajes, entre otros muchos detalles que resultan transparentes para el programador.

\subsection{Nodo cámara}

\begin{figure}[tbp]
    \centering
    \includegraphics[width=0.8\linewidth]{imagenes/CAMNOD.jpg}
    \caption[Arquitectura del nodo cámara]{Arquitectura del nodo cámara, figura extraída de \cite{andrew}}
    \label{fig:camnod}
\end{figure}

En la figura \ref{fig:camnod} se muestra el flujo del nodo cámara. Dicho nodo sigue un enfoque \textbf{multimodal}, es decir, utiliza las \textbf{características} tanto \textbf{faciales} como \textbf{corporales} del individuo para su detección y reconocimiento.

Tras recibir el frame \acrshort{rgb}, este es procesado por \textbf{dos modelos de detección}, uno destinado a rostros (YuNet en la figura \ref{fig:camnod}) y el otro a cuerpos (YOLO en la figura \ref{fig:camnod}), de los que se obtienen las \glspl{bbox} (o recortes) de los rostros y cuerpos detectados, que se comparan para asegurar que cada recorte facial se encuentra contenido en cada recorte corporal. Posteriormente, se procede al \textbf{reconocimiento facial} (ArcFace) y corporal (OSNet), que devuelven los respectivos \glspl{embedding} de los que se computa la \textbf{distancia coseno} (otorga el grado de similitud entre dos vectores considerando el ángulo entre ellos) con todos los \glspl{embedding} (sujetos) guardados en la base de datos, de forma que se obtiene una lista ordenada de los sujetos \textbf{según el grado de similitud}.

La identidad del individuo puede determinarse como la \textbf{primera entrada} de la lista (es decir, la identidad \textit{a priori} más parecida) o puede procesarse y obtener la identidad por medio de algoritmos más avanzados (en \cite{andrew} se proponen los métodos de distancia relativa e identidad probable).

Finalmente, el nodo cámara devuelve la identidad facial y corporal de los individuos, junto a su posición \acrshort{3d}, calculada a partir de la imagen de distancias y las coordenadas de las \glspl{bbox} corporales.

%, que otorga el grado de similitud entre dos vectores \cite{andrew}

\subsection{Arquitectura extendida}
\label{sec:finalsys}

\afterpage{
    \begin{landscape}
        \begin{figure}[tbp]
            \centering
            \includegraphics[width=1.1\linewidth]{imagenes/FINALSYS.jpg}
            \caption{Arquitectura del sistema extendido, los módulos en negrita se corresponden con los nuevos componentes desarrollados.}
            \label{fig:finalsys}
        \end{figure}
    \end{landscape}
}

En la figura \ref{fig:finalsys} se muestra la arquitectura extendida, los nuevos componentes desarrollados se encuentran marcados en negrita. %, mientras que el resto de la arquitectura sigue la estructura ya comentada. 
Debido a que el nodo \acrshort{lidar} no está dentro del foco de este proyecto y tampoco se pudo realizar una migración satisfactoria a \acrshort{ros} 2 del mismo (ver sección \ref{subsec:ROS2}), se ha decidido \textbf{omitirlo} del nuevo sistema extendido. En concreto, se han propuesto los siguientes componentes:

\begin{description}
    \item[Procesamiento de vídeo] Implementa el procesamiento de \textbf{secuencias de frames} (o vídeos, en sustitución del esquema frame a frame), con el fin de explotar la coherencia espacio-temporal. Dicho módulo agrupa las \glspl{bbox} por individuos en más de un frame, que posteriormente se transforman en \glspl{embedding} y se guardan en una lista, que sirve de entrada para el módulo de \textbf{reconocimiento adaptativo}.
    \item[Reconocimiento adaptativo] Implementa la capacidad de detección e inclusión de desconocidos en el sistema (modo \textbf{\textit{Open-Set} y \textit{Open-World}}), en sustitución del reconocimiento mediante la \textbf{distancia coseno}. Evalúa%, mediante el uso de la teoría estadística (\acrshort{evt} introducido en la sección \ref{sec:incrlearning}), 
        (mediante el \acrshort{evt} introducido en la sección \ref{sec:incrlearning}) si la \textbf{mejor puntuación} obtenida de un individuo es un caso \textbf{extremo} respecto al resto de puntos (personas) de la distribución. En caso afirmativo, se trata de un conocido, en caso contrario, de un desconocido. En ambos casos, el sistema actualiza su conocimiento acerca del individuo por medio de la base de datos de \textbf{personas registradas}.
    \item[Personas registradas] representa al conocimiento existente acerca de los individuos, que es compartido por \textbf{todo el sistema}, es decir, por todos los nodos cámara. Este módulo también abarca la inicialización de dicho registro al arrancar el sistema y los distintos métodos para su compartición entre nodos.
\end{description}

Las \textbf{puntuaciones} utilizadas por el reconocimiento adaptativo no son más que los \textbf{resultados de los comités de \acrshort{svm}} sobre las secuencias de \glspl{embedding} de cada individuo (en la sección \ref{sec:val} se indaga en este asunto). En la sección \ref{sec:clasf} se ha justificado la elección de los comités de \acrshort{svm} frente a otros clasificadores por su capacidad de generalización y poder discriminativo.

%A continuación, se exponen en detalle los componentes recién comentados.

\section{Procesamiento de vídeo}
\label{sec:vídeo}
El sistema de partida trabaja a nivel de frame \cite{andrew}, es decir, devuelve predicciones de los individuos presentes en una sola imagen. Esta aproximación permite trabajar a altas frecuencias (ejemplo: devolver un reconocimiento cada 100 ms), sin embargo, las predicciones dependen enteramente de la calidad del frame (ejemplo: nivel de borrosidad). En este proyecto se ha optado por trabajar con \textbf{secuencias de frames} (o vídeos), de esta forma, se devuelve un reconocimiento más robusto basado en múltiples frames.

En cada frame, se extraen las \glspl{bbox} de los individuos presentes a partir de los modelos de detección de caras y cuerpos, que posteriormente se asocian entre el \textbf{frame actual y el anterior}, hasta establecer una secuencia completa. El resultado final es una lista (lista de entidades en la figura \ref{fig:finalsys}), que contiene listas de \glspl{bbox} transformadas en \glspl{embedding} (aplicando los modelos de reconocimiento) para cada individuo. El tamaño de la secuencia es ajustable según las necesidades del operador (ejemplo: secuencia de 10 frames, que equivale a 1 segundo si las cámaras funcionan a 10 Hz).

\begin{figure}
    \centering
    \includegraphics[width=0.8\linewidth]{imagenes/Seq.jpg}
    \caption{Nuestra aplicación del método húngaro para el seguimiento de personas.}
    \label{fig:seq}
\end{figure}

En casos como los del \gls{dataset} FACE COX \cite{cox}, donde en los vídeos siempre aparece una sola persona, la agrupación de las entidades es trivial. Sin embargo, en un vídeo donde aparecen múltiples individuos que se entrecruzan, es necesario adoptar un método para \textbf{seguir el rastro de cada persona entre frames}. Se ha aplicado el \textbf{método húngaro} (introducido en la sección \ref{sec:trackerwork}) para relacionar las \glspl{bbox} según su \textbf{grado de solape} en dos frames adyacentes, en este caso, el \textbf{frame actual y el anterior}. Se genera la matriz de costes, donde las \glspl{bbox} (o recortes) del frame anterior se encuentran en las filas y las \glspl{bbox} del frame actual en las columnas y se devuelven los pares de \textbf{menor coste}. Tras repetir el proceso en toda la secuencia, se obtiene la lista de recortes de cada persona según el rastro generado por el algoritmo. La figura \ref{fig:seq} muestra un ejemplo de funcionamiento del proceso comentado.

Los costes de cada par de la matriz se obtienen mediante el \textbf{\acrfull{iou}}, que devuelve el porcentaje de solapamiento y similitud entre \glspl{bbox} (como se muestra en la figura \ref{fig:iou}), de forma que recortes de diferente tamaño den un valor bajo aunque el solape sea alto (ejemplo: personas que coinciden en la imagen en diferentes profundidades). El resultado devuelto por el \acrshort{iou} \textbf{se invierte} (1-\acrshort{iou}), de forma que se asocia el \textbf{menor coste} a las \glspl{bbox} con \textbf{mayor solape y similitud}. Dadas dos \glspl{bbox} A y B, el \acrshort{iou} se calcula como sigue en la ecuación \ref{eq:iou}.

\begin{figure}
    \centering
    \includegraphics[width=0.5\linewidth]{imagenes/iou.jpg}
    \caption[Ejemplo de \emph{Intersection over Union}]{Ejemplo de \emph{Intersection over Union}, figura extraída de \cite{iou}}
    \label{fig:iou}
\end{figure}

\begin{equation}
    IoU = \frac{A \cap B}{A \cup B}\label{eq:iou}
\end{equation}

%Problema, si ocurre que 2 bboxes de un individuo no se solapan, y si se solapa otra bbox, por mínimo solape que sea (ejemplo:0.99) ya se realiza la asignación incorrecta.

Es posible que durante la detección, las \glspl{bbox} de una misma persona en frames consecutivos no se solapen debido a la velocidad de movimiento de la propia persona o a un movimiento de la cámara, lo que causa una \textbf{incorrecta} aplicación del método húngaro. En este caso, se asigna el \textbf{mayor coste} (valor de 1) a la relación (es decir, un valor de \acrshort{iou} de 0), lo que genera el riesgo de asociar \textbf{2 \glspl{bbox} de individuos diferentes}. Otro problema es el \textbf{no seguimiento} de la persona cuando se encuentra totalmente \textbf{ocluida}  y reaparece (ejemplo: se cruza un individuo justo delante) o si la persona gira su cabeza y su cara queda momentáneamente fuera de la visión de la cámara.

\begin{equation}
    \Delta r_{euclid} = \sqrt{\Delta x^{2} + \Delta y^{2}} \label{eq:eucl}
\end{equation}

En estos casos, se crearía una nueva entidad para el mismo individuo, lo que no es un comportamiento deseable. Por este motivo, es necesario aplicar un método que pueda \textbf{reidentificar a las personas} cuyo rastro se haya perdido temporalmente. Como ya se comentó en la sección \ref{sec:trackerwork}, el \textbf{filtro de Kalman} como método de reidentificación \textbf{no es beneficioso} para la aplicación de este proyecto, por lo que se ha optado por un método más sencillo basado en la \textbf{distancia euclidiana}. Siendo $\Delta x$ y $\Delta y$ la diferencia entre las coordenadas \acrshort{2d} de los \textbf{centros} de dos \glspl{bbox}, la distancia euclidiana se calcula como se muestra en la ecuación \ref{eq:eucl}.

\begin{figure}[tbp]
    \centering
    \includegraphics[width=1\linewidth]{imagenes/FOLREID.jpg}
    \caption{Diagrama de flujo del procesamiento de vídeo}
    \label{fig:reident}
\end{figure}

La figura \ref{fig:reident} muestra el diagrama de flujo del procesamiento de vídeo. Se itera la lista de entidades detectadas en la secuencia, que se contrasta con la lista de asignaciones del método húngaro entre dos frames. A partir de este punto, el flujo se divide en dos grandes ramas:

\begin{itemize}
    \item \textbf{La entidad posee asignación.} En este caso, si dicha asignación \textbf{no obtiene} el mayor coste, \textbf{se valida la asociación}, en caso contrario (las \glspl{bbox} del individuo no se solapan), se calcula la distancia entre el \textbf{centro} de las \glspl{bbox} con la asignación, si la distancia es \textbf{inferior a un valor de umbral} (las \glspl{bbox} son cercanas) \textbf{se valida la asociación}, en caso contrario, se ha \textbf{perdido el rastro del individuo}, por lo que se guardan las coordenadas de su \gls{bbox} para la \textbf{reidentificación}.
    \item \textbf{La entidad no posee asignación.} En este caso, se calcula la distancia entre las coordenadas guardadas y \textbf{cada \gls{bbox} sin asignación} en el frame actual, si se obtiene una distancia inferior al valor del umbral, \textbf{se asocian las \glspl{bbox}}, en caso contrario, se reanudará la reidentificación de la entidad en el \textbf{siguiente frame}.
\end{itemize}

El valor del umbral se fija de antemano \textbf{y se ajusta automáticamente en función de la resolución} de la cámara utilizando un escalado a partir de la \textbf{diagonal}. %(se calcula la diagonal de la resolución prefijada y la nueva resolución y se divide la diagonal de la resolución nueva entre la prefijada, el resultado se multiplica por el valor de umbral).

Finalmente, se añaden las nuevas entidades del frame actual (las que no han conseguido asignación por ninguno de los métodos) a la lista de entidades para ser tenidas en cuenta en futuros frames. A modo de regular el funcionamiento de la reidentificación, la localización de un individuo se abandona tras realizar un \textbf{máximo de intentos} controlado por una variable asignada a cada entidad (intentos\_loc en la figura \ref{fig:reident}). El valor de la variable se restablece si se reidentifica al individuo satisfactoriamente antes de alcanzar el límite.

\begin{figure}[tbp]
    \centering
    \begin{subfigure}[c]{0.18\textwidth}
        \includegraphics[width=\textwidth]{imagenes/eucl0.jpg}
    \end{subfigure}
    \hspace{2em}
    \begin{subfigure}[c]{0.18\textwidth}
        \includegraphics[width=\textwidth]{imagenes/eucl1.jpg}
    \end{subfigure}
    \hspace{2em}
    \begin{subfigure}[c]{0.18\textwidth}
        \includegraphics[width=\textwidth]{imagenes/eucl2.jpg}
    \end{subfigure}
    \hspace{2em}
    \begin{subfigure}[c]{0.18\textwidth}
        \includegraphics[width=\textwidth]{imagenes/eucl3.jpg}
    \end{subfigure}
    \caption{Reidentificación mediante la distancia euclidiana}
    \label{fig:eucls}
\end{figure}

En la figura \ref{fig:eucls} se expone un ejemplo real. En el primer frame, se muestran dos entidades etiquetadas con un identificador (0 y 1) y como se aplica la distancia euclidiana para la entidad 0, cuyas \glspl{bbox} no se solapan. En el segundo frame (figura \ref{fig:eucls}), la entidad 1 no encuentra ninguna asignación posible, debido a que va a ser ocluida, por lo tanto la posición de su última \gls{bbox} se guarda. En el tercer frame (figura \ref{fig:eucls}), la entidad 1 se encuentra totalmente ocluida por la entidad 0. Como en el instante posterior a dicho frame existe una \gls{bbox} sin ninguna asociación, se calcula la distancia euclidiana entre dicha \gls{bbox} respecto a la guardada de la entidad 1. Finalmente, la entidad 1 se reasigna en el último frame, debido a que la distancia calculada es inferior al umbral (figura \ref{fig:eucls}).

%Se aplica el método húngaro para asociar detecciones entre el frame anterior y el actual. Se itera la lista de entidades con los individuos detectados hasta el momento, si algún individuo posee la asignación de mayor coste o no se ha establecido, se aplica el método de reidentificación durante un máximo de frames, controlado por la variable intentos\_loc. Por cada intento fallido de localización, se decrementa dicha variable, así hasta llegar al 0, que es el punto en el que se descarta la localización. En el resto de casos, la asignación del frame anterior con el actual se realiza y se restablece de nuevo la variable de intentos (en el caso de reidentificación).

%Al final de cada iteración del módulo, se añaden las nuevas detecciones a la lista de entidades. Los nuevos casos se corresponden con asignaciones de \gls{bbox} que no se corresponden con ninguna entidad antes registrada.

\section{Reconocimiento adaptativo}
\label{sec:archrecon}

\begin{figure}[tbp]
    \centering
    \includegraphics[width=1\linewidth]{imagenes/ADAPTSYS.jpg}
    \caption{Diseño del módulo de reconocimiento adaptativo}
    \label{fig:ADAPTSYS}
\end{figure}

La figura \ref{fig:ADAPTSYS} muestra los componentes del módulo de reconocimiento adaptativo:
\begin{description}
    \item[Módulo de valoración] Devuelve las \textbf{puntuaciones mínimas} de cada comité a la secuencia de \glspl{embedding} de entrada.
    \item[Módulo de reconocimiento] Toma la decisión de reconocimiento basándose en la distribución de puntuaciones mínimas de todos los comités.
    \item[Módulo de actualización] Se encarga de crear un nuevo comité si se detecta un usuario nuevo (\textit{unknown}) o de crear una nueva \acrshort{svm} para registrar los cambios de un usuario ya registrado en el sistema (\textit{drift}).
    \item[Módulo de limitación] Reemplaza la \acrshort{svm} que menos aporta a un comité, en el caso de que se haya excedido el límite de \acrshort{svm}s en un comité.
\end{description}

La figura \ref{fig:ADAPTSYS} muestra el diseño del reconocimiento adaptativo. La secuencia de vectores del individuo son procesados por el \textbf{módulo de valoración} (módulo EDF en la figura \ref{fig:ADAPTSYS}), que devuelve las \textbf{puntuaciones mínimas} de cada comité ordenadas de menor a mayor. Así la primera puntuación (la mejor) se corresponde \textit{a priori} con el comité del individuo, mientras que el resto de puntuaciones conforman la distribución de mínimos. A partir de las puntuaciones el \textbf{módulo de reconocimiento} (módulo RDF en la figura \ref{fig:ADAPTSYS}) determina, mediante el \acrfull{evt}, si el comité corresponde realmente a un individuo (conocido) o no (desconocido). El \textbf{módulo de actualización} registra los cambios según la decisión de reconocimiento del individuo:

\begin{itemize}
    \item \textbf{El individuo es un conocido.} Se agrega al comité una nueva \acrshort{svm} entrenada con la secuencia de \glspl{embedding} del individuo como positivos y una selección aleatoria de las muestras de todos los individuos registrados \textbf{excepto el propio sujeto} como negativos.
    \item \textbf{El individuo es un desconocido.} Se crea un \textbf{nuevo comité} con una \acrshort{svm} entrenada con la secuencia de \glspl{embedding} del individuo como positivos y una selección aleatoria de las muestras de \textbf{todos los individuos registrados} como negativos.
\end{itemize}

Si el comité excede en el límite fijado de \acrshort{svm}s, el \textbf{módulo de limitación} determina la \acrshort{svm} que menos aporta a dicho comité y se \textbf{elimina}. %Sendos módulos de actualización y limitación se encargan de añadir, actualizar y eliminar las \acrshort{svm}, los comités y las muestras de los individuos a la base de datos de \textbf{personas registradas}.

Para facilitar la lectura de la memoria y, puesto que no se incluyen mejoras respecto a los diseños propuestos en \cite{Erik,CESAR}, los detalles de este módulo se han movido al apéndice \ref{chap:adaptrecon}.

\section{Personas registradas}
\label{sec:initarch}

El sistema crea y mantiene una base de datos con todas las personas registradas, que es compartida por \textbf{todos el sistema} y está formada por los comités, sus \acrshort{svm} y las muestras de cada individuo. Esta base de datos es actualizada por los módulos de actualización y limitación del reconocimiento adaptativo. A continuación se explican los diferentes procesos de \textbf{inicialización} propuestos, así como diferentes técnicas para compartir la información entre los nodos del sistema.

\subsection{Inicialización}
\label{sec:initieing}

%fundamental, debido a que la \acrshort{svm} inicial es la que define la identidad del comité. Si dicha \acrshort{svm} está compuesta por muestras de baja calidad (ejemplo: caras borrosas o parcialmente ocluidas), entonces el comité \textbf{no se identificará correctamente consigo mismo} y, por lo tanto, generará \textbf{mayor confusión} a la hora de aplicar Weibull, es decir, \textbf{se generarán más desconocidos}.

En esta primera fase del sistema se crean los registros de los individuos iniciales. Como se muestra en \cite{5samples}, el proceso de inicialización es \textbf{crítico} y depende de la calidad de las muestras escogidas. Se han propuesto las siguientes 2 aproximaciones acordes a los tipos de aprendizaje incremental vistos en la sección \ref{sec:incrtypes}.

\begin{description}
    \item[Semisupervisada] El operador realiza una selección de las muestras más representativas de cada individuo, de los que se obtienen las características (\glspl{embedding}) a partir de los modelos de reconocimiento. Una vez inicializada la base de datos, el sistema deja de recibir datos etiquetados para entrenarse con datos no etiquetados durante su operación.
    \item[No supervisada] El sistema se encarga de recoger las muestras en el instante en el que aparece un mínimo de entidades simultáneas en escena. A partir de los frames de las cámaras, que estarán sincronizadas, se obtienen las \glspl{bbox} por medio de los modelos de detección y se agrupan por individuo mediante el módulo de procesamiento de vídeo. Este método \textbf{no requiere de ninguna intervención por parte del operador}.
\end{description}

\begin{figure}
    \centering
    \includegraphics[width=1\linewidth]{imagenes/Init.jpg}
    \caption{Inicialización no supervisada del sistema}
    \label{fig:init}
\end{figure}


En la figura \ref{fig:init} se muestra el procedimiento de la inicialización no supervisada. Dado un número de cámaras \textbf{sincronizadas}, se busca un instante (o frame) en el que mínimo se detecten \textbf{5 personas} (equivalente al mínimo de puntos de la distribución para que el \acrshort{evt} sea efectivo \cite{Erik}) de forma simultánea entre todas las cámaras. En dicho instante, se aplica el método de procesamiento de vídeo para organizar las \glspl{bbox} por individuo, si no se obtienen suficientes \glspl{bbox} del sujeto, el sistema \textbf{repite el proceso de búsqueda} en otro instante. En caso contrario, se obtienen los \glspl{embedding} y se almacenan para la creación de las \acrshort{svm}. Con el fin de evitar las muestras de \textbf{mala calidad}, las \glspl{bbox} se \textbf{filtran} y se descartan si no cumplen con las siguientes condiciones:
\begin{itemize}
    \item La cara se encuentra enteramente dentro del plano
    \item La \gls{bbox} es más alta que ancha.
\end{itemize}

%Tras obtener los \glspl{embedding} por medio de uno de los 2 modos de inicialización, se crean los comités de cada usuario con una \textbf{\acrshort{svm} inicial}, que se entrena a partir de las propias muestras del individuo (conjunto de positivos) y una selección de muestras del resto de individuos (conjunto de negativos). Finalmente, los comités se añaden a la base de datos de \textbf{personas registradas}.

\subsection{Registro compartido}
\label{sec:shared}

En el sistema de partida (sección \ref{sec:partarch}), el registro inicial de individuos \textbf{no se modifica} (modo \textit{Closed-Set}), por lo tanto, no existe la necesidad de mantener una fuente centralizada de los datos y, en su lugar, cada nodo inicializa su propia copia. Para implementar el modo \textit{Open-Set} y \textit{Open-World}, es necesario \textbf{propagar los cambios} del módulo de reconocimiento adaptativo a \textbf{todos los nodos} del sistema. Debido a que el sistema es distribuido (computación en múltiples nodos) y los nodos están implementados en Python, es necesario explorar otras vías diferentes a la memoria compartida entre procesos. Se han propuesto las siguientes 2 alternativas:

\begin{description}
    \item[Base de datos centralizada] Todos los nodos acceden a una base de datos de baja latencia (ejemplo: base de datos puramente en memoria), de la que reciben el registro de personas actualizado. Dicha base de datos se encontraría en la misma máquina que el nodo integración de sensores, que realizaría las peticiones de escritura. Tras una petición de escritura, la base de datos propaga las modificaciones a todos los nodos conectados.
    \item[Red \acrshort{ros}] El nodo integración de sensores recibe los mensajes de los nodos cámara, de los que extraerá los nuevos cambios a la base de datos. Aprovechando la red creada por \acrshort{ros}, el nodo integrador difunde un mensaje con las nuevas modificaciones a un tópico en el que todos los nodos cámara estarán suscritos, de forma que puedan recibir y aplicar los cambios, manteniendo así su propia copia actualizada.
\end{description}

En ambos casos, el nodo integrador realiza un volcado periódico de la base de datos en un fichero a modo de preservar los cambios.

\begin{table}[tbp]
    \begin{tabular}{l|l|l}
        \hline
        \multicolumn{1}{c|}{\textit{\begin{tabular}[c]{@{}c@{}}Métodos de\\ compartición\end{tabular}}} & \multicolumn{1}{c|}{\textbf{Ventajas}}                                                                                                                                                             & \multicolumn{1}{c}{\textbf{Desventajas}}                                                                                                                          \\ \hline
        \textit{\begin{tabular}[c]{@{}l@{}}Base de datos\\ centralizada\\ (ej: Redis)\end{tabular}}     & \begin{tabular}[c]{@{}l@{}}-Permite realizar copias de \\ seguridad\\ -Es altamente tolerante a fallos\\ -Mayor escalabilidad\\ -Múltiples opciones para la\\ protección de los datos\end{tabular} & \begin{tabular}[c]{@{}l@{}}-Se agrega la latencia en la \\ escritura/lectura a mayores de\\ la latencia en la red\\ -Consume mucha memoria principal\end{tabular} \\ \hline
        \textit{Red ROS}                                                                                & \begin{tabular}[c]{@{}l@{}}-Lecturas y escrituras rápidas\\ (memoria de Python)\\ -Permite intercambiar mayor\\ variedad de estructuras de datos\end{tabular}                                      & \begin{tabular}[c]{@{}l@{}}-Si el nodo integrador falla, se\\ pierden los cambios. \\ -No permite realizar copias de\\ seguridad\end{tabular}                     \\ \hline
    \end{tabular}
    \caption{Ventajas y desventajas de los 2 métodos de compartición propuestos.}
    \label{tab:redvsros}
\end{table}

En la tabla \ref{tab:redvsros} se muestran las ventajas y desventajas de los 2 métodos. La base de datos centralizada (ejemplo: Redis \cite{redis}) es la solución adecuada para implementar un sistema escalable, sin embargo, la latencia añadida en las lecturas/escrituras y el elevado uso de la memoria la hacen \textbf{inabarcable} para sistemas embebidos de pocos recursos (ejemplo: Jetson Orin Nano). Por el otro lado, la red \acrshort{ros} suprime las latencias de lectura y escritura y se comporta adecuadamente con un número moderado de individuos, por lo que para este proyecto se vuelve \textbf{más conveniente}, sobretodo para dispositivos como la Jetson Orin Nano.
\chapter{Implementación del sistema}
\label{chap:impl}

\lettrine{E}{n} este capítulo se ahonda en los detalles de implementación de los nuevos componentes, así como los cambios realizados en el propio sistema base y los detalles de la migración a \acrshort{ros} 2. También se exponen los cambios en las inferencias de los modelos para su ejecución en TensorRT.

Los nuevos componentes \textbf{no se han integrado en el sistema base}, sino que se han desarrollado y probado de forma \textbf{independiente} (en el capítulo \ref{chap:systesting} se prueban dichos componentes), por lo tanto no se ha implementado la \textbf{arquitectura extendida} (sección \ref{sec:finalsys}) ni ninguno de los métodos de compartición vistos en la sección \ref{sec:shared}. Se ha omitido comentar el módulo de personas registradas, puesto a que su lógica no entraña nada destacable. Adicionalmente, la implementación del reconocimiento adaptativo se encuentra en el apéndice \ref{sec:adaptimpel}.

\section{Procesamiento de vídeo}
La implementación sigue el flujo expuesto en la figura \ref{fig:reident} (sección \ref{sec:vídeo}). El cálculo del \acrshort{iou} es sencillo y no entraña apenas complejidad computacional, se calcula la intersección de las 2 \glspl{bbox} a partir de sus coordenadas, mientras que la unión se obtiene a partir de la suma de las áreas que encierran las 2 \glspl{bbox} \textbf{menos} la intersección. Finalmente, se calcula el \acrshort{iou} según la ecuación \ref{eq:iou} (sección \ref{sec:vídeo}). La distancia euclidiana se calcula como se expone en la sección \ref{sec:vídeo} y tampoco entraña mayor dificultad.

Se utilizó la función \textit{linear\_sum\_assignment} de la librería SciPy \cite{lsa} como implementación del \textbf{método húngaro}. Dicha librería implementa el algoritmo Jonker-Volgenan, que es una variante del método húngaro con complejidad computacional \textbf{$O(n^{3})$} \cite{lsa, o3}, por lo que se vuelve impracticable para matrices en el orden de miles de elementos (ejemplo: detección de 100 personas de una cámara en un frame y su anterior) \cite{325}. Sin embargo, como en los vídeos de prueba una cámara llega a detectar como máximo 4 personas a la vez, la ejecución del algoritmo no supone apenas algún coste.

%\section{Personas registradas}
%\label{sec:init}

%En esta sección se comentan las implementaciones de los componentes encargados de crear la base de datos de las personas registradas del sistema.

\section{Creación y entrenamiento de las SVM}
\label{sec:training}

Para este proyecto, se ha utilizado la implementación de las \acrshort{svm} ofrecida por la librería de OpenCV (cv2). El código \ref{coud:training} corresponde al proceso seguido a la hora de crear y entrenar nuevas \acrshort{svm}. Si el individuo es un desconocido, entonces todas las muestras de la base de datos sirven como datos \textbf{negativos}, en caso contrario, es necesario \textbf{excluir las muestras del propio sujeto} del conjunto de negativos, de lo contrario se estaría perdiendo el poder discriminativo de las \acrshort{svm}. En el entrenamiento, se asigna la \textbf{puntuación} 1 (variable \textit{plabels}) para las muestras positivas (variable \textit{d1}) y -1 (variable \textit{nlabels}) para las negativas (variable \textit{neg\_samples}). OpenCV asigna los signos a las clases según el orden de estos junto a las etiquetas que se presentan en el entrenamiento. En el caso de presentar primero las muestras negativas con -1 y las positivas con +1, entonces la \acrshort{svm} devuelve como máximo -1 cuando una muestra pertenece a la clase positiva y como máximo un +1 para las clases negativas. Se ha aplicado este comportamiento debido a que se utiliza la \textbf{mediana} (que ordena los valores de menor a mayor) para obtener el mínimo de las puntuaciones. En cambio, las \acrshort{svm} podrían entrenarse para causar el efecto contrario, solo se requiere de cambiar el orden de aparición de las etiquetas y los signos en el entrenamiento.

%\section{Método de inicialización no supervisado}

%Como ya se explicó en la sección \ref{sec:initarch}, en el método de inicialización no supervisado se crean los comités de las nuevas entidades únicamente de la información disponible en el momento de ejecución del sistema. Para la implementación del algoritmo de la figura \ref{fig:init} se ha podido aprovechar el módulo de procesamiento de vídeo para la fase de recolección de \glspl{bbox}. El resto de detalles de implementación no entrañan nada destacable.

\section{Migración a ROS 2}
\label{subsec:ROS2}

Con el motivo del fin de soporte de \acrshort{ros} 1 \cite{ROSEOL}, se ha optado por migrar el sistema para ser ejecutado en \acrshort{ros} 2 con el fin de mantener su continuidad. El proceso de migración se ha llevado a cabo por medio de la guía oficial de \acrshort{ros} \cite{ros2tuto}. Muchos de los problemas encontrados en este proceso están relacionados con los nuevos archivos de configuración y de lanzamiento del sistema (cambio del formato \acrshort{xml} a Python de este último \cite{lanch}).

Se han migrado los nodos \textbf{cámara e integrador}. El proceso incluye cambiar las firmas de las funciones, sus parámetros y cambiar de paquetes. En esencia, la mayoría de paquetes de \acrshort{ros} 2 preservan las mismas funcionalidades e incluso mantienen las mismas interfaces de las funciones de \acrshort{ros} 1, lo que ha facilitado dicho proceso. La migración del nodo \acrshort{lidar} \textbf{no ha sido posible}, debido a que el código depende de versiones de librerías no soportadas a partir de la versión 22.04 de Ubuntu y de varios componentes de \acrshort{ros} 1 eliminados en \acrshort{ros} 2. Sería necesario rediseñar todo el código o optar por un paquete de \acrshort{ros} 2 con las mismas funcionalidades que el paquete \textit{hdl\_people\_tracking}, utilizado en \cite{andrew} para detectar personas en nubes de puntos \acrshort{3d}.

\section{Escalabilidad y tolerancia a fallos de las cámaras}
\label{sec:scal}

Todos los nodos del sistema requieren estar \textbf{sincronizados} dentro de un intervalo temporal (ejemplo: 100 milisegundos), de modo que el nodo integración de sensores no fusione información \textbf{desactualizada}. En \cite{andrew} se utiliza \textit{ApproximateTimeSynchronizer} del paquete \textit{message\_filters} de \acrshort{ros}, que implementa un algoritmo adaptativo para emparejar mensajes a partir de su \gls{timestamp} \cite{ApproximateTime}. El problema de \textit{ApproximateTimeSynchronizer} es que espera recibir datos \textbf{de todas las fuentes en todo momento}, de modo que si uno de los sensores falla, el sistema \textbf{se congela}. Como alternativa a la anterior solución, se ha propuesto que el nodo integrador \textbf{mantenga una caché} por cada nodo del que reciba mensajes, que pueden recuperarse especificando un \gls{timestamp}, de forma que si un nodo no publica a tiempo su mensaje, el integrador sigue recuperando su último mensaje en la caché dentro de un límite de \textbf{200 milisegundos} (tras superar dicho límite, el mensaje se descarta). Las cachés se implementan por medio de la clase \textit{MessageFiltersCache} del mismo paquete de \acrshort{ros} \cite{MFC}, que actúan de suscriptores de los tópicos de cada nodo.

El código \ref{coud:cash} implementa una función (\textit{process}) que recoge los datos de las cachés a una frecuencia fija (ejemplo: 10 Hz), de modo que se recuperan los últimos mensajes de los nodos que emitieron dentro del intervalo, sin esperar por nodos que hayan sufrido latencias o fallos. Todos los mensajes se guardan en una lista (variable \textit{message\_list}) con la que se llama a la fusión de los resultados (función \textit{on\_frame}) para devolver las detecciones finales.

Por otro lado, el sistema original no había sido diseñado para gestionar las detecciones provenientes de cámaras con rangos de visión \textbf{parcialmente solapados} (en las pruebas del capítulo \ref{chap:syscal} se trata este caso, puesto que se duplican literalmente las cámaras), de forma que si más de 2 cámaras detectan a la misma persona, dicha detección \textbf{se descarta en ambas cámaras}, lo que es una decisión coherente cuando se cuenta con una configuración con cámaras \textbf{sin solape}. Se ha modificado el código para evitar que en dicha situación se eliminen las detecciones y considerar solo una de ellas en el nodo integración de sensores.

\section{Códigos de inferencia para el runtime de TensorRT}
\label{sec:trtinfer}

En este proyecto se han implementado nuevas versiones de las inferencias adaptadas para su uso en \acrshort{gpu} por medio de TensorRT y PyCUDA. PyCUDA ofrece una \acrshort{api} que facilita gran parte de la interacción con \gls{CUDA}, aun así, el programador necesita gestionar temas como la creación de contextos y reserva de la memoria.

%TensorRT trabaja con su propio formato de los modelos, por lo que es necesario realizar una conversión de los mismos a dicho formato mediante las herramientas expuestas en el apéndice \ref{chap:procedure}. En el proceso de conversión se aplica un proceso de selección de optimizaciones según la arquitectura de la \acrshort{gpu}, que prueba distintas tácticas para la distribución del trabajo y diferentes precisiones para quedarse con la combinación más rápida, sin comprometer la precisión del modelo.

El código \ref{coud:trtskel} muestra el esqueleto de dichas inferencias, cada vez que se crea una instancia del modelo en Python (invocación a \textit{\_\_init\_\_}), se reservan los siguientes recursos:
\begin{description}
    \item[\gls{CUDA} context] Entorno en el que se ejecutan los \glspl{stream} de \gls{CUDA}. A nivel del sistema, se reserva un contexto por nodo cámara, que es compartido a su vez por todos sus modelos.
    \item[\gls{CUDA} \gls{stream}] Cola de operaciones que utilizará la \acrshort{gpu} para realizar las transferencias de memoria y ejecutar los \glspl{kernel} de las inferencias. A nivel de sistema, se reserva un \gls{stream} para cada modelo.
    \item[Execution context] Contexto para ejecutar las inferencias a partir de un \textbf{CudaEngine} (modelo deserializado). Se pueden crear múltiples contextos de ejecución para un mismo \textbf{CudaEngine}.
    \item[Device input] Memoria del \gls{tensor} de entrada reservado en la \acrshort{gpu} mediante un \textbf{mem\_alloc}. Se especifica el tamaño en bytes del \gls{tensor} que representa los datos de entrada.
    \item[Device output] Memoria del \gls{tensor} de salida reservado en la \acrshort{gpu} mediante un \textbf{mem\_alloc}. Se especifica el tamaño en bytes del \gls{tensor} que representa los datos de salida.
    \item [Bindings] Lista de direcciones de memoria de los inputs/outputs de la inferencia. Aplica a uno o a varios \textbf{execution contexts}.
\end{description}

La función \textit{infer} del código \ref{coud:trtskel} se invoca por cada vez que se recibe una \textbf{imagen} (o un \gls{batch} de imágenes) a procesar. Se ejecuta la función PREPROCESSING con la entrada, que representa el pipeline (conjunto de operaciones) de preprocesado específico de la imagen para cada modelo. Los datos preprocesados se copian a la memoria de la \acrshort{gpu} (o \textit{device}) a través de un \textit{memcpy} y se realiza la inferencia en el \textbf{execution context} por medio del \textit{\gls{stream}} de \gls{CUDA} reservado. Antes de la inferencia, se especifica el tamaño concreto de los datos de entrada (función \textit{set\_input\_shape}), ya que en un procesamiento en modo \gls{batch}, el conjunto de imágenes puede diferir. Finalmente, los resultados se copian de vuelta a la memoria de la \acrshort{cpu} (o \textit{host}) y se ejecuta el pipeline de postprocesado del resultado definido en la función POSTPROCESSING. Las operaciones de copia de datos y la inferencia se ejecutan de forma \textbf{asíncrona}, por lo que es necesario introducir una barrera de sincronización \textbf{a nivel de \gls{stream}} una vez todas las operaciones se emitan, de forma que la próxima llamada a \textit{infer} no sobrescriba la memoria de la \acrshort{gpu} mientras esta sigue realizando la inferencia anterior.
\chapter{Análisis de rendimiento}
\label{chap:cnn}

\lettrine{E}{n} este capítulo se comentan los modelos de redes neuronales utilizadas y se expone una discusión detallada acerca de las pruebas y de los resultados de rendimiento de los modelos en los diferentes dispositivos del proyecto.

\section{Modelos de redes neuronales utilizadas}
\label{sec:models}

A continuación se exponen las \acrshort{cnn} utilizadas en los nodos cámara, que también se emplearon en \cite{andrew}:

\begin{description}
    \item[YuNet] Es un modelo de detección facial diseñado para sistemas con recursos muy limitados, con tan solo \textbf{75856 parámetros} \cite{wu2023yunet}. Debido a los múltiples \textit{outputs} del modelo, complejos de procesar, se ha optado por utilizar la versión de OpenCV, que simplifica todo el proceso de pre y postprocesado. El modelo en formato \acrshort{onnx} se encuentra en este enlace \footnote{https://github.com/opencv/opencv\_zoo/blob/main/models/face\_detection\_yunet/face\_detection\_yunet\_2023mar.onnx}.
    \item[ArcFace] \textit{Additive Angular Margin Loss} o ArcFace es una función de pérdida (\gls{lossfunc}) que mejora el poder discriminativo de los modelos de reconocimiento facial, es decir, maximiza el margen que separa a las clases y minimiza el margen de los datos que pertenecen a una misma clase \cite{deng2019arcface}. Los \glspl{embedding} con los que opera ArcFace pertenecen al modelo ResNet-100. Se emplea el archivo de pesos de una implementación en TensorFlow Lite, accesible mediante el siguiente enlace \footnote{https://www.digidow.eu/f/datasets/arcface-tensorflowlite/model.tflite}.
    \item[YOLO] \textit{You Only Look Once} (YOLO) es la red más rápida y simple de detección de objetos en tiempo real, que logra mantener una buena precisión \cite{redmon2016you, mdpi}. Existen múltiples versiones de YOLO, en \cite{andrew} se probaron las versiones 3, 5 y 8 en el tamaño nano, más ligero y veloz, pero al mismo tiempo más impreciso. En este proyecto a mayores se ha probado la versión 11, la última existente a día de hoy, en su tamaño nano. El modelo se puede obtener mediante la \acrshort{api} de Python de ultralytics, que permite realizar exportaciones con diferentes ajustes (ver sección \ref{sec:exports}).
    \item[OSNet] \textit{Omni-Scale Network} (OSNet) es un modelo de reidentificación de personas que emplea características de diferentes tamaños (\textit{omni-scale features}) para la clasificación. Destaca por ser extremadamente ligero y por su elevada precisión \cite{zhou2019omni}. Se realizó una exportación a partir del modelo implementado en Pytorch (\cite{andrew}).
\end{description}

\section{Dataset de las pruebas}
\label{sec:modset}

A modo de evaluar el rendimiento en términos de precisión de los modelos de visión artificial, se parte de un \gls{dataset} compuesto de la información extraída de un fichero \gls{rosbag}, que contiene los datos generados por las cámaras Kinect \acrshort{rgbd} y el sensor \acrshort{lidar} del Summit\_XL durante la realización de una prueba en el laboratorio de robótica del \acrshort{citic}, en el que 6 personas caminan alrededor del robot y se entrecruzan durante 105 segundos (aproximadamente). El robot también se desplaza en su propio eje, generando así imágenes borrosas y cambios impredecibles en la posición de las personas.

Existe un fichero \acrshort{json} con el \gls{dataset} específico para cada cámara. En el apéndice \ref{sec:datcam} se expone su composición.

\section{Realización de las pruebas}
\label{sec:probes}

Para la evaluación del rendimiento, se parten de unos tests que reproducen los videos de las dos cámaras (ver sección \ref{sec:hw}) y alimenta a los modelos de redes neuronales con los respectivos frames, de forma que al momento de terminar el procesamiento de un frame, el modelo puede continuar inmediatamente con el siguiente. Las salidas devueltas por las \acrshort{cnn} se contrastan con la información del \textit{\gls{dataset}} de la respectiva cámara (visto en la anterior sección), que devuelve un resultado final de precisión.

Se realiza una medición de la latencia de la inferencia, en el caso de los modelos de detección, y de la inferencia más el reconocimiento en el caso de los modelos de reconocimiento. En ambos casos, se emplea la función \textbf{time} del módulo \textbf{time} de Python para obtener los tiempos en cada frame. Finalmente, todas las latencias se fusionan a partir de la media.

En las pruebas de los modelos de detección, se obtienen los frames directamente de los videos. El test determina una detección como coincidente si los porcentajes de intersección entre la \gls{bbox} obtenida respecto a la de referencia y viceversa superan o igualan un umbral. En este caso, dicho umbral se ha fijado en el 50\% \cite{andrew}.

En el caso de los modelos de reconocimiento, se recupera para cada frame una lista de recortes generados por los modelos de detección \textbf{previamente a la ejecución de la prueba}. El test determina un reconocimiento como correcto si la predicción obtenida es la misma que la etiqueta (\textit{\gls{gt}}) recogida en el \gls{dataset}. Como se ha comentado en la sección \ref{sec:basearch}, existen varios métodos para escoger la identidad ganadora dentro de un conjunto de identidades probables, por simplicidad, siempre se escoge la primera predicción de la lista como la ganadora.

Para la ejecución de inferencias, se han empleado los \glspl{rt} de OpenVINO, TensorRT y OpenCV, este último para el caso concreto de la red YuNet. OpenVINO ejecuta las inferencias \textbf{puramente en \acrshort{cpu}}, mientras que TensorRT realiza las inferencias en \textbf{\acrshort{gpu}}. OpenCV permite la ejecución en el propio \gls{rt} que ofrece, que opera \textbf{puramente en \acrshort{cpu}} o permite realizar la inferencia por medio de CUDA, que aplica la \acrshort{gpu}.

Es importante destacar que los \glspl{rt} de OpenVINO y TensorRT soportan precisión mixta, es decir, se ejecutan las capas de un modelo en la precisión que menor latencia reporte dentro de una lista de precisiones (ejemplo: \acrshort{fp}16 o \acrshort{fp}32) \cite{mixed}. Esta funcionalidad puede emplearse siempre que la arquitectura de los dispositivos la implemente. Las últimas versiones de OpenCV también implementan soporte para ambas precisiones \acrshort{fp}32 y \acrshort{fp}16. Los modelos del proyecto se han compilado para utilizar la precisión mixta en \acrshort{fp}16 o \acrshort{fp}32.

\section{Resultados}
\label{sec:modelic}

%%%%%%%%%%%%%%%%%%%%%%%%%%%%%%%%%%%%%%%%%%%%%%
%%%% Tabla de modelos de todos los equipos %%%
%%%%%%%%%%%%%%%%%%%%%%%%%%%%%%%%%%%%%%%%%%%%%%
\begin{landscape}
    \centering
    % Please add the following required packages to your document preamble:
    % \usepackage{multirow}
    \begin{table}[]
        \begin{tabular}{lll|ll|llll}
            \hline
                                                                                                                       &                                                         &                                                                 & \multicolumn{2}{c|}{\textit{Desktop}}                                          & \multicolumn{4}{c}{\textit{Jetson embedded systems}}                                                                                                                                                                                                                                                               \\
                                                                                                                       &                                                         &                                                                 & \multicolumn{1}{c}{\textbf{\begin{tabular}[c]{@{}c@{}}Lab \\ PC\end{tabular}}} & \multicolumn{1}{c|}{\textbf{Laptop}}                 & \textbf{\begin{tabular}[c]{@{}l@{}}Xavier \\ NX\end{tabular}} & \textbf{\begin{tabular}[c]{@{}l@{}}Orin \\ Nano\end{tabular}} & \textbf{\begin{tabular}[c]{@{}l@{}}AGX \\ Orin\end{tabular}} & \textbf{\begin{tabular}[c]{@{}l@{}}AGX \\ Thor\end{tabular}} \\ \hline
            \multicolumn{1}{c}{\multirow{4}{*}{\textit{\begin{tabular}[c]{@{}c@{}}YOLO11n \\ (640x640)\end{tabular}}}} & \multicolumn{1}{l|}{\multirow{2}{*}{\textit{OpenVINO}}} & \textit{\begin{tabular}[c]{@{}l@{}}Latency\\ (ms)\end{tabular}} & 14.7                                                                           & 29.5                                                 & 146.3                                                         & 77.9                                                          &                                                              &                                                              \\ \cline{3-3}
            \multicolumn{1}{c}{}                                                                                       & \multicolumn{1}{l|}{}                                   & \textit{F1\_score}                                              & 88.1                                                                           & 88.1                                                 & 88                                                            & 88                                                            &                                                              &                                                              \\ \cline{2-9}
            \multicolumn{1}{c}{}                                                                                       & \multicolumn{1}{l|}{\multirow{2}{*}{\textit{TensorRT}}} & \textit{\begin{tabular}[c]{@{}l@{}}Latency\\ (ms)\end{tabular}} & 5.9                                                                            & 5.7                                                  & 28.7                                                          & 22.4                                                          &                                                              &                                                              \\ \cline{3-3}
            \multicolumn{1}{c}{}                                                                                       & \multicolumn{1}{l|}{}                                   & \textit{F1\_score}                                              & 88.9                                                                           & 88.9                                                 & 87.7                                                          & 89.1                                                          &                                                              &                                                              \\ \hline
            \multirow{4}{*}{\textit{\begin{tabular}[c]{@{}l@{}}YuNet*\\ (480x480)\end{tabular}}}                       & \multicolumn{1}{l|}{\multirow{2}{*}{\textit{OpenVINO}}} & \textit{\begin{tabular}[c]{@{}l@{}}Latency\\ (ms)\end{tabular}} & 3.1                                                                            & 3.8                                                  & 33.5                                                          & 14.9                                                          &                                                              &                                                              \\ \cline{3-3}
                                                                                                                       & \multicolumn{1}{l|}{}                                   & \textit{F1\_score}                                              & 88.1                                                                           & 88.1                                                 & 87.9                                                          & 87.9                                                          &                                                              &                                                              \\ \cline{2-9}
                                                                                                                       & \multicolumn{1}{l|}{\multirow{2}{*}{\textit{TensorRT}}} & \textit{\begin{tabular}[c]{@{}l@{}}Latency\\ (ms)\end{tabular}} & 2.4                                                                            & 3.2                                                  & 14.5                                                          & 11.8                                                          &                                                              &                                                              \\ \cline{3-3}
                                                                                                                       & \multicolumn{1}{l|}{}                                   & \textit{F1\_score}                                              & 88.1                                                                           & 88.9                                                 & 87.9                                                          & 87.9                                                          &                                                              &                                                              \\ \hline
            \multirow{4}{*}{\textit{\begin{tabular}[c]{@{}l@{}}ArcFace \\ (112x112)\end{tabular}}}                     & \multicolumn{1}{l|}{\multirow{2}{*}{\textit{OpenVINO}}} & \textit{\begin{tabular}[c]{@{}l@{}}Latency\\ (ms)\end{tabular}} & 5.6                                                                            & 7.3                                                  & 23.5                                                          & 15.9                                                          &                                                              &                                                              \\ \cline{3-3}
                                                                                                                       & \multicolumn{1}{l|}{}                                   & \textit{F1\_score}                                              & 44.7                                                                           & 44.7                                                 & 44.6                                                          & 44.6                                                          &                                                              &                                                              \\ \cline{2-9}
                                                                                                                       & \multicolumn{1}{l|}{\multirow{2}{*}{\textit{TensorRT}}} & \textit{\begin{tabular}[c]{@{}l@{}}Latency\\ (ms)\end{tabular}} & 1.7                                                                            & 1.3                                                  & 7                                                             & 4.8                                                           &                                                              &                                                              \\ \cline{3-3}
                                                                                                                       & \multicolumn{1}{l|}{}                                   & \textit{F1\_score}                                              & 44.6                                                                           & 44.6                                                 & 45                                                            & 45.2                                                          &                                                              &                                                              \\ \hline
            \multirow{4}{*}{\textit{\begin{tabular}[c]{@{}l@{}}OSNet\_x1 \\ (256x128)\end{tabular}}}                   & \multicolumn{1}{l|}{\multirow{2}{*}{\textit{OpenVINO}}} & \textit{\begin{tabular}[c]{@{}l@{}}Latency\\ (ms)\end{tabular}} & 4.5                                                                            & 5.9                                                  & 45.1                                                          & 21.7                                                          &                                                              &                                                              \\ \cline{3-3}
                                                                                                                       & \multicolumn{1}{l|}{}                                   & \textit{F1\_score}                                              & 58.9                                                                           & 58.9                                                 & 54.3                                                          & 59.4                                                          &                                                              &                                                              \\ \cline{2-9}
                                                                                                                       & \multicolumn{1}{l|}{\multirow{2}{*}{\textit{TensorRT}}} & \textit{\begin{tabular}[c]{@{}l@{}}Latency\\ (ms)\end{tabular}} & 2.7                                                                            & 2.3                                                  & 13.4                                                          & 10.1                                                          &                                                              &                                                              \\ \cline{3-9}
                                                                                                                       & \multicolumn{1}{l|}{}                                   & \textit{F1\_score}                                              & 58.9                                                                           & 58.9                                                 & 59.2                                                          & 59.2                                                          &                                                              &                                                              \\ \hline
        \end{tabular}
        \caption{Resultados de modelos entre dispositivos.
            \newline* CPU=\textbf{OpenCV en CPU}; GPU=\textbf{OpenCV CUDA}}
    \end{table}
    \label{tab:equips}
\end{landscape}

En la tabla \ref{tab:equips} se muestran los resultados de latencia (en milisegundos) y de la métrica \textbf{\textit{F1\_score}} para los modelos de detección y \textbf{\textit{precision}} para los modelos de reconocimiento (ver métricas en el apéndice \ref{sec:modeleval}). Cada resultado refleja la media y la desviación típica de \textbf{5 ejecuciones}, se ejecuta un test previo a modo de calentamiento para cada modelo.

En el caso del modelo YOLO, los 2 equipos de sobremesa y la Jetson AGX Thor se quedan por debajo de 30 ms en la versión de OpenVINO, mientras que la latencia en el resto de dispositivos es significativamente mayor. Es interesante como entre el portátil y el PC del laboratorio hay \textbf{8 milisegundos de diferencia} a favor del último, siendo ambas \acrshort{cpu}s de la misma generación (Intel Core i7).

Al trasladar la carga de la inferencia de YOLO a la \acrshort{gpu} cuando se utiliza TensorRT, el tiempo de inferencia es hasta \textbf{7 veces} menor en el caso de la Jetson AGX Thor y \textbf{5 veces} menor en el caso de la Jetson Xavier NX. Las Jetson Xavier NX y Orin Nano logran alcanzar los \textbf{35 y 45 \acrshort{fps}} respectivamente, lo que hace posible la ejecución en tiempo real (fijada a 10 \acrshort{fps}) en estos dispositivos.

Un punto a favor para la AGX Thor es haber obtenido la menor latencia al ejecutar YOLO11n en \acrshort{gpu}, estando notablemente por debajo de los equipos de sobremesa en la versión de OpenVINO. Gran parte de la diferencia en la latencia viene de la implementación del pre y postprocesado de YOLO, que resulta muy costoso y se realiza en la \acrshort{cpu}, que es donde los equipos de sobremesa destacan. Respecto al postprocesado, se ha optimizado para aprovechar los arrays y operaciones de NumPy (ver apéndice \ref{sec:postyolo}), de esta forma ha sido posible recortar varios milisegundos en todos los equipos.

%TODO: En la tabla \ref{tab:equips} se muestran los resultados obtenidos en 2 equipos x86, comparándolos con la Jetson Orin Nano, el sistema embebido con el que se ha obtenido el mejor rendimiento. En equipos x86, la latencia experimentada  es hasta \textbf{5.3 veces menor} (YOLOv8n) para los modelos ejecutados en CPU y hasta \textbf{4.9 veces menor} (YuNet) para los modelos ejecutados en GPU que la obtenida por la Jetson Orin Nano. Debido a que OpenVINO está pensado para hardware de Intel, las optimizaciones en la Jetson no suponen ninguna mejora, al contrario que en los equipos x86 con una CPU \textbf{Intel i7}. Por otro lado, es sorprendente que también existan diferencias significativas respecto a la Jetson \textbf{cuando se realizan las inferencias en GPU}. La diferencia más notoria es con YuNet en el PC del laboratorio, que posee una \textbf{GeForce GTX 1650}, que está \textbf{una generación por detrás} de la GPU de la Jetson Orin (las arquitecturas son Turing y Ampere respectivamente) \cite{archs}. Esto implica que, para la carga de trabajo de este proyecto, \textbf{la GPU no es el factor predominante} (TODO: no lo creo, no tiene mucho sentido si digo que se reduce la latencia hasta 7 veces usando la GPU, y que la AGX Thor es la que mejor ejecuta YOLO cuando se utiliza la GPU).

En el caso de YuNet, el uso de CUDA como \gls{rt} supone un salto menor respecto a TensorRT, ya que OpenCV integrado con CUDA no aprovecha todos los componentes de la arquitectura de la \acrshort{gpu}, como los tensor cores. Excepto en las Jetson Xavier NX y AGX Thor, el resto de dispositivos no llegan a recortar ni 1 milisegundo o, en el caso de la AGX Orin, incluso llegan a aumentar la latencia. También es importante considerar que YuNet está compuesto por una arquitectura extremadamente ligera (ver tabla \ref{tab:modelspecs}), por lo que no hay mucho margen de mejora en la latencia de inferencia, en cambio, el postprocesado entraña operaciones costosas que se ejecutan en la \acrshort{cpu}, como el algoritmo \gls{nms} (al igual que YOLO). Esta podría ser la principal causa del excelente rendimiento en los equipos de sobremesa.

En el caso del modelo ArcFace, el uso de TensorRT reduce a la mitad o 3 veces la latencia, llegando a reducirse \textbf{5 veces} en el caso del portátil.

OSNet\_x1, en comparación con ArcFace, no representa muchas variaciones en la latencia para los equipos de sobremesa. En cambio, en los dispositivos Jetson el salto es más notorio. A diferencia de YuNet, ni el pre ni el postprocesado resultan un problema, de hecho, aprovechando la disponibilidad del código de PyTorch, se ha implementado el preprocesado internamente en el modelo, de forma que se ejecuta en la \acrshort{gpu} junto con los \glspl{kernel} de la inferencia (ver apéndice \ref{sec:preosnet} para más detalles). La ejecución del preprocesado en \acrshort{gpu} ha supuesto una mejora de aproximadamente 1 milisegundo en la ejecución.

\section{Discusión}
\label{sec:discuss}

Las \acrshort{gpu}s de los equipos de sobremesa rinden de una forma excelente, consiguiendo un salto de mejora equiparable al de los dispositivos Jetson, cuyas \acrshort{gpu}s poseen los tensor cores, más potentes que los núcleos de CUDA.

%Los resultados tan bajos de latencia en los equipos de sobremesa son especialmente sorprendentes, teniendo en cuenta de que las operaciones de copia de memoria y sincronización deberían de añadir una sobrecarga notoria en las inferencias.

%El uso de memoria fijada mapeada en los dispositivos Jetson supone una ligera reducción en la latencia (en torno a 1 milisegundo según las pruebas), que no compensa la latencia de procesamiento. De todos modos, se consiguen ahorrar las llamadas a cudaMemcpy en los sistemas con \acrshort{gpu}s integradas, lo que ayuda a reducir la congestión de la cola de operaciones de un \gls{stream} en la \acrshort{gpu}.

La Jetson AGX Orin, a pesar de doblar los recursos computacionales de la Orin Nano, apenas consigue una mejora en cuanto a reducción de latencia, al mismo tiempo que se queda atrás de los resultados logrados en la AGX Thor. Un ejemplo es la ejecución de YOLO11n en TensorRT, cuya diferencia entre ambas AGX es de \textbf{15 milisegundos} a favor de la AGX Thor, que es una diferencia enormemente significativa (TODO: porqué?).

Las arquitecturas anteriores a Blackwell demuestran ser más conservadoras a la hora de aplicar optimizaciones, mientras que esta última arquitectura opta por las tácticas más rápidas, aunque impliquen una pérdida en la precisión. Un claro ejemplo es el modelo OSNet\_x1, la tabla \ref{tab:equips} muestra los resultados del modelo compilado con un nivel de optimización de 0 en \textit{trtexec} (builderOptimizationLevel igual a 0) que no aplica la generación de \glspl{kernel} dinámicos (que equivale a no aplicar ninguna optimización de la arquitectura y escoger la primera táctica que funcione correctamente). Al aplicar la generación de \glspl{kernel} dinámicos (nivel de optimización > 1), la Jetson AGX Thor consigue unos excelentes \textbf{1.7 milisegundos}, llegando a ser la triunfadora, a costa de una precisión en torno al \textbf{40\%}.

La ejecución de los modelos en \acrshort{gpu} resulta beneficiosa, con una reducción de la latencia de mínimo la mitad en todos los casos, a excepción del modelo YuNet. Esto puede deberse a la ejecución de las inferencias en CUDA y a que de por sí el modelo se ha diseñado de forma ligera para su ejecución optimizada en \acrshort{cpu}s.

TensorRT exprime todas las bondades que ofrece cada arquitectura de \acrshort{gpu}s de NVIDIA, demostrando resultados que superan en rendimiento a equipos con \acrshort{cpu}s de Intel respecto a \acrshort{arm}, como es el caso de la Jetson AGX Thor, aunque en el caso de OSNet, dichas optimizaciones pueden afectar significativamente en la precisión al ser demasiado agresivas.
\chapter{Pruebas de escalabilidad del sistema}
\label{chap:syscal}

\lettrine{E}{n} este capítulo se realiza una evaluación del rendimiento del sistema al elevar el número de cámaras implicadas.

\section{Realización de las pruebas}

Para obtener resultados de precisión del sistema, se dispone de un \gls{dataset} que representa el \gls{gt} del nodo integración de sensores (ver sección \ref{sec:basearch}). La prueba de la que se ha extraído el \gls{dataset} corresponde con la misma del \gls{dataset} de las cámaras. En el apéndice \ref{sec:datsys} se comenta su composición.

Es importante remarcar que todas las pruebas de esta sección se realizan en \textbf{tiempo real}. Según el proyecto, la frecuencia del tiempo real puede variar, en este caso, se espera que el sistema trabaje a \textbf{10 Hz} (o 10 \acrshort{fps}), que es la frecuencia a la que ambos sensores \acrshort{lidar} y Kinect devuelven datos. Los modelos de todas las pruebas realizadas en este capítulo se han ejecutado en el \gls{rt} de \textbf{TensorRT} (salvo YuNet, que utiliza CUDA por medio de OpenCV).

Se inicializa el sistema con el nodo integración de sensores, el nodo \acrshort{lidar} y el número de nodos cámara deseado (ver sección \ref{sec:basearch}). Se reproduce el fichero \gls{rosbag} una vez inicializado el sistema, que lo nutrirá con la información generada por los sensores en \textbf{tiempo real}. Cada predicción generada por el nodo integrador se almacena en memoria y se vuelca en un fichero \acrshort{json} una vez el sistema se detiene (por ejemplo, con un Ctrl+C). Los datos guardados en el fichero \acrshort{json} se procesan contra el \textit{\gls{dataset}} y se devuelven los resultados finales en forma de las siguientes métricas:

%En cada iteración del procesado, se ejecutan \textbf{4 redes neuronales (\acrshort{cnn})}. 2 de ellas, YOLOv8n y YuNet, para la detección de cuerpos y de caras respectivamente. Las otras 2, OSNet\_x1 y ArcFace, para el reconocimiento de cuerpos y caras respectivamente. A pesar de lo que la figura \ref{fig:finalsys} muestra, las redes de detección YOLOv8n y YuNet \textbf{no se ejecutan en paralelo}, lo mismo sucede con las redes ArcFace y OSNet\_x1. Se ha considerado ejecutar en paralelo estas redes. Sin embargo, las limitaciones de memoria principal de la Jetson restringen tomar esta aproximación.

\begin{itemize}
    \item \textit{\textbf{Det precision} (det\_p)}: proporción de personas detectadas en la posición \textbf{\acrshort{3d}} correcta.
    \item \textit{\textbf{Det recall} (det\_r)}: se define igual que el \textit{recall} para los modelos de detección (ver apéndice \ref{subsec:det}). El cálculo de esta métrica \textbf{se ha modificado} respecto a \cite{andrew}, el \textit{TP + FN} del \textit{recall} se ha cambiado por el número total de detecciones del \gls{dataset}, a diferencia de \cite{andrew}, que considera el número total de detecciones del \gls{dataset} dentro de los \textbf{frames que coinciden} en \gls{timestamp} con la salida del nodo integrador, de modo que las caídas en el número de frames coincidentes (principalmente debido a la congestión del sistema) \textbf{no afectan} al valor del \textit{recall}, lo que no es deseable para analizar la escalabilidad del sistema. Este cambio en el cálculo \textbf{varía ligeramente} los resultados obtenidos del \textit{recall} respecto a \cite{andrew}.
          %siendo \textit{det\_tp} una detección positiva y \textit{n\_gt} el número de detecciones registradas en el \gls{dataset}, el \textit{det\_r} se calcula como sigue en la ecuaci
    \item \textit{\textbf{Ident F1 score} (idf1)}: se corresponde con el \textit{F1\_score} global de los modelos de reconocimiento (ver apéndice \ref{subsec:recon}).
    \item \textit{\textbf{Ident precision} (idp)}: es la precisión global de los modelos de reconocimiento.
\end{itemize}

La prueba del sistema da una detección como \textbf{positiva} si la posición \acrshort{3d} predicha es igual a la del \gls{dataset} dentro de un umbral de distancia máximo de \textbf{40 centímetros}. Adicionalmente, el retardo máximo para que un resultado del nodo cámara sea procesado por el nodo integrador es de \textbf{200 milisegundos}, como se expone en la sección \ref{sec:scal} (se han establecido estos valores, ya que son los mismos que los utilizados en \cite{andrew} para las pruebas del sistema).

Para cada prueba, también se mostrará la proporción de los recursos utilizados del dispositivo durante la ejecución del sistema (las pruebas se han realizado \textbf{sin el entorno gráfico}). Se empleará la mediana de todas las lecturas de recursos en una ejecución completa del sistema, ya que es robusta frente a los picos de uso de recursos. En dispositivos Jetson, el comando \textbf{tegrastats} devuelve toda la información necesaria, en cambio, para el resto de dispositivos se obtienen de la siguiente forma:
\begin{itemize}
    \item Carga de la \acrshort{cpu}: media de las mediciones devueltas por el comando \textbf{sar -u}.
    \item Ocupación de la memoria de la \acrshort{cpu}: media de las mediciones devueltas por el comando \textbf{sar -r}.
    \item Carga de la \acrshort{gpu}: media de las mediciones devueltas por el comando \textbf{nvidia-smi}.
    \item Ocupación de la memoria de la \acrshort{gpu}: media de las mediciones devueltas por el comando \textbf{nvidia-smi}.
    \item Ancho de banda de la transferencia de datos por la red: media de las mediciones devueltas por el comando \textbf{ifstat}.
\end{itemize}

\section{Entorno de las pruebas}

Debido a que el \gls{rosbag} utilizado es un archivo pesado (41 \acrshort{gb}), en ciertas situaciones ha sido necesario reproducir dicho fichero desde una \textbf{fuente externa} al no disponer de almacenamiento local suficiente.

\begin{equation}
    (|imagen\_RGB| + |imagen\_profundidad|)*ncams + |nube\_puntos\_LiDAR|
\end{equation}
\label{eq:transmision}

La comunicación con la fuente externa se realiza a través de una red \acrshort{lan} \textbf{Gigabit Ethernet}, que posee un ancho de banda limitado para la transmisión de imágenes \acrshort{rgbd} (máximo teórico de 125 \acrshort{mb}/s). La fórmula \ref{eq:transmision} muestra el tamaño de los datos que debe transmitir la fuente externa cada 100 ms (que es la frecuencia a la que trabaja el sistema). Siendo $|imagen\_RGB| = 40 \acrshort{mb}/s$ (resolución de 1280x1024), $|imagen\_profundidad| = 20 \acrshort{mb}/s$ (resolución de 640x480) y $|nube\_puntos\_LiDAR| = 5 \acrshort{mb}/s$ (los datos se han tomado del comando \textbf{rostopic bw}), el ancho de banda necesario es de \textbf{125 \acrshort{mb}/s} para dos cámaras (\textit{ncams}=2), lo que ya \textbf{equivale} al ancho de banda máximo teórico.

\begin{figure}
    \centering
    \includegraphics[width=1\linewidth]{imagenes/SYSTESTING.jpg}
    \caption{Reproducción y transmisión del \gls{rosbag} por la red}
    \label{fig:systesting}
\end{figure}

La figura \ref{fig:systesting} ilustra el proceso de transmisión de los datos por la red \acrshort{lan}. Para reducir el ancho de banda necesario para la transmisión, se ha optado por desplegar nodos encargados de comprimir la imagen \acrshort{rgb} desde el equipo en el que se reproduce el \gls{rosbag} (\textit{remote side} en la figura \ref{fig:systesting}). La compresión se realiza por medio del paquete \textit{image\_transport} de \acrshort{ros}, que logra reducir hasta 10 veces el tamaño de la imagen \acrshort{rgb} (resultando en 4 \acrshort{mb}/s). Los nodos cámara (\textit{host side} en la figura \ref{fig:systesting}) se suscriben al tópico generado por el nodo que descomprime la imagen (\textit{image\_transport} \textit{decompress} en el \textit{host side}).

En el caso de la imagen de distancias, en sí no es un objeto pesado, por lo que no necesita compresión. Sin embargo, la frecuencia elevada de transmisión (30 Hz), hace que se requiera de un mayor ancho de banda. Como el resto de los sensores funcionan a 10 Hz, la tasa de la imagen de distancias puede reducirse a dicha frecuencia, de forma que el sistema no percibe el cambio y se logra reducir el ancho de banda. Con reducir la tasa de envío del nodo de \acrshort{ros} encargado de publicar la imagen sería suficiente, pero debido a que en el \gls{rosbag} dicho nodo se grabó a 30 Hz, se ha optado por ejecutar el paquete \textit{topic\_tools} de \acrshort{ros}, que permite publicar una réplica de un tópico a una menor frecuencia.

Debido a que el \gls{rosbag} solo contiene los datos de 2 cámaras, para realizar la prueba de escalabilidad con un mayor número de cámaras, ha sido necesario \textbf{replicar} los tópicos de las imágenes mediante el paquete de \acrshort{ros} \textit{topic\_tools} (figura \ref{fig:systesting}), de forma que los nuevos nodos se pueden suscribir a dichos tópicos.

\section{Resultados}
\label{sec:scaleresults}

A continuación se exponen los resultados de las pruebas de escalabilidad ejecutadas en la familia Intel (equipos de sobremesa) y en 2 dispositivos de la familia Jetson (Jetson Orin Nano y Jetson AGX Thor). Los resultados marcados con guion indican ausencia del dato por imposibilidad de realizar la prueba.

\subsection{Familia Intel}

\begin{table}[]
    \centering
    \begin{tabular}{cc|ccc}
        \hline
        \multicolumn{2}{c|}{\multirow{2}{*}{\textit{\begin{tabular}[c]{@{}c@{}}Number of\\ cameras\end{tabular}}}} & \multicolumn{3}{c}{\textit{Intel family}}                                                                                       \\
        \multicolumn{2}{c|}{}                                                                                      & \textbf{Lab PC}                           & \textbf{Laptop \footnotemark[1]} & \textbf{Reference \cite{andrew}}                 \\ \hline
        \multicolumn{1}{c|}{\multirow{4}{*}{\textit{\begin{tabular}[c]{@{}c@{}}2\\ cameras\end{tabular}}}}         & \textit{det\_p}                           & 87.2 ± 0.8                       & 87.0 ± 0.4                       & \textbf{91.2} \\ \cline{2-2}
        \multicolumn{1}{c|}{}                                                                                      & \textit{det\_r}                           & 51.7 ± 1.3                       & 52.0 ± 1.3                       & \textbf{52.8} \\ \cline{2-2}
        \multicolumn{1}{c|}{}                                                                                      & \textit{idf1}                             & 58.2 ± 4.5                       & 59.3 ± 5.4                       & \textbf{65.0} \\ \cline{2-2}
        \multicolumn{1}{c|}{}                                                                                      & \textit{idp}                              & 83.5 ± 6.6                       & 85.4 ± 7.9                       & \textbf{93.1} \\ \hline
        \multicolumn{1}{c|}{\multirow{4}{*}{\textit{\begin{tabular}[c]{@{}c@{}}5\\ cameras\end{tabular}}}}         & \textit{det\_p}                           & 85.7 ± 0.5                       & \textbf{86.1 ± 0.7}              & -             \\ \cline{2-2}
        \multicolumn{1}{c|}{}                                                                                      & \textit{det\_r}                           & 51.1 ± 1.3                       & \textbf{51.8 ± 1.9}              & -             \\ \cline{2-2}
        \multicolumn{1}{c|}{}                                                                                      & \textit{idf1}                             & 57.2 ± 5.2                       & \textbf{57.9 ± 4.4}              & -             \\ \cline{2-2}
        \multicolumn{1}{c|}{}                                                                                      & \textit{idp}                              & 82.6 ± 7.6                       & \textbf{82.7 ± 5.5}              & -             \\ \hline
        \multicolumn{1}{c|}{\multirow{4}{*}{\textit{\begin{tabular}[c]{@{}c@{}}10\\ cameras\end{tabular}}}}        & \textit{det\_p}                           & \textbf{87.8 ± 1.1}              & 84.9 ± 1.9                       & -             \\ \cline{2-2}
        \multicolumn{1}{c|}{}                                                                                      & \textit{det\_r}                           & 46.3 ± 2.7                       & \textbf{50.8 ± 2.1}              & -             \\ \cline{2-2}
        \multicolumn{1}{c|}{}                                                                                      & \textit{idf1}                             & 52.6 ± 2.8                       & \textbf{56.6 ± 3.0}              & -             \\ \cline{2-2}
        \multicolumn{1}{c|}{}                                                                                      & \textit{idp}                              & 80.7 ± 3.0                       & \textbf{81.8 ± 3.7}              & -             \\ \hline
    \end{tabular}
    \caption{Rendimiento del sistema en la familia Intel (\acrshort{ros} 1)}
    \label{tab:scaleperf}
\end{table}
%\end{landscape}
\footnotetext[1]{Se reciben los datos del archivo \gls{rosbag} desde una fuente externa}

La tabla \ref{tab:scaleperf} muestra los resultados de la familia Intel en función del número de cámaras del sistema en \textbf{\acrshort{ros} 1}. La columna \textit{Reference} muestra los resultados obtenidos en el dispositivo de pruebas utilizado en \cite{andrew}, en el que solo se registraron datos del sistema funcionando con un máximo de 2 cámaras.

% TODO: En el caso de las 2 cámaras, no se ve una mejora respecto a los resultados de referencia, de hecho se experimenta una ligera caída en los resultados...

% Please add the following required packages to your document preamble:
% \usepackage{multirow}
\begin{table}[]
    \begin{tabular}{cc|llll}
        \hline
        \multicolumn{2}{c|}{\multirow{3}{*}{\textit{\begin{tabular}[c]{@{}c@{}}Number of\\ cameras\end{tabular}}}} & \multicolumn{4}{c}{\textit{Desktop}}                                                                                                                        \\
        \multicolumn{2}{c|}{}                                                                                      & \multicolumn{2}{c}{\textbf{Lab PC}}  & \multicolumn{2}{c}{\textbf{Laptop (*)}}                                                                              \\ \cline{3-6}
        \multicolumn{2}{c|}{}                                                                                      & \multicolumn{1}{l|}{\textit{CPU}}    & \multicolumn{1}{l|}{\textit{GPU}}       & \multicolumn{1}{l|}{\textit{CPU}} & \multicolumn{1}{l|}{\textit{GPU}}      \\ \hline
        \multicolumn{1}{c|}{\multirow{2}{*}{\textit{\begin{tabular}[c]{@{}c@{}}2\\ cameras\end{tabular}}}}         & \textit{\%load}                      & 6.3                                     & \multicolumn{1}{l|}{30}           &                                   &    \\ \cline{2-2}
        \multicolumn{1}{c|}{}                                                                                      & \textit{\%mem}                       & 9.27                                    & \multicolumn{1}{l|}{21}           &                                   &    \\ \hline
        \multicolumn{1}{c|}{\multirow{2}{*}{\textit{\begin{tabular}[c]{@{}c@{}}5\\ cameras\end{tabular}}}}         & \textit{\%load}                      & 13.47                                   & \multicolumn{1}{l|}{48.46}        &                                   &    \\ \cline{2-2}
        \multicolumn{1}{c|}{}                                                                                      & \textit{\%mem}                       & 16.06                                   & \multicolumn{1}{l|}{56}           &                                   &    \\ \hline
        \multicolumn{1}{c|}{\multirow{2}{*}{\textit{\begin{tabular}[c]{@{}c@{}}10\\ cameras\end{tabular}}}}        & \textit{\%load}                      & 44                                      & \multicolumn{1}{l|}{88.22}        & 44.82                             & 64 \\ \cline{2-2}
        \multicolumn{1}{c|}{}                                                                                      & \textit{\%mem}                       & 27.87                                   & \multicolumn{1}{l|}{100}          & 73.1                              & 88 \\ \hline
    \end{tabular}
\end{table}
\captionof{table}{Recursos utilizados del sistema al escalar múltiples cámaras}
\label{tab:scaleresources}

En general, los resultados entre el PC del laboratorio y el portátil son \textbf{muy similares}, esto es especialmente relevante debido al esfuerzo adicional de compresión/descompresión y transmisión de mensajes por la red en el caso del portátil, lo que demuestra la efectividad de las herramientas de red de \acrshort{ros} y de la arquitectura propuesta (figura \ref{fig:systesting}), que han logrado reducir el ancho de banda necesario para la transmisión de mensajes (en la tabla \ref{tab:scaleresources} se muestra un 61\% de utilización respecto al máximo teórico de una red GigabitEthernet (125 \acrshort{mb}/s) en el procesamiento de 10 cámaras). La latencia asociada a la compresión en el origen, transmisión del mensaje y descompresión de un frame \acrshort{rgb} en el destino es de aproximadamente \textbf{50 milisegundos} (obtenido calculando la diferencia entre el tiempo actual y el \gls{timestamp} de generación del mensaje inicial) que, según los resultados, el portátil es capaz de compensar incluso con 10 cámaras.

El portátil escala correctamente hasta las 10 cámaras, a diferencia del PC del laboratorio, que experimenta una caída en el \textit{det\_r}. Dicha caída puede deberse a una carga excesiva en la \acrshort{gpu} del PC del laboratorio, mientras que la \acrshort{gpu} del portátil la consigue tolerar en mayor medida (en la tabla \ref{tab:scaleresources} se muestra una utilización del 84\% respecto al 72\% de la \acrshort{gpu} del portátil), principalmente por un mayor número de \glspl{sm} en la \acrshort{gpu} Ampere (20) respecto a la Turing (14) y del \textbf{casi triple} de núcleos de CUDA (2048 y 896 respectivamente).

% ESTO NO ES CIERTO, al reducir el intervalo a 200 ms baja el recall (a 47, que es MENOS que la referencia) pero no sube la precisión, esto puede deberse a que se acepta la llegada de los mensajes de las cámaras con un máximo de \textbf{500 milisegundos} de retardo respecto a los \textbf{200 milisegundos} configurados en la prueba de referencia, lo que hace que se devuelvan predicciones con posiciones más atrasadas y, por lo tanto, que superen la holgura de 40 centímetros para asumirlas correctas. En cuanto a la métrica \textit{idp} todos los sistemas se encuentran relativamente a la par.

%a que el nodo integrador debe de fusionar los datos procedentes de todos los sensores, al aumentar su número también se aumenta la cantidad de procesamiento. Por lo tanto, más tiempo se demorará en devolver una predicción, que se emparejará con el frame del \gls{dataset} con el mismo \gls{timestamp}, que corresponde a un instante real \textbf{anterior}.

%PRECISION DE FAMILIA INTEL: la caida en la precision no se debe al slope ni al tamaño de la cola. El problema debe de estar en el nodo LiDAR.

%PRECISIÓN AL AUMENTAR EL Nº DE CÁMARAS: creo q el nodo integrador incluye detecciones repetidas (con las mismas posiciones exactas) cuando no se conoce la identidad (la etiqueta como unk), ya que con entidades desconocidas no se guardan duplicados. Esto produce una proliferación de det_fps, ya que al haber puntos repetidos, hay más detecciones pero menos matches, por lo tanto aumentan los unmatches y así los det_fps.

% OLD: La precisión en los reconocimientos (métrica \textit{idp}) no experimenta caídas notorias (excepto por la situación de las 5 cámaras en el portátil, que sorprendente se recupera al subir a las 10 cámaras) al aumentar la cantidad de cámaras, lo que demuestra que los modelos de reconocimiento siguen funcionando correctamente ante un enorme estrés del sistema. El \textit{idf1} depende en cierta medida del \textit{det\_r}, por lo que es de esperar que dicho valor descienda junto al \textit{det\_r}, que en cierto modo es compensado por el \textit{idp}.

\subsection{Familia Jetson}

\begin{landscape}
    \centering
    % Please add the following required packages to your document preamble:
    % \usepackage{multirow}
    \begin{table}[]
        \begin{tabular}{cc|ll|l}
            \hline
            \multicolumn{2}{c|}{\multirow{2}{*}{\textit{\begin{tabular}[c]{@{}c@{}}Number of\\ cameras\end{tabular}}}} & \multicolumn{2}{c|}{Desktop}                                      & \textit{Jetson embedded systems}                                                                   \\
            \multicolumn{2}{c|}{}                                                                                      & \multicolumn{1}{c}{\textbf{Laptop (*)}}                           & \textbf{Reference \textbackslash{}cite\{andrew\}} & \multicolumn{1}{c}{\textbf{Orin Nano}}         \\ \hline
            \multicolumn{1}{c|}{\multirow{4}{*}{\textit{\begin{tabular}[c]{@{}c@{}}2\\ cameras\end{tabular}}}}         & \textit{\begin{tabular}[c]{@{}c@{}}Det\\ precision\end{tabular}}  &                                                   & 0.95                                   & 0.93  \\ \cline{2-2}
            \multicolumn{1}{c|}{}                                                                                      & \textit{\begin{tabular}[c]{@{}c@{}}Det\\ recall\end{tabular}}     &                                                   & 0.347                                  & 0.356 \\ \cline{2-2}
            \multicolumn{1}{c|}{}                                                                                      & \textit{\begin{tabular}[c]{@{}c@{}}Iden\\ F1 score\end{tabular}}  &                                                   &                                        & 0.464 \\ \cline{2-2}
            \multicolumn{1}{c|}{}                                                                                      & \textit{\begin{tabular}[c]{@{}c@{}}Iden\\ precision\end{tabular}} &                                                   & 0.931                                  & 0.936 \\ \hline
            \multicolumn{1}{c|}{\multirow{4}{*}{\textit{\begin{tabular}[c]{@{}c@{}}5\\ cameras\end{tabular}}}}         & \textit{\begin{tabular}[c]{@{}c@{}}Det\\ precision\end{tabular}}  &                                                   &                                        &       \\ \cline{2-2}
            \multicolumn{1}{c|}{}                                                                                      & \textit{\begin{tabular}[c]{@{}c@{}}Det\\ recall\end{tabular}}     &                                                   &                                        &       \\ \cline{2-2}
            \multicolumn{1}{c|}{}                                                                                      & \textit{\begin{tabular}[c]{@{}c@{}}Iden\\ F1 score\end{tabular}}  &                                                   &                                        &       \\ \cline{2-2}
            \multicolumn{1}{c|}{}                                                                                      & \textit{\begin{tabular}[c]{@{}c@{}}Iden\\ precision\end{tabular}} &                                                   &                                        &       \\ \hline
            \multicolumn{1}{c|}{\multirow{4}{*}{\textit{\begin{tabular}[c]{@{}c@{}}10\\ cameras\end{tabular}}}}        & \textit{\begin{tabular}[c]{@{}c@{}}Det\\ precision\end{tabular}}  &                                                   &                                        &       \\ \cline{2-2}
            \multicolumn{1}{c|}{}                                                                                      & \textit{\begin{tabular}[c]{@{}c@{}}Det\\ recall\end{tabular}}     &                                                   &                                        &       \\ \cline{2-2}
            \multicolumn{1}{c|}{}                                                                                      & \textit{\begin{tabular}[c]{@{}c@{}}Iden\\ F1 score\end{tabular}}  &                                                   &                                        &       \\ \cline{2-2}
            \multicolumn{1}{c|}{}                                                                                      & \textit{\begin{tabular}[c]{@{}c@{}}Iden\\ precision\end{tabular}} &                                                   &                                        &       \\ \hline
        \end{tabular}
    \end{table}
    \captionof{table}{Rendimiento del sistema en ROS 2}
    \label{tab:scaleros2}
\end{landscape}

En la tabla \ref{tab:scaleros2} se muestra el rendimiento del sistema en \textbf{\acrshort{ros} 2} para los equipos Jetson Orin Nano y Jetson AGX Thor. En estos dispositivos no es posible instalar \acrshort{ros} Noetic, al menos para poder contar con las últimas versiones de las librerías. Por los motivos comentados en la sección \ref{subsec:ROS2}, el sistema en \acrshort{ros} 2 no puede ejecutar el nodo \acrshort{lidar}, por lo que los resultados mostrados solo tienen en cuenta los nodos cámara junto al integrador.

El \textit{det\_r} se muestra claramente inferior respecto a los resultados de la familia Intel, ya que el \acrshort{lidar} otorgaba cobertura completa a la reducida visión de las cámaras. La Jetson AGX Thor logra superar el valor del equipo de referencia en esta métrica, mientras que la Jetson Orin Nano se queda a 3 puntos de superarlo. Los resultados de precisión (\textit{det\_p} e \textit{idp}) se mantienen entre todos los dispositivos. Estos datos demuestran el excelente rendimiento de las \acrshort{gpu}s de las Jetson, capaces de compensar las limitaciones de una \acrshort{cpu} de \acrshort{arm} respecto a un Intel Core i5 (procesador del equipo de referencia). También es importante recordar que el cálculo del \textit{det\_r} se ha modificado respecto a \cite{andrew}, por lo que se muestra un \textit{recall} ligeramente inflado en la referencia.

% Please add the following required packages to your document preamble:
% \usepackage{multirow}
\begin{table}[]
    \begin{tabular}{cc|cccc}
        \hline
        \multicolumn{2}{c|}{\multirow{3}{*}{\textit{\begin{tabular}[c]{@{}c@{}}Number of\\ cameras\end{tabular}}}} & \multicolumn{4}{c}{\textit{Jetson embedded systems}}                                                                                                   \\
        \multicolumn{2}{c|}{}                                                                                      & \multicolumn{2}{c}{\textbf{Orin Nano}}               & \multicolumn{2}{c}{\textbf{AGX Thor}}                                                           \\ \cline{3-6}
        \multicolumn{2}{c|}{}                                                                                      & \multicolumn{1}{c|}{\textit{CPU}}                    & \multicolumn{1}{c|}{\textit{GPU}}     & \multicolumn{1}{c|}{\textit{CPU}} & \textit{GPU}        \\ \hline
        \multicolumn{1}{c|}{\multirow{2}{*}{\begin{tabular}[c]{@{}c@{}}2\\ cameras\end{tabular}}}                  & \%load                                               & 48.2                                  & \multicolumn{1}{c|}{56.5}         & 9.6          & 12.0 \\ \cline{2-2}
        \multicolumn{1}{c|}{}                                                                                      & \%mem                                                & \multicolumn{2}{c|}{51.9}             & \multicolumn{2}{c}{8.2}                                 \\ \hline
        \multicolumn{1}{c|}{\multirow{2}{*}{\begin{tabular}[c]{@{}c@{}}5\\ cameras\end{tabular}}}                  & \textit{\%load}                                      & 87.8                                  & \multicolumn{1}{c|}{71.0}         & 18.5         & 33.0 \\ \cline{2-2}
        \multicolumn{1}{c|}{}                                                                                      & \textit{\%mem}                                       & \multicolumn{2}{c|}{91.1}             & \multicolumn{2}{c}{10.5}                                \\ \hline
        \multicolumn{1}{c|}{\multirow{2}{*}{\begin{tabular}[c]{@{}c@{}}10\\ cameras\end{tabular}}}                 & \textit{\%load}                                      & -                                     & \multicolumn{1}{c|}{-}            & 45.0         & 88.0 \\ \cline{2-2}
        \multicolumn{1}{c|}{}                                                                                      & \textit{\%mem}                                       & \multicolumn{2}{c|}{-}                & \multicolumn{2}{c}{15.2}                                \\ \hline
    \end{tabular}
    \caption{Recursos utilizados del sistema al escalar múltiples cámaras (solo cámaras en \acrshort{ros} 2)}
\end{table}
\label{tab:scaleresourcesros2}

Se experimenta una ligera caída en el \textit{det\_p} al aumentar a 5 cámaras en la Jetson AGX Thor, ya que se generan más resultados, por lo que aumenta la probabilidad de fusionar información de distintos instantes temporales en el nodo integrador. El \textit{det\_r} se mantiene constante, lo que significa que el dispositivo tolera sin problema la carga introducida (el bajo uso de recursos en la tabla \ref{tab:scaleresourcesros2} lo avala). Sin embargo, en la Jetson Orin Nano el \textit{det\_r} cae drásticamente debido a la saturación de los recursos (acorde a la tabla \ref{tab:scaleresourcesros2}), tanto de memoria, lo que impide la realización de las inferencias, como de procesamiento, lo que implica en un retardo de las respuestas y, por consiguiente, a que se consideren resultados desactualizados que conducen a una pérdida en la precisión.

% RECALL: En la Thor con 2 y 5 cámaras el n_gt se mantiene encima de los 6000 mientras q con 10 cams baja a 4300. Creo q el recall debe ser: det_fp/6259 y 6259 es el número de detecciones totales del ground truth, sin recortar nada.

La Jetson AGX Thor sufre una caída en el \textit{det\_p}, pero sobre todo en el \textit{det\_r} al procesar 10 cámaras. Esto se debe a la elevada utilización de la \acrshort{gpu} (como se muestra en la tabla \ref{tab:scaleresourcesros2}), que provoca una reducción en el número de detecciones totales, además de su precisión. Este hecho lo demuestra la ejecución del modelo YuNet en la \acrshort{cpu}, que ha logrado aumentar el \textit{det\_p} en 4 puntos respecto a ejecutarlo en \acrshort{gpu}. Se ha escogido YuNet, debido a que la reducción en la latencia mediante CUDA no es tan efectiva como en TensorRT y a que la gestión de la memoria realizada por OpenCV es \textbf{mucho menos eficiente} (en la tabla \ref{tab:scaleresourcesros2} se puede ver como el uso de memoria con 10 cámaras y YuNet en \acrshort{cpu} es \textbf{menor} que con 5 cámaras y YuNet en \acrshort{gpu}).

\section{Discusiones}

En los dispositivos de sobremesa y en la Jetson AGX Thor, se ha logrado escalar el sistema hasta un número de cámaras más que suficiente para el caso de uso de este proyecto (para cubrir un espacio de 360º no es necesario disponer de 10 cámaras), al menos para ser procesado por un solo dispositivo. En la prueba se llegan a detectar un máximo de \textbf{5 personas simultáneamente} entre las 2 cámaras, por lo que se podría asumir que en dichos dispositivos el sistema empieza a resentirse al detectar y reconocer a \textbf{26 personas de forma simultánea} (si se asume 3*6 + 2*4, siendo 6 el número de cámaras que detectan a 3 personas y 4 el número de cámaras que detectan a 2 personas). Por otro lado, la \textbf{Jetson Orin Nano} ha logrado resultados decentes para 2 cámaras, llegando a sufrir una caída crítica al operar con 5 cámaras. Este dispositivo en particular sería especialmente útil debido a su muy bajo consumo y peso, lo que permite su fácil integración en un robot móvil.

Los 8 \acrshort{gb}s de memoria de la Jetson Orin Nano se vuelven \textbf{insuficientes}, debido a todos los recursos necesarios a reservar, como los contextos de CUDA, que pesan en torno a cientos de \acrshort{mb}s, que se multiplican por 4 modelos y a su vez por 5 cámaras. Se ha intentado mitigar la caída en el rendimiento por medio de ciertas optimizaciones de uso de memoria en los códigos de Python, que han permitido ahorrar hasta \textbf{400 \acrshort{mb}s} en la inicialización del sistema, y que se citan en el apéndice \ref{chap:optimus}. Aun así, es necesario reducir la elevada demanda de procesamiento que se hace inabarcable para la Jetson Orin Nano y por ende impide al \textit{det\_r} despegar.

Es importante destacar que las pruebas se realizaron con \textbf{6 personas registradas en la base de datos}, cada una con 4 vectores descriptores (2 correspondientes a la cara y las otras 2 al cuerpo). Aumentar este número tendría consecuencias en los modelos de reconocimiento, ya que iteran toda la base de datos para calcular las distancias del vector con cada individuo, para devolver la mejor coincidencia. Según el dispositivo, esta operación tarda menos o alrededor de 1 milisegundo, lo que podría aumentar en varios milisegundos si se añaden por ejemplo 100 personas.

Por último, el uso de la \acrshort{gpu} ha permitido elevar al sistema más allá de lo que la \acrshort{cpu} puede afronta, no solo por la distribución de los recursos, sino por el uso de TensorRT, que minimiza el uso de la memoria y de la carga computacional, a diferencia de OpenCV integrado con CUDA, cuya gestión de recursos hace que, en el caso de la Jetson AGX Thor, sea más viable la ejecución en \acrshort{cpu}. Sin embargo, se han deslumbrado los límites computacionales de dicha aceleradora, por lo que se vuelve necesario aplicar técnicas, tanto software como hardware, para maximizar su utilización. La \textbf{\gls{quant} de los modelos a INT8} y/o el uso del hardware \textbf{Deep Learning Accelerator (DLA)} \cite{DLA} integrado en las Jetson son pasos que podrían contribuir a este proceso.
\chapter{Pruebas y resultados del sistema extendido}
\label{chap:systesting}

\lettrine{E}{n} este capítulo se exponen las pruebas y los resultados que miden las capacidades de los nuevos componentes implementados.

\section{Realización de las pruebas}

Las pruebas que analizan los nuevos componentes se realizaron por \textbf{separado del sistema} bajo el modelo de reconocimiento facial \textbf{ArcFace}. Por lo tanto, se sigue el mismo procedimiento que el expuesto en la sección \ref{sec:probes} para los modelos de reconocimiento y se aplica el mismo \gls{dataset}, salvo por los siguientes cambios en las pruebas:

\begin{itemize}
    \item El módulo de reconocimiento adaptativo procesa \textbf{secuencias de frames} (vídeos), que se obtienen mediante la técnica de procesamiento de vídeo (sección \ref{sec:vídeo}).
    \item Se devuelven las métricas de \textbf{\textit{precision}, \textit{recall}} y \textbf{\textit{F1\_score}}, en vez de únicamente la métrica \textit{precision} para evaluar los reconocimientos (ver apéndice \ref{subsec:recon} para más detalles).
\end{itemize}

A modo de evaluar las capacidades del reconocimiento adaptativo desarrollado, se han realizado las siguientes pruebas:

\begin{itemize}
    \item \textbf{Ajuste de parámetros del aprendizaje y reconocimiento.} Se realiza un conjunto de pruebas con diferentes valores de parámetros para evaluar su impacto en el rendimiento.
    \item \textbf{Métodos de inicialización del sistema.} Se prueban las 2 técnicas de inicialización propuestas (sección \ref{sec:initieing}), que son una \textbf{semisupervisada} (las muestras son seleccionadas por el operador) y la otra \textbf{no supervisada} (el propio sistema extrae las muestras).
\end{itemize}

Todas las pruebas mencionadas se ejecutan en las modalidades \textbf{\textit{Open-Set}}, que en este caso \textbf{detecta desconocidos y actualiza las entidades ya registradas}\footnote{Habitualmente, en el modo \textit{Open-Set} no se aplica la actualización de las entidades} y \textbf{\textit{Open-World}}, que realiza las mismas funciones que \textit{Open-Set}, pero con la capacidad de \textbf{registrar a los desconocidos en el sistema} (como ya se comentó en la sección \ref{sec:incrtypes}).

\section{Ajuste de parámetros del aprendizaje y reconocimiento}
\label{seq:paramtunning}

En esta sección se realiza un estudio acerca de la configuración óptima de parámetros en términos de rendimiento del sistema. A continuación, se exponen los parámetros utilizados y su función, junto con el rango de valores probado.

\begin{figure}
    \centering
    \includegraphics[width=0.8\linewidth]{imagenes/SOLAPING.jpg}
    \caption{Ejemplo de funcionamiento del parámetro \textit{step\_size}}
    \label{fig:SOLAPING}
\end{figure}

\begin{description}
    \item[Umbral de Weibull (\textit{Tw})] Valor de umbral que determina si una entidad es \textbf{desconocida o no}. El rango probado es (0.05 0.1 0.25). No se muestran valores más altos, debido a que no presentan variaciones notorias en los resultados respecto al valor 0.25.
    \item[Percentil (\textit{percent})] Valor del percentil aplicado en la obtención de la puntuación mínima de cada comité (en el apéndice \ref{sec:val} se explica el uso de los percentiles). El rango probado es (25 50 75).
    \item[Número de \acrshort{svm}s (\textit{max\_svm})] Valor que representa el tamaño máximo de un comité. El rango probado es (1 3 7 10),  valores más altos implicarían un mayor coste computacional en cada iteración.
    \item[Número de negativos (\textit{nneg})] Representa el tamaño del conjunto de muestras negativas. El rango probado es (10 50). No se han podido probar valores más altos, debido a la escasez de muestras disponibles.
    \item[Número de positivos] Establece el número de muestras positivas que define a un comité. El tamaño se ha fijado en 5 muestras, ya que en \cite{5samples} se ha demostrado que otorga buen rendimiento, al mismo tiempo que utiliza el mínimo de muestras posibles.
    \item[Tamaño de secuencia (\textit{frame\_block})] Número de frames que conforman una secuencia de vídeo. Se ha fijado el valor en 15 frames, que equivalen a 1,5 segundos de vídeo con las cámaras funcionando a 10 Hz.
    \item[Tamaño de salto (\textit{step\_size})] Establece el salto en el número de frames entre secuencias. Gracias a este parámetro, se obtiene un mayor número de secuencias a partir de un mismo vídeo de prueba. La figura \ref{fig:SOLAPING} muestra un ejemplo con un tamaño de secuencia de 15 y un tamaño de salto de 5 (que es el que se va a mantener para las pruebas).
\end{description}

A continuación se exponen los resultados obtenidos para cada combinación en \textit{Open-Set} y \textit{Open-World} como una media de 5 ejecuciones. A modo de obtener más puntos para definir las distribuciones de Weibull, se ha expandido el sistema con \textbf{4 individuos} a mayores de los 6 presentes en el vídeo de prueba, cuyas muestras se han adquirido de la misma forma (capturas de las cámaras Kinect). %TODO: poner fotos??? %Se inicializa cada comité con 5 frames representativos (que corresponde con el \textbf{tamaño de la plantilla}).

\subsection{Modo \textit{Open-Set}}
\label{sec:osparams}

\begin{table}[]
    \centering
    \begin{tabular}{c|c|c|c|c|c}
        \hline
        \textit{}          & \textit{Best result} & \textit{2º best result} & \textit{3º best result} & \textit{4º best result} & \textit{5º best result} \\ \hline
        \textit{Precision} & 77.8 ± 2.6           & 79.1 ± 1.7              & 76.0 ± 2.7              & 79.6 ± 2.4              & \textbf{81.9 ± 3.3}     \\
        \textit{Recall}    & 92.9 ± 1.4           & 89.7 ± 1.4              & \textbf{93.5 ± 1.1}     & 88.5 ± 1.3              & 85.1 ± 3.0              \\
        \textit{F1 score}  & \textbf{84.6 ± 1.0}  & 84.1 ± 1.4              & 83.8 ± 1.4              & 83.8 ± 1.6              & 83.4 ± 2.2              \\ \hline
        \textit{Tw}        & 0.25                 & 0.1                     & 0.25                    & 0.1                     & 0.05                    \\
        \textit{Percent}   & 25                   & 25                      & 25                      & 25                      & 25                      \\
        \textit{Max\_svm}  & 10                   & 7                       & 7                       & 10                      & 10                      \\
        \textit{Nneg}      & 50                   & 50                      & 50                      & 50                      & 50                      \\ \hline
    \end{tabular}
    \caption{Mejores resultados del ajuste de parámetros en \textit{Open-Set}.}
    \label{tab:osparams}
\end{table}

La tabla \ref{tab:osparams} muestra los 5 mejores resultados de las pruebas del análisis de parámetros ordenados según la métrica \textit{F1\_score}.

El parámetro \textbf{\textit{percent}} muestra una tendencia clara al valor \textbf{25}. Un percentil de 25 implica escoger las 2/3 primeras \acrshort{svm} para otorgar la puntuación final del comité. En el caso de este sistema, cuando un comité se activa con sus muestras, un porcentaje no precisamente bajo (aproximadamente el 30-40\%) de las \acrshort{svm} devuelven una respuesta \textbf{negativa} frente a las muestras de su propia clase, mientras que otra parte del comité devuelve valores cercanos al 0 (lo que se podría considerar como indecisión en la respuesta), por lo que realmente \textbf{no se compone una mayoría} de respuestas positivas. Por este motivo, el sistema prefiere un \textbf{percentil bajo} (25) frente a la \textbf{votación por mayoría} (mediana, percentil 75). Los \textbf{valores extremos} en el percentil implican en un \textbf{bajo \textit{recall}}, un percentil de 100 implica obtener el mayor valor (que se corresponde con la peor coincidencia), este valor puede proceder de una \acrshort{svm} de otro individuo (pudo incluirse debido a un reconocimiento erróneo previo), que hace que el comité no se active ante sus muestras. Lo mismo sucede con un percentil de 0.\\
\indent Para el parámetro \textbf{\textit{nneg}} se considera más efectivo un tamaño de 50 del conjunto de negativos frente al valor de 10, ya que con 50 muestras se puede albergar un \textbf{mayor número de individuos} y, por ende, la \acrshort{svm} puede definir mejor el hiperplano que separa las muestras positivas de las negativas, luego es capaz de \textbf{discriminar mejor} dichas muestras.\\
\indent En cuanto al valor del parámetro \textbf{\textit{Tw}} no existe una tendencia clara. Atendiendo a los valores de \textit{precision} y \textit{recall}, se puede apreciar como al \textbf{aumentar el umbral de Weibull}, el \textbf{\textit{recall} aumenta a costa de la precisión}, debido a que se está agregando cierta tolerancia de la mejor puntuación a pertenecer a la distribución de puntuaciones no coincidentes, por lo tanto, \textbf{menos desconocidos} se detectan (aumenta el \textit{recall}) a costa de aceptar un mayor número de \textbf{predicciones erróneas} (baja la precisión), un valor bajo demuestra justo lo contrario. Un valor de \textbf{0.1} parece ser la opción \textbf{más equilibrada} entre un excelente \textit{recall} y una excelente precisión para \textit{Open-Set}.\\
\indent En el caso del parámetro \textbf{\textit{max\_svm}}, tampoco es seguro qué único valor escoger. Puede afirmarse que un valor alto devuelve resultados más robustos, esto se debe a que se implica un mayor número de \acrshort{svm} en cada reconocimiento, lo que da lugar a más diversidad y por lo tanto a unos comités que se activan en una mayor cantidad de diversas situaciones de la entidad.

\subsection{Modo \textit{Open-World}}

% Please add the following required packages to your document preamble:
% \usepackage[table,xcdraw]{xcolor}
% Beamer presentation requires \usepackage{colortbl} instead of \usepackage[table,xcdraw]{xcolor}
\begin{table}[]
    \begin{tabular}{l|l|l|l|l}
        \hline
        \textit{}                                                              & \textit{Best result} & \textit{2º best result}     & \textit{3º best result}     & \textit{4º best result}     \\ \hline
        \textit{F1 score}                                                      & 78.1                 & {\color[HTML]{000000} 66.3} & {\color[HTML]{000000} 64.8} & {\color[HTML]{000000} 64.6} \\
        \textit{\begin{tabular}[c]{@{}l@{}}Number of\\ ensembles\end{tabular}} & 20                   & 33                          & 27                          & 56                          \\
        \textit{Tw}                                                            &                      &                             &                             &                             \\
        \textit{Percent}                                                       &                      &                             &                             &                             \\
        \textit{Max\_svm}                                                      &                      &                             &                             &                             \\
        \textit{Nneg}                                                          &                      &                             &                             &                             \\ \hline
    \end{tabular}
    \captionof{table}{Mejores resultados del ajuste de parámetros en \textit{Open-World}.}
    \label{tab:owparams}
\end{table}

Al igual que en la anterior sección, en la tabla \ref{tab:owparams} se muestran los cinco mejores resultados, ordenados acorde al \textbf{\textit{F1\_score}}.

Los parámetros \textbf{\textit{nneg}} y \textbf{\textit{percent}} permanecen constantes, acorde a las pruebas de la anterior sección.

A la hora de escoger un valor de \textbf{\textit{Tw}} es necesario tener en cuenta una métrica adicional, que es el \textbf{número de comités} que se han generado (\textit{Number of ensembles} en la tabla \ref{tab:owparams}), contando los 10 comités iniciales. Un \textbf{valor alto de \textit{Tw}} implica una \textbf{reducción} en la creación de \textbf{comités}, debido al \textbf{elevado \textit{recall}} y por ende a un \textbf{menor número de desconocidos}. Al reducir el valor de \emph{Tw} se \textbf{elevan notablemente} los \textbf{comités} creados, pasando de 18 a 34 comités, que se corresponden con \textbf{comités duplicados}. Los comités duplicados se entrenan con las muestras del propio individuo tanto en el \textbf{conjunto de positivos} como en el de \textbf{negativos} (ya que dicho individuo ya se encuentra registrado en el sistema), lo que \textbf{anula el poder discriminativo} de las \acrshort{svm}, traduciéndose en mayor cantidad de falsos desconocidos y predicciones erróneas (tanto el \emph{precision} como el \emph{recall} \textbf{bajan}), proliferando aún más la creación de estos comités.

En cuanto al parámetro \textbf{\textit{max\_svm}}, los resultados conducen a la misma conclusión que en las pruebas de la sección \ref{sec:osparams}, salvo por el valor de 3 en uno de los resultados, que implica que el sistema prefiere un valor de \textit{Tw} de 0.25, aunque solo opere con 3 \acrshort{svm}s por comité (realmente 1 \acrshort{svm}, por el percentil 25).

\subsection{Discusiones}
\label{sec:discusparamtunning}

Los resultados de precisión en el modo \textit{Open-World} son claramente \textbf{inferiores} al modo \textit{Open-Set}. La principal causa es la generación de \textbf{comités duplicados} en el modo \textit{Open-World}, que se crean como respuesta a ciertas condiciones del individuo (ejemplo: una orientación diferente de la cara) que el comité original no es capaz de reconocer, de forma que se adueñan de parte de su identidad. Por este motivo, es necesario implementar un método que permita \textbf{asociar comités duplicados} y/o mejorar el reconocimiento por medio del \acrfull{evt}.

%TODO: no lo creo, sino como se explica la subida en el recall???? Creo q el sistema es capaz de crear comités que se ajusten mejor a las situaciones concretas del individuo que con el comité original no sucede (ejemplo: una determinada orientación de la cara), esto quiere decir que . El principal factor de esta caída es la existencia de \textbf{comités duplicados} para un individuo, lo que aumenta las probabilidades de otorgar un reconocimiento \textbf{desconocido}. Para colmo, cabe la posibilidad de que las \acrshort{svm} que pertenecen a los comités duplicados se entrenen con muestras del \textbf{mismo individuo} en los conjuntos \textbf{positivo y negativo}, lo que podría \textbf{anular el poder discriminatorio} de dichas \acrshort{svm}, lo que implica en una \textbf{pérdida de precisión}. Por otro lado, el \textit{recall} aumenta, precisamente por la creación de nuevos comités, que a.

El valor del parámetro \textbf{\textit{Tw}} es \textbf{clave} para el correcto funcionamiento del sistema, mientras que en el modo \textit{Open-Set} un valor bajo de este parámetro resulta en un balance muy favorable para la precisión y el \textit{recall}, en el modo \textit{Open-World} es \textbf{perjudicial} para ambas métricas, debido a que fomenta la creación de comités duplicados. En cambio, un \textbf{valor alto} de este parámetro en el modo \textbf{\textit{Open-World}}, aunque a priori sea negativo para la precisión, es la \textbf{mejor opción} para reducir la creación de comités duplicados.

Uno de los objetivos del sistema es obtener la \textbf{mayor diversidad intra-comité} y la \textbf{mayor especificidad inter-comité}, de esta forma, el comité responde a la mayor variedad de situaciones de su entidad, mientras que no se activa ante otros individuos. En este sistema, una proporción de las \acrshort{svm} creadas \textbf{son muy similares entre sí}, dando lugar a poca información nueva del individuo. Esto se debe principalmente al \textbf{\textit{step\_size}} introducido en las pruebas, que hace que las \acrshort{svm}s compartan frames entre sí.

%TODO: porqué un 40% de las SVM devuelven respuestas negativas ante sus muestras?, no creo q sea porque las svm son de otros comités (quizá 1 si), si son muy especificas eso puede explicarlo, lo bueno es que en todas las situaciones que he visto siempre hay al menos 3/4 svms activadas, por lo que dentro de lo especifico no lo es tanto.

%Disponer de un conjunto mayor de negativos para entrenar las \acrshort{svm} podría mejorar las respuestas de los comités.

%Un \textbf{percentil alto} encajaría si las SVM son \textbf{específicas} entre sí (la mayoría de las SVM devuelven match). En cambio, un \textbf{bajo percentil} funcionaría con unas SVM \textbf{más diversas} (se espera que la minoría de las SVM den un resultado coincidente).

%TODO: En relación con el anterior punto, el número de SVM influye en la decisión de que percentil tomar. Si se utilizan 3 SVM por comité y la mediana, al menos 2 SVM deben de devolver un score bajo para dar un resultado coincidente. Por otro lado, si se utilizan 10 SVM y un percentil de 30, sólo 3 de las 10 SVM necesitan estar de acuerdo para devolver un match. En los resultados se muestra como un comité de 3 SVM y TW=0.05 destaca respecto a un comité de 1 SVM y TW=0.5. Los resultados utilizando 10 SVM por comité no llegan a mejorar notablemente el valor de 3 SVM, esto es de gran relevancia, puesto que se reduce el número de iteraciones de cada reconocimiento de 10*n a 3*n, siendo n el número de comités que existen y asumiendo que todos los comités han llegado al límite de SVM. Con estos resultados se ha demostrado que implementar varias SVM por comité mejora el recall manteniendo la precisión. 1 sóla SVM por comité es efectiva si dicha SVM es representativa del individuo, en caso contrario puede devolver respuestas coincidentes acerca de una entidad distinta del comité, generando ruido que perjudica a las predicciones de Weibull. Un mayor número de SVM combinado con el percentil devuelve un resultado conforme dicta un grupo que omite los scores de SVM intrusas (de entidades distintas) y permite expulsarlas mediante los criterios ya comentados en la sección \ref{seq:limmod}.

%La calidad de la función de reconocimiento viene determinada por el número de puntos que componen la función de Weibull, cuantos más puntos, mayor definición y por tanto reconocimientos más precisos \cite{Erik}. Con un universo de 20 individuos, solo 5 puntos componen la función de Weibull (de las 20 entidades solo 10 se registran en el sistema, a modo de probar el open-set, y de esos 10, la mitad definen la función), lo que lleva a un comportamiento más impreciso que con un universo mayor. El sistema inicial cuenta con un universo de \textbf{6 individuos}, lo que claramente \textbf{es insuficiente} para realizar una diferenciación precisa de los IoI respecto a lo desconocido.

\section{Métodos de inicialización del sistema}

En esta sección se evalúa el sistema en los modos \textit{Open-Set} y \textit{Open-World} en las 2 técnicas de inicialización vistas en la sección \ref{sec:initieing} (semisupervisada y no supervisada). Para mostrar el notorio impacto del parámetro \textbf{\textit{Tw}} en los resultados, se han probado los valores \textbf{0.25} y \textbf{0.1} en todas las pruebas de la sección. El resto de valores de los parámetros son los mismos de la anterior sección, salvo por los parámetros \textbf{\textit{max\_svm}}, \textbf{\textit{percent}} y \textbf{\textit{nneg}}, a los que se les ha asignado los valores \textbf{10, 25 y 50} respectivamente, acorde a los resultados obtenidos en la anterior sección.

En la técnica no supervisada, el sistema ha podido encontrar como máximo \textbf{5 personas} en un instante del vídeo (el mínimo necesario, como se expone en la sección \ref{sec:initieing}) entre las 2 cámaras. Debido a que el \gls{dataset} contiene \textbf{6 personas}, se ha decidido añadir a la entidad restante durante la operación del sistema, en el momento que el módulo de reconocimiento adaptativo reconozca a un desconocido.

%La situación ideal sería encontrar un instante en el que todas las personas del vídeo (6 en total) aparezcan. Como mínimo para que el sistema se inicialice es necesario que aparezcan 5 entidades, que equivale al número mínimo de puntos necesarios para definir la distribución de Weibull. En el \gls{dataset} de prueba solo aparece un instante de 5 individuos que cumpla con las condiciones anteriores. Por lo tanto, es necesario añadir a la entidad restante durante la inferencia en el caso de la técnica no supervisada.

\subsection{Modo \textit{Open-Set}}

\begin{table}[]
    \centering
    \begin{tabular}{c|cc|cc}
        \hline
        \textit{}          & \multicolumn{2}{c|}{\textit{Semisupervised}} & \multicolumn{2}{c}{\textit{Non supervised}}                                          \\ \hline
        \textit{Tw}        & \multicolumn{1}{c|}{0.25}                    & 0.1                                         & \multicolumn{1}{c|}{0.25} & 0.1        \\ \hline
        \textit{Precision} & \textbf{79.1 ± 1.2}                          & \textbf{79.6 ± 3.2}                         & 66.7 ± 1.7                & 69.1 ± 3.1 \\
        \textit{Recall}    & \textbf{95.1 ± 1.3}                          & \textbf{89.6 ± 1.7}                         & 89.2 ± 0.9                & 80.2 ± 2.5 \\
        \textit{F1\_score} & \textbf{86.3 ± 0.9}                          & \textbf{84.2 ± 2.2}                         & 76.3 ± 1.2                & 74.2 ± 1.5 \\ \hline
    \end{tabular}
    \caption{Resultados según la inicialización del sistema en el modo \textit{Open-Set}}
    \label{tab:osinittab}
\end{table}

\begin{figure}[tbp]
    \centering
    \begin{subfigure}[t]{0.3\textwidth}
        \includegraphics[width=\textwidth]{imagenes/supervised.jpg}
        \caption{modo semisupervisado}
    \end{subfigure}
    \hspace{2cm}
    \begin{subfigure}[t]{0.3\textwidth}
        \includegraphics[width=\textwidth]{imagenes/nosupervised.jpg}
        \caption{modo no supervisado}
    \end{subfigure}
    \caption{Conjuntos de muestras según la inicialización}
    \label{fig:inits}
\end{figure}

En la tabla \ref{tab:osinittab} se muestran los resultados para los 2 tipos de inicialización. La figura \ref{fig:inits} expone un ejemplo de las muestras recogidas según la técnica. En la izquierda, puede verse que las muestras tienen una buena resolución, apenas poseen borrosidad, contienen suficiente luminosidad y son todas caras frontales, justo lo contrario que las muestras de la derecha. Por un lado, la buena calidad de las muestras de la inicialización semisupervisada (izquierda de la figura \ref{fig:inits}) permiten al sistema responder ante una mayor variedad de situaciones del individuo, por lo tanto es la que mejores resultados arroja. Por otro lado, la baja calidad y escasa variedad de las muestras escogidas del modo no supervisado (derecha de la figura \ref{fig:inits}) limitan la capacidad del sistema de reconocer la entidad ante cambios en las condiciones (ejemplo: orientación de la cara) y de recoger las características que la diferencian del resto de individuos. Esto supone la principal causa del descenso del \textit{F1\_score} en 10 puntos respecto al modo semisupervisado.

A pesar de la caída en el rendimiento en la técnica no supervisada, se logra que el sistema \textbf{no colapse}, lo que indica que el módulo de procesamiento de vídeo y la inclusión del nuevo individuo durante la operación del sistema \textbf{funcionan correctamente}.

Atendiendo al valor del parámetro \textit{Tw}, el valor de 0.25 obtiene un mejor \textit{F1\_score}, ya que a pesar de ser levemente inferior en la precisión respecto al valor 0.1, el \textit{recall} mejora con creces, demostrando así que el sistema puede resolver más reconocimientos sin apenas impacto en la precisión.

\subsection{Modo \textit{Open-World}}

\begin{table}[]
    \centering
    \begin{tabular}{c|cc|cc}
        \hline
        \textit{}                                                              & \multicolumn{2}{c|}{\textit{Semisupervised}} & \multicolumn{2}{c}{\textit{Non supervised}}                                          \\ \hline
        \textit{Tw}                                                            & \multicolumn{1}{c|}{0.25}                    & 0.1                                         & \multicolumn{1}{c|}{0.25} & 0.1        \\ \hline
        \textit{Precision}                                                     & \textbf{66.0 ± 4.0}                          & \textbf{55.2 ± 3.4}                         & 45.0 ± 4.7                & 20.9 ± 5.8 \\
        \textit{Recall}                                                        & \textbf{95.8 ± 0.8}                          & \textbf{90.0 ± 1.9}                         & 92.4 ± 1.2                & 72.8 ± 5.4 \\
        \textit{F1\_score}                                                     & \textbf{78.1 ± 2.9}                          & \textbf{68.4 ± 3.1}                         & 60.4 ± 4.2                & 32.2 ± 7.4 \\
        \textit{\begin{tabular}[c]{@{}c@{}}Number of\\ ensembles\end{tabular}} & \textbf{19 ± 2}                              & \textbf{32 ± 3}                             & 25 ± 2                    & 43 ± 5     \\ \hline
    \end{tabular}
    \caption{Resultados según la inicialización del sistema en el modo \textit{Open-World}}
    \label{tab:owinittab}
\end{table}

En la tabla \ref{tab:owinittab} se muestran los resultados esta vez para \textit{Open-World}. Se puede apreciar una caída en la precisión de la técnica no supervisada, mayor que en el modo \textit{Open-Set}. Al jugar con la inclusión de nuevos comités, el sistema se decanta por crear un nuevo comité si el reconocimiento no es claro, lo que se acentúa en muestras extraídas durante la operación (generalmente de baja calidad), que conforman comités que responden a menor variedad de situaciones del individuo y, por ende, a generar más comités que cubran dichas situaciones.

En el caso del modo \textit{Open-World}, la variación en el umbral de Weibull (\textit{Tw}) tiene un mayor impacto respecto a \textit{Open-Set}, debido a la creación de nuevos comités que, por un lado, ayuda a potenciar el \textit{recall}, mientras que por otro la precisión cae significativamente. En el caso de la técnica no supervisada, el valor de 0.1 genera una \textbf{caída dramática} respecto a unos resultados de por sí \textbf{ínfimos} al usar un valor de 0.25. La acumulación excesiva de comités duplicados puede volverse contraproducente, ya que, a pesar de que por un lado ayuda a definir mejor la distribución de Weibull con más puntos, por otro se devuelven puntuaciones más cercanas a la ganadora, debido a las similitudes entre los comités, lo que genera \textbf{indecisión} en el sistema y por ende a que el \textit{recall} y la precisión disminuyan.

\subsection{Discusiones}

Mientras que el modo \textit{Open-Set} logra mantener el tipo en ambas técnicas de inicialización, el modo \textit{Open-World} \textbf{colapsa} en la técnica no supervisada. La baja calidad de las muestras iniciales impide la extracción de las características discriminatorias de un individuo respecto al resto del mundo, lo que supone la causa principal del deterioro en el rendimiento, acentuado en el caso del modo \textit{Open-World}, debido a la creación de más comités con poca diferenciación entre sí, lo que conduce a más dudas en los reconocimientos y por consiguiente al desmoronamiento del sistema.

El inevitable ruido que contienen las muestras también es uno de las principales causas de la caída en la precisión, por lo que es necesario adoptar métodos para hacer una selección según el grado de borrosidad u orientación de la cara, más allá del filtrado inicial propuesto y de añadir las muestras con puntuaciones cercanas al 0, que no implica necesariamente que sean las mejores para mejorar la generalización de los comités.
\chapter{Conclusiones y trabajo futuro}
\label{chap:conclusiones}

\lettrine{Ú}{l}timo capítulo en el que exponen las conclusiones de los resultados y los hitos alcanzados, así como los errores cometidos y las lecciones aprendidas. También se comentan algunas vías en las que reforzar o investigar para extender el trabajo de este proyecto.

\section{Conclusiones}

La ejecución de los modelos de redes neuronales y del sistema en múltiples arquitecturas era uno de los objetivos principales, que se ha cumplido con hasta 6 arquitecturas distintas. Los resultados arrojados muestran las similitudes y diferencias entre las arquitecturas de dichos dispositivos. Adicionalmente, se ha probado el sistema ante cargas de trabajo crecientes, cuyo objetivo es acotar el número máximo de cámaras que el sistema puede soportar, todo operado en tiempo real.

Se ha exprimido la capacidad de las \acrshort{gpu}s de todos los dispositivos del proyecto. Los resultados del sistema obtenidos en las placas NVIDIA Jetson llegan incluso a superar el sistema de referencia, que dispone de una \acrshort{cpu} superior, demostrando así la capacidad de las \acrshort{gpu}s de hacer frente a cargas de trabajo relacionadas con la visión artificial.

La integración del sistema en contenedores de Docker fue clave para el éxito del análisis, ya que de otro modo no sería posible integrar el sistema en todas las arquitecturas, teniendo en cuenta que 3 de los dispositivos (Orin Nano, AGX Orin y AGX Thor) pasaron a estar disponibles con tan sólo 3 meses de antelación (desde finales de Septiembre).

Por motivos relacionados con el fin de soporte de \acrshort{ros} 1 y de poder aplicar las últimas librerías disponibles para los distintos dispositivos, se ha realizado una migración del sistema a \acrshort{ros} 2.

Por otra parte, se ha incorporado una nueva funcionalidad, que permite al sistema actualizar su conocimiento ante nuevos individuos detectados. Fue necesario cambiar el funcionamiento del sistema base, de forma que pueda procesar secuencias de frames en vez de un solo frame. Debido al contexto de las pruebas, que implican a múltiples individuos que se entrecruzan, se han tenido que desarrollar algoritmos que mantengan el rastro de dichos individuos, con capacidad de reidentificación ante la pérdida de visión los mismos. Todos los componentes se han sometido a análisis, que exponen el comportamiento del sistema desde múltiples puntos de vista. TODO: resultados

\subsection{Seguimiento del proyecto}

Las principales causas de los retrasos en el proyecto fueron las pruebas de rendimiento de los modelos y del sistema.

Aunque el motor de Docker despliegue el mismo software con el mismo \acrshort{so} y las mismas versiones de las librerías, las arquitecturas de los dispositivos hacen que los resultados de latencia y precisión cambien completamente.

También se producen fallos en las fases menos esperadas, por ejemplo, al momento de exportar el modelo YOLO11n a \acrshort{onnx}. En todos los dispositivos, el modelo se exporta sin problema utilizando PyTorch en \acrshort{cpu}, a excepción de la Jetson AGX Orin, cuya exportación se realizó a partir de la versión de PyTorch para \acrshort{gpu}.

Por otro lado, aunque el sistema de partida permitía operar con más de 2 cámaras, este era muy sensible a los retardos de cualquiera de las cámaras, además de que no estaba pensado para funcionar asumiendo cierto solape entre las cámaras. Por estos motivos, se tuvo que rediseñar parte de la lógica del sistema para lograr la escalabilidad.

\section{Trabajo Futuro}

Los nuevos componentes de la arquitectura extendida (sección \ref{sec:finalsys}) se han probado de forma aislada bajo un modelo de reconocimiento facial, debido a no disponer del tiempo suficiente. La integración de dichos componentes en el sistema implicaría muchos cambios en el código, aunque no debería de afectar significativamente a la lógica del sistema base ni a la interacción con \acrshort{ros}.

El método de reconocimiento adaptativo se ha probado únicamente con recortes faciales, como en \cite{Erik, CESAR}. Se cree que dicho método puede funcionar para recortes de cuerpos.

El modo \textit{Open-World} no se ha comportado según lo previsto. Se ha observado que dicho sistema es \textbf{muy sensible} a la hora de ajustar el parámetro del umbral de Weibull. Ajustando el umbral a un valor bajo, el número de comités nuevos por entidad se dispara, ya que añade incertidumbre al sistema rápidamente (puesto que no se generan casos extremos si se activa más de un comité). Un valor alto del umbral mitiga este problema, a costa de comprometer la precisión. Se podría estudiar la implementación de un nuevo criterio para agrupar los comités y así solventar esta problemática.

%Se ha comprobado que el rendimiento del reconocimiento adaptativo es similar entre 1 y 10 \acrshort{svm}s por comité. Este fenómeno puede justificarse por la alta especificidad .

El método de reidentificación de entidades mediante la distancia Euclidiana reporta problemas cuando las cámaras cambian de posición (ejemplo: el robot se desplaza), ya que se guardan las coordenadas \acrshort{2d} de una persona que pasan a ocupar el lugar de otro individuo. La imagen de distancia de las cámaras Kinect puede otorgar una posición en el espacio \acrshort{3d} de forma que la posición de la persona desaparecida se mantiene.

En el anexo \ref{chap:coherence} se comenta la implementación de un módulo para la resolución de conflictos cuando dos cámaras \textbf{sin solape} predicen al mismo individuo. Debido a la nula mejora en el rendimiento por los motivos comentados, sería necesario solventar los problemas del método de clasificación implementado, o incluso eliminarlo y replantear un método nuevo.

%%%%%%%%%%%%%%%%%%%%%%%%%%%%%%%%%%%%%%%%
% Apéndices, glosarios y bibliografía  %
%%%%%%%%%%%%%%%%%%%%%%%%%%%%%%%%%%%%%%%%

\appendix
\chapter{Reconocimiento adaptativo}
\label{chap:adaptrecon}

\lettrine{E}{n} este apéndice se presenta el diseño y la implementación de la capacidad de reconocimiento con adaptación a los cambios (o aprendizaje incremental) propuesta en \cite{Erik, CESAR}.

\section{Diseño}

\begin{figure}[tbp]
    \centering
    \includegraphics[width=1\linewidth]{imagenes/ADAPTSYS.jpg}
    \caption{Diseño del módulo de reconocimiento adaptativo}
    \label{fig:cloneADAPTSYS}
\end{figure}

En la figura \ref{fig:cloneADAPTSYS} se muestra la arquitectura del módulo de reconocimiento con capacidad de adaptación a los cambios, ya introducido en la sección \ref{sec:archrecon}. A continuación se comenta en profundidad todos sus componentes.

\subsection{Módulo de valoración}
\label{sec:val}

Es el módulo encargado de devolver una puntuación comparable de cada individuo que será aplicada en la decisión de reconocimiento.

Por cada secuencia de entrada, siendo esta una secuencia de \textit{\gls{embedding}} (figura \ref{fig:cloneADAPTSYS}), se calculan las \textbf{puntuaciones (scores) para cada comité}. La puntuación de un comité es a su vez un valor consensuado entre los resultados de las predicciones de las \acrshort{svm} que lo conforman, el criterio de consenso (o de fusión) se basa en un percentil (generalmente la media). Aplicar percentiles es igual a escoger un conjunto amplio o reducido de \acrshort{svm}, ya que pueden existir \acrshort{svm} dañinas para el comité (ejemplo: corresponden a otra persona), los percentiles ayudan a tolerar los resultados de dichas \acrshort{svm}.

Para cada comité, se ejecutan las siguientes funciones:
\begin{itemize}
    \item \textbf{FDF} (Frame Decision Function): se calculan los scores de cada \acrshort{svm} contra \textbf{un frame} de la secuencia y se fusionan las salidas (o puntuaciones) en una única puntuación, que representa la puntuación del comité para dicho frame. Antes de la fusión, se aplica la normalización Euclidiana a cada puntuación con el fin de hacerlas comparables. Siendo $x_{i}$ un \textit{\gls{embedding}} o la salida de una \acrshort{svm}, se obtiene el vector normalizado $s_{i}$ como sigue en la ecuación \ref{eq:l2norm}.
    \item \textbf{SDF} (Sequence Decision Function): se encarga de fusionar todas las puntuaciones de la anterior función para obtener un único resultado que representa a la \textbf{secuencia}.
\end{itemize}

\begin{equation}
    s_{i} = \frac{x_{i}}{\left\| x_{i}\right\|_{2}} \label{eq:l2norm}
\end{equation}

\subsection{Módulo de reconocimiento}
\label{sec:reckon}

Este módulo determina si la entidad detectada se corresponde a un individuo previamente reclutado o a una entidad \textbf{desconocida}. Para tomar la decisión de si un sujeto es o no un desconocido, se ha implementado un algoritmo basado en el teorema estadístico \acrfull{evt} o teorema de Fisher–Tippett–Gnedenko, que permite \textbf{detectar extremos} en distribuciones estadísticas. Dicho teorema dicta que la distribución de los valores máximos o mínimos sigue una de las siguientes 3 distribuciones: Gumbel, Frechet, Weibull \cite{rudd2017extreme, scorenorm, Erik}. Gumbel y Frechet son distribuciones \textbf{no acotadas}, mientras que Weibull es una distribución acotada. Para sistemas de reconocimiento que calculan las distancias o similitudes entre puntuaciones (ejemplo: valores devueltos por una \acrshort{svm} para una clase), la distribución de puntos para cada clase sigue una distribución de Weibull, ya que los valores se encuentran acotados \cite{scorenorm}.

Siendo \textit{f(x)} la distribución de puntuaciones de los comités \textbf{que no corresponden con el sujeto}, se analiza si la puntuación del comité que se corresponde supuestamente con el sujeto es un \textbf{extremo} de la distribución \textit{f(x)}, en dicho caso, la entidad se reconoce claramente diferenciada con el resto, en caso contrario, se denota como \textbf{desconocido}.

\begin{figure}
    \centering
    \includegraphics[width=1\linewidth]{imagenes/FITS.jpg}
    \caption{Ejemplos de distribuciones de Weibull y como el umbral (Tw) distingue entre un desconocido (\textit{unknown}) y una deriva (\textit{drift}).}
    \label{fig:FITS}
\end{figure}

Se calcula la probabilidad de pertenencia a la distribución de la puntuación en cuestión, es decir, la salida de la función \acrfull{pdf} de Weibull. Si la probabilidad se encuentra por debajo de un \textbf{umbral} (Tw en la figura \ref{fig:cloneADAPTSYS}), entonces se trata de un \textbf{extremo} \cite{Erik, scorenorm}. En la situación de la derecha de la figura \ref{fig:FITS} se muestra un caso en el que la puntuación ganadora es un extremo, al contrario de la gráfica de la izquierda.

Las puntuaciones de todos los comités se ordenan por su valor (función sort de la figura \ref{fig:cloneADAPTSYS}), siendo la mejor puntuación la primera de la lista, es decir, la más baja (en la sección \ref{sec:init} se explica el porqué) y el resto de puntuaciones las no coincidentes.

El algoritmo expuesto otorga un conocimiento robusto, ya que convierte puntuaciones (o scores) concretos de una \acrshort{svm} en probabilidades que siguen una teoría estadística. Esto permite la \textbf{fusión de datos de diferentes fuentes} como nuevas \acrshort{cnn} de reconocimiento o nuevas cámaras diferentes a las Kinect en el sistema \cite{scorenorm}.

\subsection{Módulo de actualización}
Es el módulo que implementa la actualización de los comités tras el reconocimiento realizado en la anterior función. Según la entidad predicha, se toma una cierta decisión de actualización. Si el sujeto es \textbf{desconocido}, se crea un nuevo comité con una \acrshort{svm} que se añade al registro. La \acrshort{svm} se entrena con las muestras (\textit{\glspl{embedding}} de caras o cuerpos) obtenidas del propio sujeto como conjunto de positivos, mientras que el conjunto de negativos se compone de muestras aleatorias extraídas directamente de la base de datos. Si el sujeto es \textbf{conocido}, se agrega una nueva \acrshort{svm} al comité, entrenada con las muestras del propio individuo como positivos y como negativos un muestreo aleatorio de todas las entidades \textbf{excepto el propio sujeto}.

Una condición necesaria para que este módulo se ejecute es que la secuencia de entrada \textbf{contenga un mínimo de \textit{\glspl{embedding}} para la inicialización de la \acrshort{svm}}, dicho mínimo es fijado por el operador de antemano.

Otra precondición, que aplica a las entidades conocidas, es comprobar si las muestras a añadir son lo suficientemente representativas. Las puntuaciones cerca del cero indican que las muestras se encuentran en el borde de lo que es nuevo y de lo que la \acrshort{svm} ya conoce. Si la mediana de las puntuaciones de los \glspl{embedding} para un comité se encuentra por debajo de un umbral (\textit{update\_th}), se procede a seleccionar las muestras más cercanas al 0 (función \textit{sample\_selection} en la figura \ref{fig:cloneADAPTSYS}), que compondrán el conjunto de positivos de la \acrshort{svm}.

\textbf{El tamaño del conjunto de positivos y de negativos es prefijado por el operador}, en la sección \ref{seq:paramtunning} se evalúa su impacto en el rendimiento.

\subsection{Módulo de limitación}
\label{seq:limmod}

Dicho módulo se encuentra inherente al de actualización. Si un comité excede un número prefijado de \acrshort{svm} almacenadas, se toma una decisión para eliminar una de las \acrshort{svm} según los siguientes criterios.

\subsubsection{Diversidad}

Este criterio devuelve un valor que representa el grado de diferenciación de una \acrshort{svm} respecto a su comité. Se escoge un conjunto aleatorio de \textit{\glspl{embedding}} de entre todos los comités y se calcula la puntuación de cada \acrshort{svm} del comité. Posteriormente, se acumula el producto de los signos entre scores y se multiplica por -1, de esta forma, el \acrshort{svm} más discordante (es decir, el único que devuelve un resultado contrario respecto a una amplia mayoría) obtendrá un mayor valor de este criterio y viceversa. El valor de diversidad de cada \acrshort{svm} se calcula como en \cite{Erik}. Dado un conjunto de \glspl{embedding} \{$x_{0}$, $x_{1}$, ..., $x_{Q-1}$\} y un comité $e^{k}$ con N \acrshort{svm}s \{$h^{k}_{0}$, $h^{k}_{1}$, ..., $h^{k}_{N-1}$\}, $D(h^{k}_{i})$ se calcula como sigue en la ecuación \ref{eq:diversity}, donde \textit{sgn} es la función \textit{sign}, que devuelve 1 o -1 dependiendo del signo del número real.

\begin{align}
    D(h_{i}^{k})= \sum_{j=0;j\neq i}^{N-1} d(h_{i}^{k}, h_{j}^{k}) \label{eq:diversity} \\
    d(h_{i}^{k}, h_{j}^{k}) = -\frac{1}{Q}\sum_{q=0}^{Q-1} sgn(h_{i}^{k}(x_{q})) \cdot sgn(h_{j}^{k}(x_{q}))
\end{align}

\subsubsection{Coherencia}

Este criterio viene a determinar la precisión de una \acrshort{svm} en el reconocimiento respecto a las salidas de su comité. El valor de coherencia determina cuantas veces una \acrshort{svm} devuelve el mismo signo que el resultado consensuado del comité para un conjunto de \textit{\glspl{embedding}}. Si los signos coinciden, se suma 1 al valor de coherencia, en caso contrario, se resta 1 a dicho valor, de forma que un valor alto se corresponde con una buena precisión. En el momento de creación de una \acrshort{svm}, este valor se inicializa a 0 y se va acumulando. Dado un comité ganador $e^{k}$ de N \acrshort{svm}s \{$h^{k}_{0}$, $h^{k}_{1}$, ..., $h^{k}_{N-1}$\} y un conjunto de \textit{\glspl{embedding}} \{$x_{0}$, $x_{1}$, ..., $x_{Q-1}$\}, $C_{k,i}$ se calcula como sigue en \cite{CESAR} (ecuación \ref{eq:coherence}), donde $1[\cdot]$ representa a un condicional que devuelve 1 si la condición se cumple o 0 en caso contrario.

\begin{align}
    C_{k,i} = 1[sgn(e^{k}) == sgn(\bar{y})] - 1 [sgn(e^{k}) \neq sgn(\bar{y})] \label{eq:coherence} \\
    \bar{y} = \frac{1}{Q} \sum_{q=1}^{Q-1} h_{n}^{k} (x_{q})
\end{align}

\subsubsection{Favorabilidad}

Finalmente, los dos criterios anteriores se fusionan en un valor llamado \textbf{índice de favorabilidad}. La fórmula para calcular dicho índice es la misma que en \cite{CESAR}. Dado el valor de coherencia $C_{k,i}$ de la \acrshort{svm} $h^{i}$ del comité ganador $e^{k}$ y el valor de diversidad $D(h^{m})$, la favorabilidad se calcula como sigue en la figura \ref{eq:fav}, donde son $\alpha$ y $\gamma$ constantes que ajustan el peso de la coherencia y la diversidad respectivamente.

\begin{equation}
    F(h_{i}^{k}) = \alpha C_{k,i} + \gamma D(h_{i}^{k}) \label{eq:fav}
\end{equation}

Como en los dos criterios anteriores, cuanto mayor sea el valor devuelto, mejor se valora la \acrshort{svm}. De este modo, la \acrshort{svm} con el menor valor de dicho índice \textbf{se elimina del comité}.

\section{Implementación}

En esta sección se explican algunos de los detalles de implementación de los componentes que conforman esta funcionalidad. Para la mayoría de componentes se ha tomado como base el siguiente repositorio \cite{src}.

\subsection{Módulo de valoración}

Como se expuso en la sección \ref{sec:val}, el módulo de valoración devuelve un conjunto de puntuaciones que representan a la secuencia para todos los comités.

La implementación sigue el funcionamiento descrito en la sección \ref{sec:val}. Para calcular los percentiles y la mediana en las funciones FDF y SDF se utilizan las funciones \textit{percentile} y \textit{median} respectivamente de la librería NumPy.

\subsection{Módulo de reconocimiento}

\begin{lstlisting}[language=Python, float=t, label=coud:weib, caption=Módulo de reconocimiento, basicstyle=\footnotesize]
import numpy as np
from scipy import stats

def weib(x,scale,shape):
    if x < 0:
      return 0
    return (shape / scale) * (x / scale) ** (shape - 1) * np.exp(- (x / scale) ** shape)

def RDF(R, sorted_ens_scr, Tw):
  #R: original set of ensemble scores
  #sorted_ens_scr: sorted set of ensemble scores
  #Tw: Weibull threshold
  c = sorted_ens_scr[0]
  set_of_mins = sorted_ens_scr[1:int(len(sorted_ens_scr))]
  m = np.median(sorted_ens_scr[1:])
  v = np.abs(set_of_mins - m) #Non-match distribution
  cand_scr = np.abs(c - m)  #Best score
  #Obtain shape and scale Weibull parameters
  shape,loc,scale = stats.weibull_min.fit(v, floc=0)  
  pdf = weib(cand_scr, scale, shape)  #Calculate PDF
  # Decision
  if (pdf<Tw):
    decision = np.where( R == c )[0][0] # Drift
  else:
    decision = -1 # Unknown
  return decision
\end{lstlisting}

Como se mencionó en la sección \ref{sec:reckon}, el módulo de reconocimiento determina si la entidad detectada se corresponde a un individuo previamente reclutado o a una entidad \textbf{desconocida}. El código \ref{coud:weib} muestra la implementación de dicha funcionalidad. Se parte de la función RDF (Recognition Decision Function), que recibe el conjunto de puntuaciones ordenado y devuelve la decisión de reconocimiento.

Del conjunto de puntuaciones ordenadas, se excluye la mejor puntuación de todos los comités (es decir, la puntuación más baja) y se compone la distribución de puntuaciones no coincidentes. Se calcula para cada punto la distancia respecto a la mediana (variable v), de forma que los puntos de la distribución se distancian del mejor comité. Se obtienen los parámetros \textit{shape} y \textit{scale} a partir de la distribución, que modelan la función de Weibull. Finalmente, se calcula la probabilidad de pertenencia del mejor comité a la distribución (función weib) y se toma la decisión en base a un umbral (Tw). Si la probabilidad es inferior al umbral, se reconoce el caso como un extremo y se asigna la identidad correspondiente al sujeto (\textit{drift}), en caso contrario se devuelve como desconocido (\textit{unknown}).

Para obtener los parámetros que componen la distribución de Weibull, se utiliza la función \textit{fit} de la clase \textit{weibull\_fit} del módulo \textit{stats} de la librería SciPy.

\subsection{Módulo de actualización}

La implementación del módulo sigue el flujo descrito en la sección \ref{sec:reckon}. En el caso de una clase conocida, la nueva \acrshort{svm} se guarda en el comité junto a la selección de positivos, que sirven para entrenar nuevas \acrshort{svm}, en el caso contrario se crea una nueva entrada en la base de datos con la nueva etiqueta del individuo (\textbf{person\_id}), el comité con su \acrshort{svm} inicial (\textbf{ensemble}) y la selección de positivos (\textbf{descriptors}) y la lista de valores de coherencia (\textbf{coherence}) inicializada a 0. En la selección de los positivos se aplica la función de valor absoluto a la lista de puntuaciones de las muestras, que posteriormente se ordena de menor a mayor y se recuperan los primeros valores, que son los más cercanos al 0.

La creación de las \acrshort{svm} se realiza a partir de la función \textit{def\_svm} expuesta en la sección \ref{sec:init}.

\subsection{Módulo de limitación}
En el \textbf{módulo de actualización}, si el comité excede el máximo de \acrshort{svm} establecido, este módulo se activa para hallar la \acrshort{svm} que menos contribuye, por tanto la que debe de abandonar el comité.

Se invoca a la función que calcula el criterio de favorabilidad, que otorga un valor para cada \acrshort{svm} a partir de la suma de los criterios de diversidad y coherencia. Se almacenan los resultados en un array de NumPy y se ordenan los valores devueltos mediante la función \textit{argsort} de NumPy, por lo que la \acrshort{svm} a eliminar se corresponde con la primera entrada del vector (es decir, el que tenga el valor de favorabilidad más bajo).

Los valores de coherencia se calculan sobre el valor acumulado de cada \acrshort{svm} cada vez que se invoca a este módulo. Por simpleza, se ha optado por reiniciar este valor en todo el comité a 0 cada vez que se reemplaza una \acrshort{svm}, de forma que se mantenga una comparación justa entre todas las \acrshort{svm}.
\chapter{Códigos de la implementación del sistema}

\lettrine{E}{n} este apéndice se muestran los códigos de implementación explicados en el capítulo \ref{chap:impl}.

%, basicstyle=\footnotesize]
\begin{lstlisting}[language=Python, float=t, label=coud:training, caption={Entrenamiento de las SVM, sección \ref{sec:training}}]     
# sample = set of embeddings from CNNs, for ArcFace and OSNet_x1, each sample is composed of 512 embeddings
def def_svm(embeddings, idx, d1, unk):
# embeddings shape: (num_entities, num_svm, num_samples, embeddings)
# idx: index that points to the entity ensemble in the database
# d1: positive samples 
# unk: flag to mark if d1 samples belong to an unknown entity
# nneg: number of samples that make up the set of negatives.
    neg_embeddings = deepcopy(embeddings)
    if not unk:
        neg_embeddings.pop(idx)
    neg_samples = [neg_samples for num_entities in neg_embeddings for num_svm in num_entities for num_samples in num_svm]
    k = nneg if len(neg_samples) > nneg else len(neg_samples)
    neg_samples = random_pick(np.array(neg_samples), size=k) # Negative selection
    trainData = np.vstack([neg_samples, d1], dtype=np.float32)
    nlabels = -np.ones([neg_samples.shape[0], 1], dtype=np.int32)
    plabels = np.ones([len(d1), 1], dtype=np.int32)
    labels = np.vstack([nlabels, plabels])
    svm = cv2.ml.SVM_create()
    svm.setType(cv2.ml.SVM_C_SVC)
    svm.setKernel(cv2.ml.SVM_LINEAR)
    svm.setClassWeights(np.array([1,100], dtype = np.float32))
    svm.setTermCriteria((cv2.TERM_CRITERIA_COUNT, 1000, 1.e-06))
    svm.train(trainData.astype(np.float32), cv2.ml.ROW_SAMPLE, labels.astype(np.int32))
    return svm
\end{lstlisting}

\begin{lstlisting}[language=Python, float=t, label=coud:trtskel, caption={Esqueleto del código de las inferencias con TensorRT, sección \ref{sec:trtinfer}}]
import tensorrt as trt
import pycuda.driver as cuda

# BATCH_SIZE: number of images to be processed on a single inference
# NUM_CHANNELS: number of image channels, typically 3 (red, green and blue)
# width, height: width and height of input image
# features: number of features provided by model output
# ifp: input tensor float precision of the model
# ofp: output tensor float precision of the model
class CNN_model:
    def __init__(self, cuda_ctx):
        super().__init__()
        self.ctx = cuda_ctx
        self.ctx.push()
        f = open(str('path_to_trt_engine'), "rb")
        runtime = trt.Runtime(trt.Logger(trt.Logger.WARNING)) 
        self.cuda_engine = runtime.deserialize_cuda_engine(f.read())
        self.context = self.cuda_engine.create_execution_context()
        self.stream = cuda.Stream()
        self.input_size = BATCH_SIZE * NUM_CHANNELS * width * height * ifp().itemsize
        self.output_size = BATCH_SIZE * features * ofp().itemsize
        self.d_input = cuda.mem_alloc(self.input_size)
        self.d_output = cuda.mem_alloc(self.output_size)
        self.bindings = [int(self.d_input), int(self.d_output)]
        self.outputs = numpy.empty([BATCH_SIZE, features], dtype=ofp)
        for binding in range(self.cuda_engine.num_io_tensors):
          tensor_name = self.cuda_engine.get_tensor_name(binding)
          self.context.set_tensor_address(tensor_name, self.bindings[binding])
        self.ctx.pop()
    def infer(self, data):
      self.ctx.push()
      try:
        preprocessed_input = PREPROCESSING(data)
        tensor = self.engine.get_tensor_name(0) # Get input tensor name
        self.context.set_input_shape(tensor,(data.shape[0],height,width,NUM_CHANNELS))
        cuda.memcpy_htod_async(self.d_input, preprocessed_input.astype(ifp), self.stream)
        self.context.execute_async_v3(self.stream.handle)
        cuda.memcpy_dtoh_async(self.outputs, self.d_output, self.stream)
        self.stream.synchronize()
        return POSTPROCESSING(self.outputs)
      finally:
        self.ctx.pop()
\end{lstlisting}

\begin{lstlisting}[language=Python, float=t, label=coud:cash, caption=Recolección de mensajes de las cachés de los sensores (ROS Noetic), basicstyle=\footnotesize]
def process(self):
# lidar_cash: MessageFiltersCache corresponding to LiDAR node.
# cam_cash: list of MessageFiltersCache objects, as many as camera nodes are launched.
    #Collect lidar message
    last_stamp = self.lidar_cash.getLatestTime()
    elapsed = rospy.Time.now().to_sec() - last_stamp.to_sec() if last_stamp else 1
    lidar = self.lidar_cash.getElemBeforeTime(last_stamp) if elapsed <= 0.5 else None
    nl = False if lidar else True
    message_list = [lidar] if not nl else []
    #Collect cameras messages
    pre = len(message_list)
    for cam in self.cam_cashes:
      camlast = cam.getLatestTime()
      report = cam.getElemBeforeTime(camlast)
      if report:
        camlapsed = rospy.Time.now().to_sec() - camlast.to_sec()
        if camlapsed <= 0.5:
          message_list.append(report)
    post = len(message_list)
    nc = True if pre == post else False
    if len(message_list) > 0:
      #Call integration node with collected messages
      self.on_frame(message_list, nl, nc)
    else:
      print("No new content in the queue")
\end{lstlisting}
\include{anexos/gestion}
\chapter{Métricas de evaluación de modelos de visión artificial}
\label{sec:modeleval}

\lettrine{E}{n} este apéndice se presentan los \textit{items} y métricas utilizadas para evaluar los modelos de detección y reconocimiento de este proyecto.

\section{Matriz de confusión}
\label{sec:metriks}

Para evaluar la clasificación de los modelos de visión artificial del proyecto, se empleó la famosa matriz de confusión, que es una tabla compuesta por los siguientes \textit{items}:
\begin{itemize}
    \item \textit{\textbf{True Positive (TP)}}: predicción positiva correcta. Para los modelos de detección, sería detectar correctamente a una persona, mientras que para los modelos de reconocimiento, sería asignar la etiqueta correcta a la persona detectada.
    \item \textit{\textbf{True Negative (TN)}}: predicción negativa correcta. Para los modelos de detección, sería no detectar correctamente a una persona, lo que carece de sentido a la hora de evaluar un buen detector. Para los modelos de reconocimiento, sería predecir que la clase (persona en este caso) no pertenece a ninguna de las registradas, es decir, predecir un \textbf{desconocido}. Este componente \textbf{no se llega a utilizar} para las pruebas de detección, por los motivos comentados, ni para las de reconocimiento, ya que siempre se asume la entidad desconocida como \textbf{incorrecta}.
    \item \textit{\textbf{False Positive (FP)}}: predicción positiva errónea. Para los modelos de detección, sería detectar una persona donde no la hay, mientras que para los modelos de reconocimiento, sería otorgar una etiqueta incorrecta a la persona detectada.
    \item \textit{\textbf{False Negative (FN)}}: predicción negativa errónea. Para los modelos de detección, sería no detectar a la persona presente, mientras que para los modelos de reconocimiento, sería asignar la entidad de desconocido a una persona ya registrada en el sistema.
\end{itemize}

Estos componentes se aplicarán a los \glspl{dataset} de prueba formados a partir de los videos grabados. En un entorno real de operación no sería posible aplicar dichas métricas por falta de un \textit{\gls{gt}}.

\section{Modelos de detección}
\label{subsec:det}

\begin{itemize}
    \item \textbf{\textit{Precision}}: \textbf{cuando el modelo detecta un objeto}, cuantas veces la detección corresponde con una persona presente.
          \begin{equation}
              Precision = \frac{TP}{TP + FP}
          \end{equation}
    \item \textbf{\textit{Recall}}: \textbf{cuando realmente hay una persona}, cuantas veces el modelo detecta a esa persona.
          \begin{equation}
              Recall = \frac{TP}{TP + FN}
          \end{equation}
    \item \textbf{\textit{F1\_score}}: se calcula como una media armónica entre el \textit{recall} (proporción de objetos detectados) y el \textit{precision} (proporción de detecciones correctas) \cite{andrew}.
          \begin{equation}
              F1\_score = 2* \frac{Precision * Recall}{Precision + Recall}
          \end{equation}

\end{itemize}

\section{Modelos de reconocimiento}
\label{subsec:recon}

Las métricas de evaluación se calculan igual que en los modelos de detección. A continuación se exponen las diferencias en el significado:
\begin{itemize}
    \item \textbf{\textit{Precision}}: \textbf{dentro de las detecciones realizadas por el respectivo modelo}, cuantas veces se ha reconocido correctamente al individuo.
    \item \textbf{\textit{Recall}}: \textbf{dentro del conjunto de reconocimientos correctos y desconocidos}, cuantas veces se ha reconocido al individuo correctamente. Dicho en otras palabras, el \textit{recall} mide la confianza del propio sistema a la hora de asignar identidades (el desconocido se asume como una indecisión del sistema de asignar una identidad en pruebas en las que únicamente aparecen personas conocidas).
    \item \textbf{\textit{F1\_score}}: lo mismo que en los modelos de detección.
\end{itemize}
\chapter{Procedimiento de optimización de redes neuronales}

\section{Procedimiento}

A continuación, se explican los pasos seguidos que permitieron obtener los resultados de la sección \ref{sec:modelic}. Los pasos se siguen como en \cite{tutos}.

\subsection{Conversión a ONNX}

Los modelos del proyecto se encuentran en diferentes formatos, existen integraciones para utilizar los modelos directamente de TensorFlow y de PyTorch. Sin embargo, se ha optado por usar exclusivamente el formato \acrshort{onnx}. Su principal desventaja frente a otros formatos son la incompatibilidad de ciertas capas de redes neuronales. En cambio, es la solución que otorga el mayor rendimiento. Según el formato se han aplicado las siguientes herramientas en la exportación:

\begin{itemize}
    \item \textbf{PyTorch}: la API de Python ya proporciona una herramienta (torch.onnx.export).
    \item \textbf{TensorFlow}: se usa la librería de Python \textbf{tf2onnx}.
\end{itemize}

\subsection{Conversión a TensorRT engine}
Los archivos \textit{engine} de TensorRT (extensión .trt o .engine) representan los modelos finales optimizados por dicha herramienta. El proceso de conversión se puede realizar a través de la API de Python o C++ o mediante el ejecutable \textit{trtexec}, TODO: que además devuelve resultados del \gls{profiling} de los modelos. Según la distribución del software,  el ejecutable \textit{trtexec} se encuentra en rutas distintas o ni siquiera se encuentra instalado, por lo que una forma más cómoda y general es usar la API.

\begin{lstlisting}[language=Python, float=t, label=coud:trtshapes, caption=Tensores de entrada y de salida del modelo ArcFace, basicstyle=\footnotesize]
=== Input and Output Info ===
TensorIOMode.INPUT - Name: input_image, Dtype: DataType.FLOAT, Shape: (-1, 112, 112, 3)
TensorIOMode.OUTPUT - Name: Identity, Dtype: DataType.FLOAT, Shape: (-1, 512)
\end{lstlisting}

En \ref{coud:trtshapes} se muestran las características de un modelo transformado en un \textit{engine} de TensorRT. La aparición de un -1 en la primera dimensión de los tensores indica que el modelo soporta \textit{\glspl{batch}} de tamaño dinámico.

\subsection{Códigos de inferencia}

El último paso es emplear los modelos obtenidos. La estructura de los códigos parte de los ya creados en \cite{andrew} para los modelos de este proyecto, salvo por el paso de transferir el trabajo a la \acrshort{gpu}. PyCUDA ofrece una API que facilita gran parte de la interacción con CUDA, aun así, se necesita gestionar en el código temas como la creación de contextos y reserva de la memoria.

TODO: comentar el código.

\subsection{Características de los modelos}
Hablar de los 4 modelos que usamos

\subsection{Compilación de OpenCV con soporte para CUDA}

En \cite{andrew} se utiliza una versión del modelo YuNet otorgada por la librería OpenCV, esta versión abstrae al programador de implementar las fases de pre y postprocesado de dicho modelo, que para este caso resultan muy complejas de implementar. Como ya se comentó en la sección \ref{sec:sw}, OpenCV dispone de soporte para CUDA, aunque es necesario compilar el programa desde el fuente. El proceso de compilación seguido es el mismo que en \cite{OpenCVCUDA}.

TODO: algo más.
\chapter{Implementación de la coherencia entre múltiples cámaras}
\label{chap:coherence}

Contar con más de una cámara permite realizar un mayor número de predicciones (amplía los grados de visión), cada una de ellas \textbf{más robustas} que con una sola cámara. Asumiendo ningún grado de solapamiento entre cámaras, puede afirmarse que un mismo individuo \textbf{no puede aparecer en el rango de más de una cámara} en un mismo instante de tiempo, es decir, existe una \textbf{coherencia espacio-temporal}. En base a esta afirmación, pueden detectarse errores de una misma predicción en múltiples cámaras e incluso adoptar ciertas técnicas para solucionarlas.

Cuando una incoherencia se detecta, o una o ninguna de las cámaras que han devuelto la predicción se corresponde realmente con la entidad, por lo que se procede a aplicar el siguiente método para averiguarlo:
\begin{itemize}
    \item De la lista de caras extraída de cada cámara, se escoge el frame más representativo, es decir, el de menor score.
    \item Del frame obtenido de cada cámara, se aplica la \textbf{distancia coseno} respecto a los frames de referencia de la entidad repetida y se fusionan todas las distancias en una mediana.
    \item Por cada valor de distancia, se comprueba si está por debajo de un umbral prefijado, en caso de que solo una cámara dé afirmativo, se asigna la entidad a dicha cámara, en caso negativo, se procede con el siguiente paso.
    \item Cada cámara posee un \textit{\gls{buffer}} que contiene las predicciones realizadas en la anterior secuencia. Se comprueba si la entidad actual se encuentra en dicho \textit{\gls{buffer}}, si solo una de las cámaras devuelve positivo, entonces se asigna la entidad a dicha cámara, en caso contrario, el método termina sin haber llegado a un acuerdo y se reconoce la entidad como desconocida para todas las cámaras.
\end{itemize}

La figura TODO muestra un caso real de aplicación,\dots

TODO: no es mas conveniente ejecutarlo antes del update module?

TODO: las secuencias de pocos frames (ej: 2) son muy propensas a resultados erroneos, especialmente si introducen mucho ruido, como frames borrosos, con mucha oclusión, variación en la iluminación, entre otros muchos factores. Debido a que este sistema maneja secuencias solapadas (ej: frames 0 a 10, frames 5 a 15) se puede deducir que es muy probable que la identidad a reconocer en la secuencia actual sea la misma identidad que en la secuencia anterior. Dependiendo de diversos factores, como el número de frames de la secuencia, se puede otorgar un mayor peso al reconocimiento anterior.

Debido a que este módulo no supone ninguna mejora en el rendimiento, en cierto modo debido a los casos escasos de conflictos y, por otra parte, al rendimiento pobre de la clasificación mediante la distancia coseno, se optó por no incluirse en la sección de pruebas.
\chapter{Pequeñas optimizaciones al código del sistema}
\label{chap:optimus}

\lettrine{E}{n} este apéndice se exponen algunas optimizaciones realizadas al código del sistema en términos de reserva de memoria y rendimiento.

\section{Optimizaciones en la reserva de memoria}

Debido a las limitaciones de memoria de la Jetson Orin Nano durante la ejecución del sistema (sección \ref{sec:syscal}), se ha optado por modificar el código a modo de hacerlo más eficiente en términos de reserva de memoria. Los cambios realizados son los siguientes:
\begin{itemize}
    \item Sustituir \textit{imports} de librerías "pesadas" (ejemplo: OpenCV o ultralytics, en torno a cientos de \acrshort{mb}) por otras más ligeras (ejemplo: numpy o scipy, en torno a decenas de \acrshort{mb}) con funciones alternativas, sin comprometer en exceso la precisión y latencia de las inferencias (ejemplo: sustituir cv2.copyMakeBorder por np.pad, o cv2.imread por PIL.Image).
    \item Uso de \textit{slots}: la variable \textit{\textunderscore \textunderscore slots\_\_} evita la creación del diccionario dinámico de Python (\textit{\_\_dict\_\_}), que almacena las variables de una \textbf{clase}. Los \textit{\_\_slots\_\_} son más rápidos y eficientes que los diccionarios dinámicos de Python, aunque también más extrictos (no se pueden crear más atributos que los especificados en \textit{\_\_slots\_\_}). Un ejemplo de uso se muestra en el código \ref{coud:slots}.
\end{itemize}

\begin{lstlisting}[language=Python, float=t, label=coud:slots, caption=Uso de \emph{\_\_slots\_\_} para evitar la generación de diccionarios dinámicos]
class CameraNode(parent):
__slots__ = ["face_detector", "face_recognizer", "person_detector", "person_recognizer", "camera", "identity", "publisher", "image_pub"]

    def __init__(self, camera):
        self.face_detector = importer.load_model(config.FACE_DETECTION_MODEL)
        self.face_recognizer = importer.load_model(config.FACE_RECOGNITION_MODEL)
        self.person_detector = importer.load_model(config.PERSON_DETECTION_MODEL)
        self.person_recognizer = importer.load_model(config.PERSON_RECOGNITION_MODEL)
        ...
\end{lstlisting}

El modelo ArcFace utilizado pesa \textbf{79 \acrshort{mb}} en su versión optimizada para TensorRT (en el formato \acrshort{onnx}, pesa \textbf{152 \acrshort{mb}}), 10 veces más que cualquiera de los otros modelos empleados (el tamaño es en torno a 8 \acrshort{mb}). Utilizando un modelo ArcFace más pequeño podría reducir en torno a cientos de \acrshort{mb} la ejecución del sistema.

\section{Modificaciones en el pre y postprocesado de modelos}
\label{chap:prepost}

Los pipelines de pre y postprocesado de los modelos suponen uno de los principales retos de los dispositivos Jetson, debido a limitada capacidad de procesamiento de las \acrshort{cpu} ARM, en este capítulo se exponen las soluciones aplicadas para mitigar el impacto computacional de dichos procesos en los modelos YOLO y OSNet.

\subsection{Postprocesado YOLO}
\label{sec:postyolo}

\subsection{Preprocesado OSNet}
\label{sec:preosnet}

Gracias a que se disponía del código de PyTorch del modelo, se ha podido integrar el pipeline de preprocesado de OSNet. El preprocesado de OSNet lleva a cabo las siguientes operaciones aplicadas a la imagen de entrada:
\begin{itemize}
    \item Intercambiar el orden de los canales de color (\acrshort{bgr} -> \acrshort{rgb}), ya que OpenCV trabaja con imágenes en \acrshort{bgr} y TensorRT con imágenes en \acrshort{rgb}.
    \item Reescalar la imagen a la resolución de entrada del modelo.
    \item Normalizar la imagen (dividir sus valores por 255, restar la media y dividir el resultado por la desviación típica).
    \item Intercambiar el orden de las dimensiones de la imagen: HWC (Height, Width, Channels) -> CHW, que es el modo que recomienda TensorRT para presentar los datos.
    \item Expandir una dimensión: CHW -> NCHW (donde N es el tamaño del \gls{batch}).
\end{itemize}

\begin{lstlisting}[language=Python, float=t, label=coud:preosnet, caption=Preprocesado de OSNet integrado en su estructura, basicstyle=\footnotesize]
pixel_mean = [0.485, 0.456, 0.406]
pixel_std = [0.229, 0.224, 0.225]
mean = torch.tensor(pixel_mean).view(1, 3, 1, 1)
std = torch.tensor(pixel_std).view(1, 3, 1, 1)

def forward(self, x, return_featuremaps=False):
    x = x.permute(2, 0, 1) # Swap dimension order
    x = x.unsqueeze(0) # Expand dims
    x = x[:,[2, 1, 0], :, :]  # BGR to RGB
    x = x / 255.0 # Normalization
    x = (x - mean) / std # Normalization
    x = self.featuremaps(x)
    ...

\end{lstlisting}

El código \ref{coud:preosnet} muestra un extracto de la función \textit{forward} del modelo en el que se muestran las operaciones del preprocesado. Se han tenido que intercambiar funciones de OpenCV por alternativas integradas de los arrays de NumPy (ejemplo: cvtColor por permute), el reescalado de la imagen se ha dejado en el código de la inferencia.

Finalmente, se crea la instancia del modelo en PyTorch y se exporta mediante la función \textbf{torch.onnx.export}.

Es \textbf{muy importante} nombrar de forma única los \glspl{tensor} que se declaren, ya que especialmente en las últimas versiones de \gls{onnx} esto puede causar errores de \textbf{grafos acíclicos} en los modelos exportados.
\chapter{Datasets del sistema}
\label{chap:datsets}

\lettrine{E}{n} este apéndice se exponen los elementos que conforman los \glspl{dataset} de las pruebas de los capítulos \ref{chap:cnn}, \ref{chap:syscal} y \ref{chap:systesting}.

Con el fin de evaluar el rendimiento del sistema en términos de precisión, se dispone principalmente de 2 \glspl{dataset} en formato \acrshort{json}, uno que contiene el \gls{gt} para evaluar el nodo cámara y el otro para evaluar el nodo integración de sensores. A continuación se explican los detalles de ambos.

\section{Nodo cámara}
\label{sec:datcam}

Este \gls{dataset} contiene el \gls{gt} que permite evaluar la salida de los nodos cámara y que se compone de los siguientes elementos \cite{andrew}:
\begin{itemize}
    \item \textit{\textbf{frames}}: lista de todos los frames capturados por la cámara. Cada entrada en la lista sigue la siguiente estructura:
          \begin{itemize}
              \item \textit{\textbf{id}}: identificador del frame.
              \item \textit{\textbf{entities}}: lista de las personas presentes, que contiene los siguientes campos:
                    \begin{itemize}
                        \item \textbf{bbox}: \gls{bbox} corporal del sujeto, compuesta por los siguientes elementos:
                              \begin{itemize}
                                  \item \textit{\textbf{x0, x1, y0 e y1}}: Coordenadas de la \gls{bbox}.
                                  \item \textit{\textbf{conf}}: confianza de la detección.
                                  \item \textit{\textbf{height}}: altura de la \gls{bbox}.
                                  \item \textit{\textbf{width}}: anchura de la \gls{bbox}.
                                  \item \textit{\textbf{difficulty}}: valoración de la dificultad de detección (easy, medium o hard), las dificultades se asignan de acuerdo con un sistema de puntos desarrollado en \cite{andrew} que tiene en cuenta diversos factores (ejemplo: porcentaje de oclusión o borrosidad), de esta forma, se organizan los resultados a partir de este criterio. Por sencillez, los resultados reflejados son siempre globales (es decir, la media de todas las dificultades).
                              \end{itemize}
                        \item \textit{\textbf{face\_bbox}}: \gls{bbox} facial del sujeto, compuesta por los mismos elementos que el campo \textbf{bbox}, agregando dos campos más, uno con las coordenadas de los puntos faciales (\textit{landmarks}) y el otro con la orientación de la cara (si está frontal, girada a la izquierda o a la derecha), calculada a partir de los \textit{landmarks}.
                        \item \textit{\textbf{filename}}: nombre de los archivos con los recortes facial y corporal del individuo.
                        \item \textit{\textbf{person\_id}}: etiqueta asignada al individuo.
                    \end{itemize}
          \end{itemize}
    \item \textit{\textbf{res}}: resolución de la cámara.
    \item \textit{\textbf{camera}}: identificador de la cámara.
    \item \textit{\textbf{face\_padding}}: valor de relleno para aumentar el tamaño de la \gls{bbox} facial.
    \item \textit{\textbf{input}}: ruta al video generado por la cámara.
\end{itemize}

\section{Nodo integración de sensores}
\label{sec:datsys}

Este \gls{dataset} contiene la información del \gls{gt} que permite evaluar la salida del nodo integrador y que se compone de los siguientes elementos \cite{andrew}:
\begin{itemize}
    \item \textbf{\textit{frames}}: lista con todos los frames del \gls{dataset}, cada frame contiene los siguientes elementos:
          \begin{itemize}
              \item \textbf{\textit{secs}}: \gls{timestamp} del tiempo de UNIX en segundos.
              \item \textbf{\textit{nsecs}}: \gls{timestamp} del tiempo en nanosegundos del segundo correspondiente al campo secs.
              \item \textbf{\textit{people}}: lista de objetos, cada uno compuesto por los siguientes elementos:
                    \begin{itemize}
                        \item \textbf{\textit{person\_id}}: etiqueta que identifica al usuario detectado (ejemplo: carlos).
                        \item \textbf{\textit{position}}: coordenadas del espacio \acrshort{3d} compartido por las cámaras y el \acrshort{lidar} en el que se sitúa dicho usuario.
                    \end{itemize}
          \end{itemize}
\end{itemize}

%\printglossary[title={Glossary}, type=main]
%\printglossary[title={Acronyms}, type=\acronymtype]

%\bibliographystyle{IEEEtranN}
%\bibliography{bibliografia/bibliografia}
%\cleardoublepage

\printglossary[type=\acronymtype,title=Relación de Acrónimos]
\printglossary[title=Glosario]

\bibliographystyle{IEEEtranN}
\bibliography{bibliografia/bibliografia}
\

\end{document}

%%%%%%%%%%%%%%%%%%%%%%%%%%%%%%%%%%%%%%%%%%%%%%%%%%%%%%%%%%%%%%%%%%%%%%%%%%%%%%%%
