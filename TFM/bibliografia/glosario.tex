%\chapter{Glosario de términos}
%\label{chap:glosario-terminos}

%\begin{description}
% \item [Aceleración] Ratio que relaciona el tiempo de ejecución de un proceso antes y después de una optimización aplicada sobre dicho proceso. 
%\end{description}
\newglossaryentry{dataset}
{
    name=dataset,
    description={Conjunto de datos normalmente extenso, que contiene información veraz utilizada para entrenar, validar o evaluar modelos de inteligencia artificial.}
}
\newglossaryentry{gt}
{
    name=ground truth,
    description={Información veraz que se proporciona a un modelo para evaluar su precisión. En visión artificial, puede referirse a una imagen etiquetada con la información del objeto que se muestra.}
}
\newglossaryentry{embsystem}
{
    name=sistema embebido,
    description={Circuito integrado capaz de realizar tareas computacionales. Puede llegar a incluir desde un simple microcontrolador hasta una arquitectura multiprocesador con dispositivos periféricos y red \cite{embsys}.}
}
\newglossaryentry{bbox}
{
    name=bounding box,
    description={Área que encierra a un objeto identificado por un modelo de detección y que otorga el tamaño y la posición de dicho objeto. Sirve para evaluar la precisión de modelos de detección, implementar algoritmos de seguimiento, entre otras funciones.}
}

\newglossaryentry{pdf}
{
    name=Probability Density Function,
    description={TODO.}
}

\newglossaryentry{wrapper}
{
    name=wrapper,
    description={Interfaz que invoca al código que realiza las computaciones.}
}

\newglossaryentry{profiling}
{
    name=profiling,
    description={Fase del software en el que se analiza el rendimiento de un programa durante su ejecución. Los \textit{profilers} son las herramientas diseñadas para este fin.}
}
\newglossaryentry{batch}
{
    name=batch,
    description={Conjunto de datos que suponen la entrada de la inferencia en un modelo de inteligencia artificial.}
}
\newglossaryentry{pcl}
{
    name=nube de puntos,
    description={Mapa que alberga puntos tridimensionales que definen la forma o superficie de un objeto en el entorno. Los puntos contienen la información de las coordenadas en el formato (x, y, z), adicionalmente, puede contener información como el color o la intensidad.}
}
\newglossaryentry{timestamp}
{
    name=timestamp,
    description={Marca temporal que puede asociarse a un mensaje para indicar el instante en el que se generó o para cualquier otro evento.}
}
