%\chapter{Glosario de términos}
%\label{chap:glosario-terminos}

%\begin{description}
% \item [Aceleración] Ratio que relaciona el tiempo de ejecución de un proceso antes y después de una optimización aplicada sobre dicho proceso. 
%\end{description}
\newglossaryentry{dataset}
{
    name=dataset,
    description={Conjunto de datos normalmente extenso, que contiene información veraz utilizada para entrenar, validar o evaluar modelos de inteligencia artificial.}
}
\newglossaryentry{gt}
{
    name=ground truth,
    description={Información veraz que se proporciona a un modelo para evaluar su precisión. En visión artificial, puede referirse a una imagen etiquetada con la información del objeto que se muestra.}
}
\newglossaryentry{embsystem}
{
    name=sistema embebido,
    plural={sistemas embebidos},
    description={Circuito integrado capaz de realizar tareas computacionales. Puede llegar a incluir desde un simple microcontrolador hasta una arquitectura multiprocesador con dispositivos periféricos y red \cite{embsys}.}
}
\newglossaryentry{bbox}
{
    name=bounding box,
    plural={bounding boxes},
    description={Área que encierra a un objeto identificado por un modelo de detección y que otorga el tamaño y la posición de dicho objeto. Sirve para evaluar la precisión de modelos de detección, implementar algoritmos de seguimiento, entre otras funciones.}
}

\newglossaryentry{wrapper}
{
    name=wrapper,
    description={Interfaz que invoca al código que realiza las computaciones.}
}

\newglossaryentry{profiling}
{
    name=profiling,
    description={Fase del software en el que se analiza el rendimiento de un programa durante su ejecución. Los \textit{profilers} son las herramientas diseñadas para este fin.}
}
\newglossaryentry{batch}
{
    name=batch,
    plural={batches},
    description={Conjunto de datos que suponen la entrada de la inferencia en un modelo de inteligencia artificial.}
}
\newglossaryentry{pcl}
{
    name=nube de puntos,
    plural={nubes de puntos},
    description={Mapa que alberga puntos tridimensionales que definen la forma o superficie de un objeto en el entorno. Los puntos contienen la información de las coordenadas en el formato (x, y, z), adicionalmente, puede contener información como el color o la intensidad.}
}
\newglossaryentry{timestamp}
{
    name=timestamp,
    description={Marca temporal que puede asociarse a un mensaje para indicar el instante en el que se generó o para cualquier otro evento.}
}
\newglossaryentry{buffer}
{
    name=buffer,
    description={Estructura de datos en la que se almacena información de forma temporal.}
}
\newglossaryentry{embedding}
{
    name=embedding,
    description={Representación matemática de un objeto (ejemplo: un texto o una imagen) generada y/o consumida por un modelo en un espacio vectorial, de forma que objetos parecidos se sitúan a poca distancia entre sí y viceversa.}
}
\newglossaryentry{tensor}
{
    name=tensor,
    plural={tensores},
    description={Vector de n dimensiones que se utiliza para almacenar objetos generados y/o consumidos por los modelos de deep learning (ejemplo: datos de entrada) y que se puede ejecutar en una \acrshort{gpu}.}
}
\newglossaryentry{thread}
{
    name=thread,
    description={Un thread (o hilo/sub-proceso) es una instancia de un proceso que se programa mediante el sistema operativo para ejecutar una serie de instrucciones en una \acrshort{cpu} o \acrshort{gpu}.}
}
\newglossaryentry{lossfunc}
{
    name=loss function,
    description={Función que se ejecuta durante la inferencia de un modelo de inteligencia artificial, que rastrea el grado de pérdida de precisión del modelo entre una predicción y su valor real (\gls{gt}). Los modelos "aprenden" a realizar mejores predicciones ajustando sus parámetros, de tal forma que se reduzca el valor de esta función \cite{loss}.}
}