\chapter{Fundamentos teóricos}
\lettrine{E}{n} este capítulo se exponen los conceptos necesarios para comprender la memoria.

\section{Tipos de aprendizaje máquina semisupervisado}
El aprendizaje máquina semisupervisado parte de una fase inicial de los modelos entrenados con datos perfectamente \textbf{etiquetados}, dichos datos se complementan o se actualizan con muestras obtenidas durante la operación del sistema (no etiquetadas), de esta forma el modelo amplia su conocimiento de manera autónoma. Existen 3 tipos diferentes de aprendizaje máquina semisupervisado que se aplicarán a lo largo de esta memoria y que son los siguientes:

\subsection{\textit{Closed-Set}}
Tipo de aprendizaje que opera en un conjunto \textbf{cerrado} de objetos, siendo en este caso personas registradas de antemano. Es el modo en el que trabaja el sistema base \cite{andrew}.

\subsection{\textit{Open-Set}}
Extiende al modo \textit{Closed-Set}, de forma que puede reconocer a usuarios que no pertenezcan al conjunto de entrenamiento, es decir, a usuarios \textbf{desconocidos}.

\subsection{\textit{Open-World}}

Atendiendo a la definición de \cite{rudd2017extreme}, un sistema de reconocimiento \textit{Open-World} debe de ser capaz de realizar las siguientes 4 tareas:
\begin{itemize}
    \item Detectar desconocidos (\textbf{\textit{Open-Set}}): identificar cuando una muestra de entrada no pertenece al conjunto de datos del entrenamiento.
    \item Escoger las muestras que puedan aportar información del desconocido al modelo.
    \item Etiquetar dichas muestras, por ejemplo, con un número o un pseudónimo.
    \item Actualizar el modelo.
\end{itemize}

\section{Clasificadores}

Nearest neighbourd

Random Forest: Number of trees, maximum depth, minimum samples per leaf

Neural Networks: Learning rate, batch size, network architecture, regularization strength

Support Vector Machines: Kernel type, regularization parameter, kernel-specific parameters

Gradient Boosting: Learning rate, number of boosting rounds, maximum tree depth


\subsection{Support Vector Machine (SVM)}

\begin{figure}[tbp]
    \centering
    \includegraphics[width=0.5\linewidth]{imagenes/LINSVM.png}
    \caption{Clasificación en espacio \acrshort{2d} mediante el uso de las \acrshort{svm}, figura extraída de \cite{svn}.}
    \label{fig:linsvm}
\end{figure}

Es un algoritmo de \textbf{clasificación} que dibuja un hiperplano entre categorías de datos, buscando siempre el mayor margen (distancia entre los puntos que definen las fronteras de las categorías, denominados \textit{support vectors}), de modo que agrega cierta tolerancia al posible ruido producido. En la figura \ref{fig:linsvm} se muestra un ejemplo de clasificación en el espacio \acrshort{2d}, los cuadrados grises representan los \textit{support vectors}, que definen el margen máximo de separación entre clases.

Las \acrshort{svm} requieren de un conjunto inicial de datos para su entrenamiento (aprendizaje supervisado) y no son capaces de discernir entre clases fuera del conjunto de datos de entrenamiento (\textit{Closed-Set}), por lo que un dato desconocido se clasificaría erróneamente como una de las clases del entrenamiento \cite{rudd2017extreme}.

Estos clasificadores conforman la unidad más básica de clasificación dentro del sistema de este proyecto.

\subsection{Comités}

Se ha demostrado que múltiples \acrshort{svm} simples (ejemplo: lineales) como conjunto \textbf{generalizan mejor} que una única \acrshort{svm} compleja (ejemplo: sigmoide) \cite{malisiewicz2011ensemble}. Los comités de \acrshort{svm} también favorecen la adaptación a los cambios, ya que permiten agregar conocimiento acerca de una entidad entrenando una nueva \acrshort{svm} lineal o eliminar dicho conocimiento descartando la \acrshort{svm} correspondiente. De esta forma, se obtiene una representación de un individuo a partir de un conjunto de clasificadores ligeros, sin necesidad de reentrenar un clasificadores complejo desde cero \cite{malisiewicz2011ensemble}.

TODO: SVM lineales: que representan la distancia del punto respecto al plano lineal que distingue entre la clase del propio individuo y el resto del universo.