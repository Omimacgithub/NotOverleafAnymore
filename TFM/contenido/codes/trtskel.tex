\begin{lstlisting}[language=Python, float=t, label=coud:trtskel, caption=Esqueleto del código de las inferencias con TensorRT]
import numpy as np
import tensorrt as trt
import pycuda.driver as cuda

# BATCH_SIZE: number of images to be processed in a single inference
# NUM_CHANNELS: number of image channels, typically 3 (red, green and blue)
# size: input tensor size of the model
# osize: output tensor size of the model
# ifp: input tensor precision of the model
# ofp: output tensor precision of the model
# isversion10: flag that indicates TensorRT API version 10 when True, else version 8 
class CNN_model:
    def __init__(self, cuda_ctx):
        super().__init__()
        self.ctx = cuda_ctx
        self.ctx.push()
        f = open(str('path_to_trt_engine'), "rb")
        runtime = trt.Runtime(trt.Logger(trt.Logger.WARNING)) 
        self.engine = runtime.deserialize_cuda_engine(f.read())
        self.context = self.engine.create_execution_context()
        # Pre-allocate GPU memory using memory pool
        self.stream = cuda.Stream()
        self.input_size = BATCH_SIZE * NUM_CHANNELS * size * size * ifp().itemsize
        self.output_size = BATCH_SIZE * features * osize * ofp().itemsize
        self.d_input = cuda.mem_alloc(self.input_size)
        self.d_output = cuda.mem_alloc(self.output_size)
        self.bindings = [int(self.d_input), int(self.d_output)]

        if isversion10:
          for binding in range(self.engine.num_io_tensors):
            tensor_name = self.engine.get_tensor_name(binding)
            self.context.set_tensor_address(tensor_name, self.bindings[binding])
        self.ctx.pop()

    def infer(self, frame):
      self.ctx.push()
      try:
        outputs = np.empty([BATCH_SIZE, osize], dtype=ofp)
        preprocesed_frame = PREPROCESING(frame)
        # Transfer input data to device
        cuda.memcpy_htod_async(self.d_input, preprocesed_frame.astype(ifp), self.stream)
        if isversion10:
          self.context.execute_async_v3(self.stream.handle)
        else:
          self.context.execute_async_v2(self.bindings, self.stream.handle, None)
        # Transfer predictions back
        cuda.memcpy_dtoh_async(outputs, self.d_output, self.stream)
        self.stream.synchronize()
        return POSTPROCESING(outputs)
      finally:
        self.ctx.pop()
\end{lstlisting}