\chapter{Implementación del sistema}
\label{chap:impl}

\lettrine{E}{n} este capítulo se ahonda en los fundamentos y detalles de implementación de los nuevos componentes.

\section{Creación de las SVM}
\label{sec:training}

El código \ref{coud:training} corresponde al proceso seguido a la hora de crear y entrenar nuevos clasificadores. De todas las muestras de individuos de la base de datos, se \textbf{excluyen las del propio sujeto} en el caso de crear una \acrshort{svm} para un comité existente. Posteriormente, se extrae un subconjunto aleatorio de estas (función random\_pick), que representará al conjunto de muestras negativas del entrenamiento. Finalmente, se crea la \acrshort{svm} y se entrena con el conjunto de positivos (variable d1) y negativos, etiquetados con 1 (variable labels) y -1 (variable plabels) respectivamente. De esta forma, cuando el clasificador determine que las muestras pertenecen a la clase que representa, devolverá como máximo una puntuación de -1, en caso contrario devolverá un 1 TODO: no sé porqué se comporta al revés. Este comportamiento se ha aplicado simplemente por comodidad, pero la \acrshort{svm} podría entrenarse para que cause el efecto contrario.

%, basicstyle=\footnotesize]
\begin{lstlisting}[language=Python, float=t, label=coud:training, caption={Entrenamiento de las SVM, sección \ref{sec:training}}]     
# sample = set of embeddings from CNNs, for ArcFace and OSNet_x1, each sample is composed of 512 embeddings
def def_svm(embeddings, idx, d1, unk):
# embeddings shape: (num_entities, num_svm, num_samples, embeddings)
# idx: index that points to the entity ensemble in the database
# d1: positive samples 
# unk: flag to mark if d1 samples belong to an unknown entity
# nneg: number of samples that make up the set of negatives.
    neg_embeddings = deepcopy(embeddings)
    if not unk:
        neg_embeddings.pop(idx)
    neg_samples = [neg_samples for num_entities in neg_embeddings for num_svm in num_entities for num_samples in num_svm]
    k = nneg if len(neg_samples) > nneg else len(neg_samples)
    neg_samples = random_pick(np.array(neg_samples), size=k) # Negative selection
    trainData = np.vstack([neg_samples, d1], dtype=np.float32)
    nlabels = -np.ones([neg_samples.shape[0], 1], dtype=np.int32)
    plabels = np.ones([len(d1), 1], dtype=np.int32)
    labels = np.vstack([nlabels, plabels])
    svm = cv2.ml.SVM_create()
    svm.setType(cv2.ml.SVM_C_SVC)
    svm.setKernel(cv2.ml.SVM_LINEAR)
    svm.setClassWeights(np.array([1,100], dtype = np.float32))
    svm.setTermCriteria((cv2.TERM_CRITERIA_COUNT, 1000, 1.e-06))
    svm.train(trainData.astype(np.float32), cv2.ml.ROW_SAMPLE, labels.astype(np.int32))
    return svm
\end{lstlisting}

\section{Módulo de reconocimiento}

Esta función determina si la entidad detectada se corresponde a un individuo previamente reclutado o a una entidad \textbf{desconocida}.

El código \ref{coud:weib} muestra la implementación de la función RDF. Del conjunto de scores ordenados, se excluye el primer score de la distribución (que corresponde con el más bajo, por lo tanto el mejor) y se compone la distribución de distancias de cada puntuación respecto a la mediana. De los puntos de la distribución, se obtienen los parámetros \textit{shape} y \textit{scale}, que modelan la función de Weibull. Finalmente, se calcula la probabilidad de pertenencia a la distribución del mejor score (función weib) y se toma la decisión en base a un umbral (Tw). Si la probabilidad es inferior al umbral, se reconoce el caso como un extremo y se asigna la identidad correspondiente al sujeto (\textit{drift}), en caso contrario se devuelve como desconocido (\textit{unknown}).

\begin{lstlisting}[language=Python, float=t, label=coud:weib, caption=Módulo de reconocimiento, basicstyle=\footnotesize]
import numpy as np
from scipy import stats

def weib(x,scale,shape):
    if x < 0:
      return 0
    return (shape / scale) * (x / scale) ** (shape - 1) * np.exp(- (x / scale) ** shape)

def RDF(R, sorted_ens_scr, Tw):
  #R: original set of ensemble scores
  #sorted_ens_scr: sorted set of ensemble scores
  #Tw: Weibull threshold
  c = sorted_ens_scr[0]
  set_of_mins = sorted_ens_scr[1:int(len(sorted_ens_scr))]
  m = np.median(sorted_ens_scr[1:])
  v = np.abs(set_of_mins - m) #Non-match distribution
  cand_scr = np.abs(c - m)  #Best score
  #Obtain shape and scale Weibull parameters
  shape,loc,scale = stats.weibull_min.fit(v, floc=0)  
  pdf = weib(cand_scr, scale, shape)  #Calculate PDF
  # Decision
  if (pdf<Tw):
    decision = np.where( R == c )[0][0] # Drift
  else:
    decision = -1 # Unknown
  return decision
\end{lstlisting}

\section{Escalabilidad y tolerancia a fallos de las cámaras}
Cuando se fusionan los resultados procesados por los distintos sensores (nodo integración de sensores en la figura \ref{fig:finalsys}), se requiere que todos ellos se encuentren \textbf{sincronizados} dentro de un intervalo temporal (ejemplo: 100 milisegundos), de forma que las predicciones no se realizan a partir de información desactualizada. En \cite{andrew} se utiliza una clase del paquete \textit{message\_filters} de \acrshort{ros} llamado \textit{ApproximateTimeSynchronizer}. Esta clase utiliza un algoritmo para emparejar mensajes a partir de su \gls{timestamp} \cite{ApproximateTime}. El problema de \textit{ApproximateTimeSynchronizer} es que espera recibir datos \textbf{de todas las fuentes en todo momento}, lo que no es una situación realista y que puede provocar una caída del sistema en el momento que uno de los sensores falle. Como solución a este problema, se ha realizado una sustitución por la clase \textit{MessageFiltersCache} de la misma librería \cite{MFC}. En esta nueva implementación, cada nodo posee una caché en la que se almacenan sus mensajes, a una frecuencia establecida se recuperan los datos de todas las cachés, TODO: si en una de las cachés el último dato no se corresponde con el intervalo actual, este no se tiene en cuenta para dicho intervalo. El código \ref{coud:cash} muestra la implementación de dicho mecanismo.

\begin{lstlisting}[language=Python, float=t, label=coud:cash, caption=Recolección de mensajes de las cachés de los sensores (ROS Noetic), basicstyle=\footnotesize]
def process(self):
# lidar_cash: MessageFiltersCache corresponding to LiDAR node.
# cam_cash: list of MessageFiltersCache objects, as many as camera nodes are launched.
    #Collect lidar message
    last_stamp = self.lidar_cash.getLatestTime()
    elapsed = rospy.Time.now().to_sec() - last_stamp.to_sec() if last_stamp else 1
    lidar = self.lidar_cash.getElemBeforeTime(last_stamp) if elapsed <= 0.5 else None
    nl = False if lidar else True
    message_list = [lidar] if not nl else []
    #Collect cameras messages
    pre = len(message_list)
    for cam in self.cam_cashes:
      camlast = cam.getLatestTime()
      report = cam.getElemBeforeTime(camlast)
      if report:
        camlapsed = rospy.Time.now().to_sec() - camlast.to_sec()
        if camlapsed <= 0.5:
          message_list.append(report)
    post = len(message_list)
    nc = True if pre == post else False
    if len(message_list) > 0:
      #Call integration node with collected messages
      self.on_frame(message_list, nl, nc)
    else:
      print("No new content in the queue")
\end{lstlisting}

\section{Migración a ROS 2}
\label{subsec:ROS2}

Con el motivo del fin de soporte de ROS 1 \cite{ROSEOL}, se ha optado por migrar el sistema para ser ejecutado en ROS 2 con el fin de mantener su continuidad. El proceso de migración se ha llevado a cabo por medio de la guía oficial de \acrshort{ros} \cite{ros2tuto}. Se han migrado los nodos cámara e integrador, la migración del nodo \acrshort{lidar} requeriría actualizar el código a una versión soportada para Ubuntu 22.04 de la librería pcl, entre otros detalles que llevarían a rediseñar casi todo el código. Por último, el sensor \acrshort{lidar} instalado no es compatible con \acrshort{ros} 2, lo que impide la migración a menos que se haga un recambio.