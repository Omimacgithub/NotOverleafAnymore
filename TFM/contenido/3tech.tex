\chapter{Fundamentos tecnológicos}
\lettrine{E}{n} este capítulo se detallan las herramientas hardware y software empleadas para el desarrollo y la gestión del proyecto: dispositivos de altas prestaciones, hardware y software específicos y software genérico.

\section{Dispositivos de altas prestaciones}

Con el fin de conocer los dispositivos empleados en este proyecto, se comentan las arquitecturas de las \acrshort{gpu}s que incorporan y las características técnicas de cada una de ellos (familias Jetson e Intel).

\subsection{Arquitecturas de \acrshort{gpu}}
\label{sec:gpuses}

%Según la arquitectura de la \acrshort{gpu} de NVIDIA, se agregan nuevas características que maximizan el rendimiento de las tareas de inteligencia artificial. 
A continuación se detallan las características de las arquitecturas de \acrshort{gpu} que se tratarán en esta memoria, junto con sus \textbf{\textit{compute capabilities}}, que no son más que códigos que definen las características hardware e instrucciones soportadas. Las arquitecturas se encuentran ordenadas de menor a mayor generación.

\subsubsection{Volta}

En la arquitectura Volta (2017) \cite{volta} se introducen los \textbf{\emph{tensor cores}}, que son componentes destinados a acelerar multiplicaciones y sumas de matrices (también conocido como MAC), ampliamente utilizas para implementar capas de convoluciones en redes neuronales. Los \emph{tensor cores} también introducen el entrenamiento e inferencia en \textbf{precisión mixta en \acrshort{fp}32 y \acrshort{fp}16}, que consiste en escoger la precisión que menor latencia reporte entre una lista de precisiones (ejemplo: \acrshort{fp}32+\acrshort{fp}16) para cada capa de un modelo. El entrenamiento en precisión mixta puede otorgar hasta \textbf{3 veces mayor velocidad} que utilizar únicamente \acrshort{fp}32. Dicha arquitectura tiene el código 7.0 como \emph{compute capability} y la emplea la \textbf{Jetson Xavier NX}.

\subsubsection{Turing}
La arquitectura Turing (2018) \cite{demystifying} introduce las precisiones \textbf{INT8 e INT4}, a mayores de la precisión \acrshort{fp}16 de Volta. Turing tiene el código 7.5 como \emph{compute capability} y es empleada por la \textbf{GeForce GTX 1650} (gráfica del PC lab).

\subsubsection{Ampere}

\begin{figure}[tbp]
    \centering
    \includegraphics[width=0.5\linewidth]{imagenes/types.jpg}
    \caption[Asignación de bits según el tipo de dato]{Asignación de bits según el tipo de dato, figura extraída de \cite{tf32}}
    \label{fig:types}
\end{figure}

Con la arquitectura Ampere (2020) \cite{ampere} se incorporan los \emph{tensor cores} de tercera generación, que incluyen el formato de precisión \textbf{TF32} (Tensor Float 32). TF32 es un tipo optimizado para los tensor cores, híbrido entre \acrshort{fp}32 y \acrshort{fp}16, ya que de rango o exponente tiene los mismos bits que \acrshort{fp}32 y de mantisa o precisión tiene los mismos bits que \acrshort{fp}16 (como se muestra en la figura \ref{fig:types}). TF32 funciona igual que \acrshort{fp}32 (acepta entradas y salidas de dicho tipo) y puede otorgar velocidades hasta \textbf{20 veces} superiores que \acrshort{fp}32 sin realizar ningún cambio en el código. Por otro lado, se han incorporado nuevos tensor cores diseñados para operaciones con matrices \textbf{dispersas} (\emph{sparse tensor cores}), que son matrices con una gran proporción de ceros entre sus valores. Esta mejora puede aumentar hasta 2 veces la velocidad de la inferencia para modelos que exploten el uso de matrices dispersas.

Se introduce la funcionalidad \textbf{\acrfull{mig}}, que permite la asignación de particiones \textbf{físicas} de la \acrshort{gpu} a las aplicaciones de \gls{CUDA}, de forma que se ejecutan en unos recursos asignados, \textbf{sin interferencia} de otros procesos. En el apéndice \ref{chap:mig} se muestran algunos detalles de la funcionalidad en el caso de la Jetson AGX Thor.

La \textbf{Jetson Orin Nano}, \textbf{Jetson AGX Orin} y la \textbf{GeForce RTX 3050} (gráfica del portátil) siguen esta arquitectura, que posee desde el código 8.0 hasta el 8.7 como \emph{compute capability}.

\subsubsection{Blackwell}

Blackwell (2024) \cite{blackwell} es la última arquitectura diseñada para \acrshort{gpu}s de NVIDIA, en ella se incorporan los \emph{tensor cores} de quinta generación, que permiten operar en las precisiones \textbf{\acrshort{fp}4} y \textbf{\acrshort{fp}6} (punto flotante de 4 y 6 bits respectivamente).

Se lanza la \textbf{segunda generación del Transformer Engine} \cite{transformer}. Introducido en las arquitecturas Ada Lovelace y Hopper (posteriores a Ampere), el \textit{Transformer Engine} es una librería que permite manejar modelos en \textbf{precisiones muy reducidas} (a partir de \acrshort{fp}8 e INT8 para abajo) a través de múltiples técnicas que computan los \textbf{factores de escalado}, que permiten ajustar los valores de los \glspl{tensor} a un rango más compacto y representable por precisiones más pequeñas. En esta nueva versión para la Blackwell, se incorpora el \textbf{\textit{micro-tensor scaling}}, que aplica diferentes factores de escalado al nivel de los elementos de un \gls{tensor}. Esta nueva técnica hace posible la ejecución en \textbf{\acrshort{fp}4} y \textbf{\acrshort{fp}6} sin apenas pérdida en la precisión.

La \textbf{Jetson AGX Thor} sigue esta arquitectura, que posee desde el código 10.0 hasta el 12.1 como \emph{compute capability}.

\subsection{Familias Jetson e Intel}

Los \textbf{seis dispositivos} que se emplearon en este proyecto se han agrupado en \textbf{dos familias}. Por un lado, la \textbf{familia Jetson}, que engloba a los sistemas embebidos propietarios de NVIDIA con dicho nombre y, por el otro lado, la \textbf{familia Intel}, compuesta por un equipo de escritorio y un portátil, ambos con una \acrshort{cpu} Intel junto a una \acrshort{gpu} de NVIDIA.

% Please add the following required packages to your document preamble:
% \usepackage{multirow}
\begin{table}[]
    \centering
    \resizebox{\textwidth}{!}{\begin{tabular}{cc|cccc}
            \hline
            \multicolumn{2}{l|}{\multirow{2}{*}{}}             & \multicolumn{4}{c}{\textit{Jetson family}}                                                                                                                                                                                                                                                                                                                                                                                                             \\
            \multicolumn{2}{l|}{}                              & \textbf{\begin{tabular}[c]{@{}c@{}}Xavier \\ NX \cite{jxavier}\end{tabular}} & \textbf{\begin{tabular}[c]{@{}c@{}}Orin \\ Nano \cite{jorin}\end{tabular}}  & \textbf{\begin{tabular}[c]{@{}c@{}}AGX \\ Orin \cite{jorin}\end{tabular}}     & \textbf{\begin{tabular}[c]{@{}c@{}}AGX \\ Thor \cite{jthor}\end{tabular}}                                                                                                                                 \\ \hline
            \multirow{3}{*}{\textit{CPU}}                      & \textit{Cores}                                                               & \begin{tabular}[c]{@{}c@{}}6-core \\ NVIDIA Carmel \\ Armv8.2\end{tabular}  & \begin{tabular}[c]{@{}c@{}}6-core\\ Arm Cortex-A78AE \\ v8.2\end{tabular}     & \begin{tabular}[c]{@{}c@{}}12-core\\ Arm Cortex-A78AE \\ v8.2\end{tabular}                                 & \begin{tabular}[c]{@{}c@{}}14-core \\ Arm Neoverse-V3AE\end{tabular}                         \\ \cline{2-2}
                                                               & \textit{Max freq}                                                            & 1.9 GHz                                                                     & 1.7 GHz                                                                       & 2.2 GHz                                                                                                    & 2.6 GHz                                                                                      \\ \cline{2-2}
                                                               & \textit{Cache}                                                               & \begin{tabular}[c]{@{}c@{}}L2: 6 MB\\ L3: 4MB\end{tabular}                  & \begin{tabular}[c]{@{}c@{}}L2: 1.5MB\\ L3: 4MB\end{tabular}                   & \begin{tabular}[c]{@{}c@{}}L2: 3MB\\ L3: 6MB\end{tabular}                                                  & \begin{tabular}[c]{@{}c@{}}L2: 1MB per core\\ L3: 16MB\end{tabular}                          \\ \hline
            \multirow{4}{*}{\textit{GPU}}                      & \textit{Arch}                                                                & Volta                                                                       & Ampere                                                                        & Ampere                                                                                                     & Blackwell                                                                                    \\ \cline{2-2}
                                                               & \textit{Cores}                                                               & \begin{tabular}[c]{@{}c@{}}384 CUDA cores\\ 48 Tensor cores\end{tabular}    & \begin{tabular}[c]{@{}c@{}}1024 CUDA cores\\ 32 Tensor cores\end{tabular}     & \begin{tabular}[c]{@{}c@{}}2048 CUDA cores\\ 64 Tensor cores\end{tabular}                                  & \begin{tabular}[c]{@{}c@{}}2560 CUDA cores\\ 96 Tensor cores\\ MIG with 10 TPCs\end{tabular} \\ \cline{2-2}
                                                               & \textit{Max freq}                                                            & 1.1 GHz                                                                     & 1.02 GHz                                                                      & 1.3 GHz                                                                                                    & 1.57 GHz                                                                                     \\ \cline{2-2}
                                                               & \textit{\begin{tabular}[c]{@{}c@{}}\#SM \footnotemark[1]\end{tabular}}       & 6                                                                           & 8                                                                             & 16                                                                                                         & 20                                                                                           \\ \hline
            \multicolumn{2}{c|}{\textit{Memory}}               & \begin{tabular}[c]{@{}c@{}}128-bit\\ 8 GB LPDDR4x \\ (59.7GB/s)\end{tabular} & \begin{tabular}[c]{@{}c@{}}128-bit\\ 8 GB LPDDR5 \\ (102 GB/s)\end{tabular} & \begin{tabular}[c]{@{}c@{}}256-bit\\ 64 GB LPDDR5 \\ (204.8GB/s)\end{tabular} & \begin{tabular}[c]{@{}c@{}}256-bit\\ 128 GB LPDDR5X \\ (273 GB/s)\end{tabular}                                                                                                                            \\ \hline
            \multicolumn{2}{c|}{\textit{Network}}              & \begin{tabular}[c]{@{}c@{}}1 RJ45 \\ Gigabit Ethernet\end{tabular}           & \begin{tabular}[c]{@{}c@{}}1 RJ45 \\ Gigabit Ethernet\end{tabular}          & \begin{tabular}[c]{@{}c@{}}1 RJ45 \\ 10 Gigabit Ethernet\end{tabular}         & \begin{tabular}[c]{@{}c@{}}1 RJ45 \\ 5 Gigabit Ethernet\\ 1 QSFP28 \\ 2 x 25 Gigabit Ethernet\end{tabular}                                                                                                \\ \hline
            \multicolumn{2}{c|}{\textit{Power}}                & 10-20 W                                                                      & 7-25 W                                                                      & 15-60 W                                                                       & 40-130 W                                                                                                                                                                                                  \\ \hline
            \multicolumn{2}{c|}{\textit{AI performance}}       & 21 TOPS                                                                      & 67 TOPS                                                                     & 275 TOPS                                                                      & \begin{tabular}[c]{@{}c@{}}2070 TFLOPS \\ (FP4—Sparse)\end{tabular}                                                                                                                                       \\ \hline
            \multicolumn{2}{c|}{\textit{Price (Jan 20, 2026)}} & 399\$                                                                        & 249\$                                                                       & 1999\$                                                                        & 3499\$                                                                                                                                                                                                    \\ \hline
        \end{tabular}}
    \caption{Tabla de dispositivos Jetson del proyecto}
    \label{tab:jetsons}
\end{table}
\footnotetext[1]{Extraído de la información del sistema}

La \textbf{NVIDIA Jetson} \cite{jason} es una familia de \textbf{\glspl{embsystem}} pensados para aplicaciones de inteligencia artificial. Dichos dispositivos incluyen un kit de desarrollo llamado \textbf{JetPack}, diseñado para exprimir la potencia de la NVIDIA Jetson en aplicaciones como la robótica, la visión artificial, entre otros. Actualmente existe una amplia variedad en la potencia y precio de estos dispositivos \cite{jason}. Para este proyecto, se han empleado cuatro modelos diferentes que se exponen en la tabla \ref{tab:jetsons}.

Mientras que las \textbf{Jetson Xavier NX, Orin Nano y AGX Orin} se centran en la ejecución de \textbf{aplicaciones de inteligencia artificial en tiempo real} como la que aplica a este proyecto, la \textbf{Jetson AGX Thor} busca abarcar el ámbito de la \textbf{robótica avanzada} o \textbf{inteligencia artificial física}. La cantidad y la potencia de los núcleos de \acrshort{cpu}, junto a su amplia memoria y la funcionalidad \acrshort{mig} permite la ejecución de \textbf{múltiples procesos complejos} a un \textbf{ratio estable} (ejemplo: ejecución de un modelo de detección a 30 \acrshort{fps} constantes), posible en gran medida al particionado físico de la \acrshort{gpu} (\acrshort{mig}) \cite{physAI}.

% Please add the following required packages to your document preamble:
% \usepackage{multirow}
% Please add the following required packages to your document preamble:
% \usepackage{multirow}
\begin{table}[]
    \resizebox{\textwidth}{!}{\begin{tabular}{cc|ccc}
            \hline
            \multicolumn{2}{c|}{\multirow{2}{*}{}} & \multicolumn{3}{c}{\textit{Desktop}}                                                                                                                                                                                                                                                                                                                                                                                        \\
            \multicolumn{2}{c|}{}                  & \textbf{PC Lab}                      & \textbf{Laptop \textbackslash{}cite\{asuspender\}}                                                                                  & \textbf{Reference \textbackslash{}cite\{andrew\}}                                                                                                                                                                                              \\ \hline
            \multirow{3}{*}{\textit{CPU}}          & \textit{Cores}                       & \begin{tabular}[c]{@{}c@{}}8 performance-cores\\ 4 efficient-cores\\ Intel Core i7-12700 \textbackslash{}cite\{12700\}\end{tabular} & \begin{tabular}[c]{@{}c@{}}6 performance-cores\\ 4 efficient-cores\\ Intel Core i7-12650H \textbackslash{}cite\{12650H\}\end{tabular} & \begin{tabular}[c]{@{}c@{}}4 cores \\ Intel Core i5-2500K\\ \textbackslash{}cite\{2500K\}\end{tabular} \\ \cline{2-2}
                                                   & \textit{Freq}                        & \begin{tabular}[c]{@{}c@{}}Performance: 4.9 GHz\\ Efficient: 3.6 GHz\end{tabular}                                                   & \begin{tabular}[c]{@{}c@{}}Performance: 4.7 GHz\\ Efficient: 3.6 GHz\end{tabular}                                                     & \begin{tabular}[c]{@{}c@{}}Turbo: 3.7 GHz\\ Base: 3.3 GHz\end{tabular}                                 \\ \cline{2-2}
                                                   & \textit{Cache}                       & \begin{tabular}[c]{@{}c@{}}L2 per core: 1.25 MB\\ L3 shared: 25 MB\end{tabular}                                                     & \begin{tabular}[c]{@{}c@{}}L2 per core: 1.25 MB\\ L3 shared: 24 MB\end{tabular}                                                       & \begin{tabular}[c]{@{}c@{}}L2 per core: 256 KB\\ L3 shared: 6 MB\end{tabular}                          \\ \cline{2-2}
                                                   & \textit{TDP}                         &                                                                                                                                     &                                                                                                                                       & 95 W                                                                                                   \\ \hline
            \textit{GPU}                           & \textit{Device}                      & GeForce GTX 1650 Laptop \textbackslash{}cite\{1650\}                                                                                & GeForce RTX 3050 Laptop \textbackslash{}cite\{3050\}                                                                                  & Intel HD Graphics 3000 \textbackslash{}cite\{IHG3000\}                                                 \\ \cline{2-2}
                                                   & \textit{Cores}                       & 896 CUDA cores                                                                                                                      & 2048 CUDA cores                                                                                                                       &                                                                                                        \\ \cline{2-2}
                                                   & \textit{Max Freq}                    &                                                                                                                                     & 1.74 GHz                                                                                                                              & 1.1 GHz                                                                                                \\ \cline{2-2}
                                                   & \textit{Mem}                         &                                                                                                                                     & 4 GB GDDR6                                                                                                                            & \begin{tabular}[c]{@{}c@{}}Shared with CPU\\ (integrated)\end{tabular}                                 \\ \hline
            \multicolumn{2}{c|}{\textit{Memory}}   & 33.4 GB                              & 16 GB DDR4                                                                                                                          & 16 GB DDR3                                                                                                                                                                                                                                     \\ \hline
            \multicolumn{2}{c|}{\textit{Network}}  & Gigabit Ethernet                     & Gigabit Ethernet                                                                                                                    &                                                                                                                                                                                                                                                \\ \hline
            \multicolumn{2}{c|}{\textit{OS}}       & Ubuntu 24.04                         & Linux Mint 21.3                                                                                                                     & Arch Linux 6.9.3-arch1-1                                                                                                                                                                                                                       \\ \hline
            \multicolumn{2}{c|}{\textit{Power}}    &                                      &                                                                                                                                     &                                                                                                                                                                                                                                                \\ \hline
            \multicolumn{2}{c|}{\textit{Price}}    &                                      & 1299€                                                                                                                               &                                                                                                                                                                                                                                                \\ \hline
        \end{tabular}}
    \caption{Tabla de equipos Intel}
    \label{tab:x86}
\end{table}

Por otro lado, en la tabla \ref{tab:x86} se muestran los equipos de la \textbf{familia Intel} empleados en el proyecto. La columna \textit{Reference} muestra el equipo de pruebas utilizado en \cite{andrew} para realizar comparaciones de resultados. Mientras que el equipo de sobremesa posee una \acrshort{cpu} de mayor potencia y prestaciones, la \acrshort{gpu} del portátil cuenta con una arquitectura superior, incluyendo los \textbf{tensor y ray tracing} cores a diferencia de la \acrshort{gpu} del PC del laboratorio.

\begin{figure}[tbp]
    \centering
    \includegraphics[width=0.75\linewidth]{imagenes/memcpy.jpg}
    \caption[Arquitectura de interconexión de una \acrshort{gpu} discreta y una \acrshort{gpu} integrada]{Arquitectura de interconexión de una \acrshort{gpu} discreta y una \acrshort{gpu} integrada, figura extraída de \cite{zerocpy}}
    \label{fig:memcpy}
\end{figure}

Ambas familias poseen \textbf{diferencias significativas} que repercuten directamente en el rendimiento. Por un lado, la familia Intel posee \acrshort{gpu}s y \acrshort{cpu}s de \textbf{mayor frecuencia} respecto a la familia Jetson. En cambio, la \acrshort{gpu} GeForce GTX 1650 (gráfica del PC lab) es la única que \textbf{no posee tensor cores}. Las \acrshort{gpu}s de la familia Intel son \textbf{discretas}, es decir, tienen su propia memoria (\acrshort{dram}) separada de la \acrshort{cpu}, con la que se comunica mediante \acrshort{PCIe}. Por otro lado, las \acrshort{gpu}s de la familia Jetson se encuentran \textbf{integradas con la \acrshort{cpu}}, es decir, ambas comparten la memoria del sistema. Las \acrshort{gpu}s integradas pueden beneficiarse de la compartición de la \acrshort{dram} con la \acrshort{cpu}, evitando así operaciones de transferencia de datos como los \textbf{cudaMemcpy}. En el apéndice \ref{sec:pmm} se explican más detalles al respecto.

La figura \ref{fig:memcpy} muestra un esquema de interconexión de una \acrshort{gpu} discreta (familia Intel) y una \acrshort{gpu} integrada (familia Jetson), en la que la primera requiere un cudaMemcpy para poder disponer de los datos en la \acrshort{gpu} y la segunda evita realizar transferencias debido a que se utiliza la misma \acrshort{dram} (\textbf{zero-copy}). La conexión \acrshort{PCIe} en los equipos Intel resulta un \textbf{cuello de botella} en el ancho de banda de las transferencias de memoria entre \acrshort{cpu} y \acrshort{gpu}, con tan solo \textbf{31.5 \acrshort{gb}/s} en la versión 4. Para colmo, las memorias DDR4 de los equipos Intel se encuentran \textbf{por debajo} del ancho de banda ofrecido por \acrshort{PCIe} (\textbf{25.6 \acrshort{gb}/s}), lo que \textbf{limita el rendimiento} de operaciones de \textbf{transferencia de memoria} entre dispositivos y no aprovecha al máximo la frecuencia de trabajo de la \acrshort{gpu}.

\section{Hardware específico}
\label{sec:hw}

A continuación se comentan los componentes hardware específicos sobre los que funciona el sistema del proyecto: robot móvil y cámaras \acrshort{rgbd}.

\subsection{Robot móvil Summit\_XL}
\label{sec:summit}

\begin{figure}[tbp]
    \centering
    \begin{subfigure}[t]{0.3\textwidth}
        \includegraphics[width=\textwidth]{imagenes/KINECT.jpg}
        \caption{Cámara Microsoft Kinect}
        \label{fig:kin1}
    \end{subfigure}
    \hspace{3em}
    \begin{subfigure}[t]{0.3\textwidth}
        \includegraphics[width=\textwidth]{imagenes/SUMKINECTS.jpg}
        \caption{Cámaras Kinect instaladas}
        \label{fig:kin2}
    \end{subfigure}
    \caption{Summit\_XL con las 2 cámaras Kinect instaladas}
\end{figure}

El sistema del proyecto se ha construido para funcionar en el robot móvil \textbf{Summit\_XL} \cite{rb}. Dicho robot es de tipo \textbf{diferencial} (es decir, permite el movimiento en todas las direcciones excepto en sus laterales) y dispone de cuatro ruedas de goma para su movilidad en una amplia variedad de entornos. Entre sus componentes, destaca un sensor Velodyne \acrshort{lidar} \cite{lidar} (sensor circular de la figura \ref{fig:kin2}) que genera nubes de puntos \acrshort{3d} del entorno y una cámara \acrshort{rgb} frontal, que no se ha empleado en este proyecto debido a su baja calidad de imagen. A mayores, se instalaron \textbf{dos cámaras Kinect \acrshort{rgbd}} en los laterales (figura \ref{fig:kin2}).

El Summit\_XL utiliza una arquitectura basada en el \textbf{\textit{middleware} \acrshort{ros}} \cite{ROS}, que permite la fácil interacción e integración de múltiples sensores con el robot. Este cuenta con soporte para las versiones de \textbf{\acrshort{ros}2}, sin embargo, el modelo adquirido por el \acrshort{citic} emplea \acrshort{ros}1, que a fecha de esta memoria ya ha alcanzado su \textbf{final de vida} \cite{ROSEOL}.

\subsection{Cámaras \acrshort{rgbd}}

Para nutrir al sistema de imágenes se cuenta con las cámaras \acrshort{rgbd} Kinect de Microsoft \cite{kinect} (figura \ref{fig:kin1}). Dichas cámaras capturan la información del entorno en forma de imágenes \acrfull{rgb} y realiza reconstrucciones \acrshort{3d} en tiempo real, que otorga en forma de imágenes de profundidad o de distancias (depth).

Para este proyecto, se cuenta con dos de estas cámaras (aunque se planea añadir otras dos más), cada una con una cobertura de visión de 57º en horizontal. Dichas cámaras se disponen de forma que \textbf{no existan solapes} (en los laterales del robot, como se muestra en la figura \ref{fig:kin2}), lo que otorga un rango de visión total de 114º.

\section{Software específico}
\label{sec:sw}

A continuación se expone el software que compone el núcleo del sistema de este proyecto: \acrshort{ros}, las librerías de Python específicas y las herramientas para su virtualización y \gls{profiling}.

\subsection{ROS}
\label{sec:ros}

\acrfull{ros} \cite{ROS} es un \textit{middleware} de \textbf{código abierto} que incorpora las herramientas necesarias para la interacción y desarrollo de software en robots en múltiples lenguajes de programación (principalmente Python y C++). Cuenta con una amplia biblioteca de distribuciones y librerías desarrolladas por la comunidad, entre las que se incluyen implementaciones de algoritmos y drivers de sensores.

Cualquier entorno de desarrollo de \acrshort{ros} está compuesto por \textbf{paquetes}, que son las unidades funcionales del entorno de \acrshort{ros} y  contienen el código y los archivos de configuración y de lanzamiento de los nodos, entre otros componentes. Los entornos de \acrshort{ros} son \textbf{heterogéneos}, es decir, permiten ejecutar paquetes implementados en múltiples lenguajes de programación.

\acrshort{ros} crea y mantiene una \textbf{red distribuida} en la que los nodos se ejecutan e intercambian mensajes. Se sigue un patrón \textbf{\textit{publish/subscribe}} (publicador/subscriptor), en el que existen nodos que publican información por medio de \textbf{tópicos}, disponibles para cualquier nodo que desee suscribirse a dicho tópico.

\begin{figure}[tbp]
    \centering
    \includegraphics[width=0.56\linewidth]{imagenes/mastel.jpg}
    \caption[Arquitectura maestro-esclavo de ROS]{Arquitectura maestro-esclavo de ROS, figura extraída de \cite{mastel}}
    \label{fig:mastel}
\end{figure}

En \acrshort{ros}1, se sigue una arquitectura \textbf{maestro-esclavo} \cite{mastel} (figura \ref{fig:mastel}). El maestro registra a todos los publicadores/subscriptores de la red, de forma que si un nodo quiere suscribirse/publicar un tópico, este se comunica con el maestro para registrar/conocer la manera de conectarse a dicho tópico. El resto de las comunicaciones se realizan directamente entre los nodos. El principal inconveniente es que el maestro es un \textbf{punto único de fallo} (si este deja de estar disponible, todo el sistema falla). \acrshort{ros}2 prescinde de un nodo maestro, por lo que no sufre de dicha limitación.  

Existe una amplia variedad de distribuciones de este software según la versión del \acrshort{so}. Para este proyecto se han empleado las distribuciones \textbf{Noetic, Humble y Rolling}, para Ubuntu 20.04, 22.04 y 24.04, respectivamente. El Summit\_XL tiene la distribución Melodic, ya que tiene la versión 18.04 de Ubuntu. En las versiones de Ubuntu superiores a 20.04 \textbf{no es posible instalar} \acrshort{ros}1 (descontinuado) a menos que se compile desde el código fuente.

\subsection{Librerías de Python}
A continuación nombraremos las librerías de Python empleadas en el proyecto (todas son de \textbf{código abierto}):
\begin{description}
    \item[PyCUDA] \cite{pycuda} Es un \gls{wrapper} para Python que interacciona con el driver de \gls{CUDA}. Se ha utilizado principalmente para generar los contextos y reservar la memoria en la \acrshort{gpu} con el propósito de realizar las inferencias de los modelos.
    \item[NumPy] \cite{np} Es una librería que incluye estructuras de datos y funciones para operar con matrices multidimensionales. Se ha empleado en este proyecto para representar y manipular las imágenes y la información generada por los modelos en forma de vectores y matrices multidimensionales.
    \item[SciPy] \cite{scipy} Implementa numerosos algoritmos relacionados con la computación científica. En este proyecto se ha usado para aplicar el método húngaro, hallar los parámetros que modelan una distribución de Weibull, calcular distancias coseno, entre otras aplicaciones relacionadas con cálculos científicos.
    \item[OpenCV] \cite{cv2} Es una librería con una gran variedad de funciones relacionadas con la visión por computadora. Se ha utilizado para la obtención y el procesado de imágenes y entradas y salidas de las redes neuronales (ejemplo: \gls{nms}), este último por medio del módulo de redes neuronales profundas (dnn), que contiene múltiples algoritmos y modelos ya implementados con los que se pueden realizar inferencias por medio de su \gls{rt}. En este proyecto se ha utilizado para ejecutar un modelo de detección de caras. Dicho módulo puede ejecutarse en \textbf{CUDA}, aunque solo si se \textbf{compila el código fuente}. Para este proyecto Ee proceso seguido para compilar OpenCV es el mismo que en \cite{OpenCVCUDA}.
    \item[TensorRT] \cite{x36} Contiene una \acrshort{api} y una herramienta (\textit{trtexec}) para la optimización y ejecución de inferencias de modelos en aceleradoras de NVIDIA. La \acrshort{api} puede integrarse directamente en \textit{frameworks} como PyTorch o TensorFlow o utilizarse directamente con los modelos exportados al formato \acrfull{onnx} (ver apéndice \ref{chap:procedure} para más detalles). Se ha empleado en este proyecto para optimizar y ejecutar las inferencias de los modelos en las \acrshort{gpu} de los dispositivos.

        %La versión 10 de TensorRT está disponible para las aceleradoras de NVIDIA con un \textit{compute capability} igual o superior a 7.5. En el caso de querer usar el software con una \textit{compute capability} inferior (ejemplo: Jetson Xavier NX), es necesario instalar la versión 8 u otra anterior. El código de la \acrshort{api} generado en la versión 8 \textbf{no es compatible} con la versión 10.

    \item[OpenVINO] \cite{OpenVINO} Desarrollado por Intel para la optimización y ejecución de inferencias en \acrshort{cpu}s \acrshort{arm} e Intel y en aceleradoras de Intel (\acrshort{gpu} y \acrshort{npu}). Permite la conversión y optimización de modelos entrenados en \textit{frameworks} como Tensorflow y PyTorch para su ejecución por medio de la \acrshort{api} que ofrece. Se ha empleado para optimizar y ejecutar las inferencias de los modelos en la \acrshort{cpu} de los dispositivos.
\end{description}

%\subsection{Optimizadores y \glspl{rt} de inferencias}

%Es el software encargado de optimizar la arquitectura de los modelos y ejecutar sus inferencias. En este proyecto se han utilizado los siguientes (todos son de \textbf{código abierto}):
%\begin{description}
%    \item[TensorRT] \cite{x36} Contiene una \acrshort{api} y una herramienta (\textit{trtexec}) para la optimización y ejecución de inferencias de modelos en aceleradoras de NVIDIA. La \acrshort{api} puede integrarse directamente en \textit{frameworks} como PyTorch o TensorFlow o utilizarse directamente con los modelos exportados al formato \acrfull{onnx} (ver apéndice \ref{chap:procedure} para más detalles). Se ha empleado en este proyecto para optimizar y ejecutar las inferencias de los modelos en las \acrshort{gpu} de los dispositivos.

%La versión 10 de TensorRT está disponible para las aceleradoras de NVIDIA con un \textit{compute capability} igual o superior a 7.5. En el caso de querer usar el software con una \textit{compute capability} inferior (ejemplo: Jetson Xavier NX), es necesario instalar la versión 8 u otra anterior. El código de la \acrshort{api} generado en la versión 8 \textbf{no es compatible} con la versión 10.

%   \item[OpenVINO] \cite{OpenVINO} Desarrollado por Intel para la optimización y ejecución de inferencias en \acrshort{cpu}s \acrshort{arm} e Intel y en aceleradoras de Intel (\acrshort{gpu} y \acrshort{npu}). Permite la conversión y optimización de modelos entrenados en \textit{frameworks} como Tensorflow y PyTorch para su ejecución por medio de la \acrshort{api} que ofrece. Se ha empleado para optimizar y ejecutar las inferencias de los modelos en la \acrshort{cpu} de los dispositivos.

%   \item[OpenCV] \cite{cv2} El módulo de redes neuronales profundas (dnn) incluye modelos ya entrenados y un \gls{rt} para realizar inferencias. En este proyecto se ha utilizado un modelo de detección de caras ofrecido por dicha librería. Dicho módulo puede ejecutarse en \textbf{CUDA}, aunque solo si se \textbf{compila el código fuente}. Para este proyecto El proceso seguido para compilar OpenCV es el mismo que en \cite{OpenCVCUDA}.
%\end{description}

\subsection{Docker y Nsight Systems}

\begin{description}
    \item[Docker] \cite{docker} es un software de \textbf{virtualización} de \textbf{código abierto}, que permite realizar despliegues automatizados mediante ficheros de configuración. El software se ejecuta en un entorno aislado del sistema operativo llamado \textbf{contenedor}, dichos contenedores proporcionan seguridad y portabilidad. Esta herramienta ha sido de gran utilidad para desplegar el software del proyecto en los diferentes dispositivos.
    \item[Nsight Systems] \cite{nsys} Es una herramienta de \gls{profiling} que permite rastrear la ejecución de funciones de librerías, eventos de \gls{CUDA} y demás operaciones ejecutadas tanto en \acrshort{cpu} como en \acrshort{gpu}. Se ha utilizado para analizar el comportamiento en la ejecución de inferencias en los dispositivos de las pruebas.
\end{description}

\section{Software general}
A continuación, se mencionan las herramientas relacionadas con la gestión y desarrollo del proyecto y de la memoria. Menos Trello, todas son de \textbf{código abierto}: 
\begin{description}
    \item[Git] \cite{git} Es un sistema de control de versiones que permite mantener un historial de las versiones de un proyecto, de forma que se mantiene la trazabilidad del mismo. \textbf{GitHub} es una plataforma en la nube para alojar y compartir proyectos de Git. El código \footnote{\url{https://github.com/cvregueiro/lidargb-person-following}} y la memoria \footnote{\url{https://github.com/Omimacgithub/NotOverleafAnymore}} de este proyecto se ha subido a esta plataforma para su fácil acceso desde múltiples dispositivos.
    \item[Visual Studio Code] \cite{vscode} Es un entorno de desarrollo integrado (IDE) que contiene una amplia variedad de funciones integradas, como un cliente de Git con una sencilla interfaz y una extensión para manejar contenedores de Docker (\emph{Dev Containers}), entre las funciones más usadas en este proyecto.  También cuenta con una extensión de \textbf{LaTeX} (LaTeX Workshop) altamente configurable, que permite el autocompletado de código LaTeX y su compilación de una forma similar a Overleaf \cite{overleaf}. Se ha optado por utilizar \textbf{VS Code + LaTeX Workshop} para escribir la memoria en sustitución de Overleaf, debido a las limitaciones en la compilación de la versión gratuita de esta última herramienta.
    \item[Trello] \cite{trallao} Es un programa online para la gestión de proyectos Kanban, en el que cada proyecto es un tablero con distintas listas de tarjetas que representan a las tareas. Se ha empleado para llevar un seguimiento de las tareas del proyecto en sus diferentes fases del desarrollo.
    \item[Draw.io] \cite{drawio} Es una aplicación para la creación de todo tipo de diagramas. Se ha utilizado para crear la mayoría de diagramas y figuras de la memoria de este proyecto.
\end{description}