\chapter{Procesamiento de videos}
\label{chap:video}
\lettrine{E}{n} este capítulo se comenta todo el trabajo relacionado con la extracción de caras en secuencias de video.

El sistema actual trabaja a nivel de frame \cite{andrew}, es decir, devuelve reconocimientos de los individuos presentes en un frame. Esta aproximación permite trabajar a altas frecuencias, sin embargo, las predicciones dependen enteramente de la calidad del frame (ejemplo: nivel de borrosidad). En este proyecto se ha optado por trabajar con \textbf{secuencias de frames}, de esta forma, se devuelve un reconocimiento más robusto, aunque hace que el sistema trabaje a una menor frecuencia.

\section{Seguimiento de caras}
En casos como los del \textit{\gls{dataset}} FACE COX \cite{cox}, donde en los videos siempre aparece una sola persona, la agrupación de las caras es trivial (se sabe que todas pertenecen a la misma persona). Sin embargo, en un video donde aparecen múltiples individuos que se entrecruzan, es necesario adoptar un método para seguir el rastro de cada individuo entre frames.

El \textbf{método húngaro} es un algoritmo que resuelve el problema de la asignación óptima. Dada una matriz de costes en el que cada fila establece una correlación con cada columna, dicho algoritmo buscará la \textbf{asignación óptima}, es decir, de menor coste entre elementos, en este caso, dos frames adyacentes \cite{Hungarian}. La figura \ref{fig:Seq} muestra un ejemplo de funcionamiento del método, en cada frame, se aplica el modelo de detección de caras (YuNet) y se generan las \glspl{bbox}

\section{Reidentificación de entidades}
Es posible que durante la detección de individuos, las \glspl{bbox} de una misma persona en frames consecutivos no se solapen, debido a la velocidad de movimiento del individuo por ejemplo. En este caso, existe el riesgo de que el método húngaro asocie incorrectamente la identificación a otra persona a la que le ocurre esta misma situación. Otro problema es el no seguimiento de la persona cuando esta se encuentra totalmente ocluida (ejemplo: se cruza un individuo justo delante) y vuelve a aparecer.

El uso de \textbf{filtros de Kalman} \cite{Hungarian,MangoYOLO}.

A partir del método de tracking propuesto no es posible recuperar el rastro de las entidades que desaparecieron de forma intermitente. Para resolver este problema se ha optado por calcular la distancia euclidiana entre la última \gls{bbox} del individuo en cuestión y la \gls{bbox} de una nueva detección, si la distancia entre ellas es menor a un umbral, se reasigna la detección al individuo.

\begin{figure}
    \centering
    \includegraphics[width=1\linewidth]{imagenes/Seq.jpg}
    \caption{Funcionamiento del método húngaro para el seguimiento de caras, como salida se obtiene una lista de recortes faciales agrupados por entidad}
    \label{fig:Seq}
\end{figure}

El funcionamiento de esta técnica se muestra en la figura (TODO), se escoge un tamaño de secuencia (15 o 25 en las pruebas realizadas) y se extraen los frames. En cada frame, puede haber entre 0 y n entidades, como no se conoce la identidad de cada individuo en este punto, es necesario realizar un \textbf{rastreo} (tracking) de las entidades en cada uno de los frames. El tracking implementado calcula una matriz de costes entre las detecciones de un frame y el frame posterior, cada coste representa un valor de solape entre \glspl{bbox}, cuanto menor mayor es el solape, finalmente, se realiza una asignación óptima entre detecciones mediante el \textbf{método húngaro} (como se plantea en \cite{Hungarian}).

\section{Coherencia entre múltiples cámaras}