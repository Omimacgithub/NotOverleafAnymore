\chapter{Gestión del proyecto}
\lettrine{E}{n} este capítulo se ahondará en la planificación del proyecto, su seguimiento, el plan de gestión de riesgos empleado y se expondrá un análisis de los recursos y su costo.

\section{Análisis de requisitos}
Los requisitos definen las pautas que el sistema desarrollado debe cumplir para alcanzar los objetivos establecidos. Los requisitos pueden dividirse en los dos siguientes tipos.

\subsection{Requisitos funcionales}
Definen las funcionalidades que el sistema debe de ofrecer. Para este proyecto, se busca ampliar el sistema base para que cumpla con los siguientes requisitos:

\begin{itemize}
    \item Pasar de reconocimientos frame a frame a ventanas de frames, aprovechando así la coherencia espacio-temporal.
    \item Reconocimiento de desconocidos (\textit{Open-Set}) e inclusión al sistema (\textit{Open-World}): el sistema debe ser capaz de reconocer y agregar nuevas personas a su base de datos.
    \item El sistema debe de ser capaz de inicializarse con una cantidad muy reducida de muestras. Las muestras pueden ser provistas por el operador o incluso ser recolectadas de manera autónoma. Dado que la recopilación de datos biométricos (ejemplo: caras y cuerpos) está fuertemente regulada por la ley de protección de datos, este requisito se vuelve casi obligatorio para el sistema.
\end{itemize}
\subsection{Requisitos no funcionales}
Definen las características técnicas que debería de cumplir el sistema:
\begin{itemize}
    \item Que el sistema evolucione adecuadamente (ejemplo: evitar caídas drásticas en la precisión) a lo largo del tiempo.
    \item Operación en tiempo real: sujeto a la frecuencia del componente más "lento". En el caso de este proyecto, las cámaras RGBD otorgan una imagen RGB cada 100 ms (\textbf{10 Hz}). El objetivo es que el sistema sea capaz de mantener dicha frecuencia.
    \item Portabilidad: la ejecución del sistema no debe de depender exclusivamente de la arquitectura o de las diferentes versiones de librerías.
    \item Escalabilidad: debe permitir incluir nuevos componentes (ejemplo: cámaras) sin comprometer significativamente al rendimiento.
    \item TODO Flexible: todos los requisitos anteriores deben cumplirse para los dispositivos del proyecto.
    \item Usabilidad: el sistema debe de poder instalarse, configurarse y desplegarse de forma sencilla e intuitiva para un usuario ajeno al mismo.
\end{itemize}
\subsection{Actores}
A continuación se definen los actores (o roles) implicados y sus papeles:

\begin{itemize}
    \item Usuario o persona registrada: individuo que acepta registrar sus datos biométricos para enriquecer el sistema.
    \item Operador del sistema: encargado de ejecutar y mantener el sistema.
\end{itemize}

\section{Metodología de desarrollo}
\label{sec:method}

Para el presente proyecto se ha empleado una metodología basada en Kanban (TODO: referencia). Kanban es una herramienta de gestión \textbf{visual}, cuyo objetivo es dividir el trabajo de un proyecto en pequeñas tareas concretas, agilizando así el desarrollo. El proyecto en sí representa un tablero y cada tarea representa una tarjeta, de esta forma, el equipo de desarrollo tiene una visión global de todas las tareas del proyecto. Las tareas en Kanban pasan por múltiples fases que pueden variar según las necesidades del proyecto. Se definen las siguientes fases aplicadas de esta aproximación de Kanban:
\begin{itemize}
    \item \textbf{\textit{Backlog}}: son las tareas que se encuentran en fase de \textbf{definición} y que esperan a ser divididas en una o varias tareas más concretas.
    \item \textbf{\textit{TO-DO}}: en esta fase se ubican las tareas que representan objetivos concretos, pero que aún no se han iniciado.
    \item \textbf{\textit{DOING}}: son las tareas de la fase \textit{to-do} cuyo desarrollo ha comenzado.
    \item \textbf{\textit{DONE}}: las tareas cuyo desarrollo ha finalizado.
\end{itemize}

\section{Planificación}

La planificación del proyecto se planteó para llevarse a cabo durante casi 12 meses, desde finales de enero de 2025 hasta principios de enero de 2026. El trabajo realizado se divide en las siguientes tareas, cada una dividida en tareas más concretas:

\subsection{T0. Revisión de la literatura}
Esta tarea representa el preludio del proyecto.

\textbf{Tareas:}
\begin{itemize}
    \item Estudio de trabajos que hayan utilizado dispositivos Jetson para ejecutar modelos de inteligencia artificial de forma eficiente.
    \item Revisar la memoria del trabajo fin de grado del sistema que se pretende extender y optimizar \cite{andrew}.
\end{itemize}

\textbf{Duración estimada:} 3 semanas

\subsection{T1. Instalación y ejecución del sistema de partida}
Esta tarea pretende obtener una versión inicial del software funcionando en los dispositivos de desarrollo y/o pruebas.

\textbf{Tareas:}
\begin{itemize}
    \item Creación de un Dockerfile que incluya todas las librerías y aplicaciones necesarias para desplegar y hacer funcionar el sistema.
    \item Primeras pruebas del sistema a modo de comprobación de que todo funciona según lo esperado.
    \item Registro de primeros resultados, que servirán de línea base para futuras comparaciones.
\end{itemize}

\textbf{Duración estimada:} 2 semanas

\subsection{T2. Optimización de inferencias de los modelos}
Esta tarea pretende lograr un recorte en la latencia de los componentes más pesados del sistema, las \acrfull{cnn}, a partir de la potencia de cómputo de las \acrshort{gpu}s disponibles.

\textbf{Tareas:}
\begin{itemize}
    \item Exportación de los modelos existentes al formato \acrshort{onnx}.
    \item Conversión de los modelos \acrshort{onnx} al formato de los modelos de TensorRT mediante \textit{trtexec} o por medio de la \acrshort{api} de Python de TensorRT.
    \item Creación de códigos de inferencia que reciban como entrada el flujo de datos del sistema.
\end{itemize}

\textbf{Duración estimada:} 2 meses

\subsection{T3. Pruebas de los modelos y del sistema}
Esta tarea consiste en poner a prueba los componentes desarrollados:

\textbf{Tareas:}
\begin{itemize}
    \item Preparación y ejecución de las pruebas de modelos.
    \item Corrección en la precisión si decrece significativamente respecto a la línea base.
    \item Registro de los resultados.
    \item Preparación y ejecución de las pruebas del sistema.
    \item Corrección del sistema si los resultados de precisión decaen respecto a la línea base.
    \item Realizar pruebas variando el número de cámaras en ejecución y midiendo los recursos utilizados.
    \item Registro y discusión de los resultados.
\end{itemize}

\textbf{Duración estimada:} 1 mes

\subsection{T4. Migración a ROS 2}
Esta tarea se encarga de realizar el proceso de migración de \acrshort{ros} 1 a \acrshort{ros} 2 con el fin de seguir recibiendo soporte y permitir la ejecución del sistema en nuevas distribuciones de Ubuntu.

\textbf{Tareas:}
\begin{itemize}
    \item Preparar un nuevo Dockerfile para instalar \acrshort{ros} 2 junto al sistema.
    \item Revisar la guía de migración oficial de \acrshort{ros} 2.
    \item Realizar los ajustes necesarios para que el proyecto pueda compilar en \acrshort{ros} 2.
    \item Migrar los nodos del sistema a \acrshort{ros} 2.
    \item Comprobar que el sistema funciona correctamente en \acrshort{ros} 2.
\end{itemize}

\textbf{Duración estimada:} 3 semanas

\subsection{T5. Procesamiento de vídeo}
Debido a las características del método de aprendizaje máquina a implementar. Es necesario extender el sistema para procesar vídeos (secuencias de caras). En esta tarea, se implementa dicha funcionalidad y se comprueba que funcione según lo esperado.

\textbf{Tareas:}
\begin{itemize}
    \item Agrupar las caras que aparecen en la secuencia por las etiquetas de las entidades otorgadas por el \textit{\gls{dataset}}.
    \item Obtener una nueva versión del modelo de reconocimiento facial que sea capaz de procesar múltiples imágenes por inferencia (modo \textit{\gls{batch}}).
    \item Modificar el código para devolver un reconocimiento por cada secuencia de caras de entrada y comprobar si la precisión no decae.
    \item Una vez que el código funcione correctamente, se procede a diseñar e implementar el método de seguimiento para agrupar las caras.
    \item Tras implementar el seguimiento, se diseña e implementa el método de reidentificación.
    \item Se comprueba que el seguimiento y reidentificación es capaz de mantener o de mejorar la precisión del sistema original.
\end{itemize}

\textbf{Duración estimada:} 1 mes

\subsection{T6. Diseño e implementación del sistema extendido}
Esta enorme tarea abarca todos los pasos desde la comprensión de los conceptos hasta la obtención del código de los nuevos componentes del sistema.

\textbf{Tareas:}
\begin{itemize}
    \item Entender los conceptos que sustentan la adaptación que se quiere implementar \cite{Erik}.
    \item Diseñar los componentes necesarios para permitir el reconocimiento de entidades desconocidas (\textit{Open-Set}).
    \item Implementar dichos componentes, acoplados con el procesamiento de video.
    \item Diseñar los componentes necesarios para permitir agregar nuevo conocimiento acerca de entidades existentes o desconocidas (\textit{Open-World}).
    \item Implementar dichos componentes, acoplados con el reconocimiento de desconocidos (\textit{Open-Set}).
\end{itemize}

\textbf{Duración estimada:} 2 meses

\subsection{T7. Pruebas del reconocimiento adaptativo}
Esta tarea trata de realizar múltiples pruebas con el fin de estudiar el comportamiento del reconocimiento implementado.

\textbf{Tareas:}
\begin{itemize}
    \item Generar las pruebas que se van a realizar.
    \item Ejecutar las pruebas y extraer los resultados.
    \item Modificar el código del sistema si este no se comporta según lo esperado.
    \item Otorgar una discusión acerca de los resultados obtenidos.
\end{itemize}

\textbf{Duración estimada:} 1 mes

\subsection{T8. Análisis de rendimiento}
En esta tarea, se evalúa el rendimiento de los dispositivos que pasaron a estar disponibles durante el transcurso del proyecto, con el objetivo de establecer comparaciones y así sacar conclusiones.

\textbf{Tareas:}
\begin{itemize}
    \item Instalar el sistema mediante un Dockerfile modificado según el dispositivo.
    \item Ejecutar las pruebas y obtener los resultados.
    \item Generar una tabla comparativa entre dispositivos.
    \item Discutir los resultados obtenidos.
\end{itemize}

\textbf{Duración estimada:} 2 semanas

\subsection{T9. Documentación del proyecto}
Esta tarea representa el tiempo de vida del proyecto. Desde la concepción del mismo, se documentan todos los avances logrados. En la etapa final, se centran los esfuerzos en cubrir los últimos detalles de la memoria, concretamente los siguientes puntos:

\begin{itemize}
    \item Añadir últimos resultados y repetir pruebas en caso de que fuese necesario.
    \item Añadir discusiones acerca de las pruebas.
    \item Ajustar el contenido a los cambios surgidos en el proyecto.
    \item Añadir los capítulos de fundamentos teóricos y prácticos.
    \item Añadir la sección de seguimiento, gestión de riesgos, recursos y costes del proyecto.
    \item Añadir las conclusiones y el trabajo futuro.
    \item Confirmar la estructura de la memoria.
    \item Agregar apéndices si fuera necesario.
    \item Repaso del contenido de la memoria para correcciones y para cubrir secciones incompletas.
    \item Revisar bibliografía.
\end{itemize}

\textbf{Duración estimada:} 12 meses

\subsection{Diagrama de Gantt}

\begin{figure}[tb]
    \centering
    \includegraphics[width=1.1\linewidth]{imagenes/Gantt.jpg}
    \caption{Diagrama de Gantt del proyecto}
    \label{fig:Gantt}
\end{figure}

En la figura \ref{fig:Gantt} se muestra el diagrama de Gantt del proyecto con todas las dependencias entre tareas y su duración. Los recuadros en azul son las tareas situadas en el instante temporal predicho en la planificación, mientras que los recuadros en rojo muestran los instantes de realización reales de las tareas.

La duración de las tareas se planificó con cierta holgura, de forma que puedan reducir el impacto en tiempo ante posibles imprevistos. Teniendo como objetivo la deposición de este trabajo en febrero, se programó el desarrollo para finalizar en enero, de forma que se dispone de un mes adicional si los objetivos no se logran en el tiempo previsto.

\section{Gestión de riesgos}

\begin{table}[tbp]
    \footnotesize
    \setlength{\tabcolsep}{6pt} % Default value: 6pt
    \begin{tabular}{|p{4.5cm}|>{\centering\arraybackslash}p{1.9cm}|>{\centering\arraybackslash}p{1.6cm}|p{4.5cm}|}
        \hline
        \textbf{Riesgo}                                         & \textbf{Probabilidad (1-10)} & \textbf{Impacto (semanas)} & \textbf{Respuesta}                                  \\
        \hline
        \textit{Simulaciones lentas}                            & 10                           & 1                          & comprar nuevo portátil                              \\\hline
        \textit{El robot no arranca correctamente}              & 7                            & 1 día                      & terminal de rescate, sino backup y reinstalar       \\\hline
        \textit{Cambios de misiones y/o acciones}               & 7                            & 1                          & realizar los cambios en el código                   \\ \hline
        \textit{Pruebas reales no funcionan como en simulación} & 7                            & 1                          & resolver el fallo en el robot real                  \\ \hline
        \textit{Baja laboral del tutor}                         & 3                            & 3                          & reforzar la  memoria                                \\ \hline

        \hline
        \textit{El servicio técnico no responde}                & 5                            & 2 días                     & buscar plan alternativo                             \\\hline
        \textit{Sucesos inesperados en pruebas reales}          & 2                            & 4                          & priorizar misiones en exteriores y descartar WebRTC \\\hline
    \end{tabular}
    \caption{Gestión de riesgos del proyecto}
    \label{tab:risks}
\end{table}

La tabla \ref{tab:risks} muestra el plan de gestión de riesgos elaborado para el proyecto. Para cada riesgo, se indica su probabilidad, el impacto en semanas de retraso, así como la respuesta prevista para mitigarlo.

Todos los riesgos expuestos han terminado sucediendo, excepto el del método de seguimiento, que ha logrado funcionar mejor de lo esperado en los videos probados. También surgieron riesgos no previstos inicialmente, como que el sistema inicial no funcionase correctamente al aplicar más de 2 cámaras.

\section{Seguimiento}
%TODO: Cada una de las tareas del proyecto se añaden a la fase \textit{backlog}, que posteriormente se dividen en varias tareas que se incorporan a la fase \textit{to-do}.

%TODO: El desarrollo del proyecto en líneas generales se ha mantenido según lo previsto. Los imprevistos que surgieron se pudieron suplir en cierta medida con la realización de otras tareas, pero han provocado retardos en el proyecto. Los siguientes imprevistos:
El seguimiento del proyecto siguió el flujo explicado en la sección \ref{sec:method}. Las fases del proyecto se añadieron al \textit{backlog} del tablero del proyecto, para dividirse en pequeñas tareas realizables, que pasarían a la fase \textit{TO-DO}. Las tareas de la fase \textit{TO-DO} pasan a la fase \textit{DOING}, para indicar que están en proceso de desarrollo. Finalmente, las tareas realizadas pasan a la fase \textit{DONE}.

En la figura \ref{fig:khan} se muestra una instantánea del tablero Kanban de este proyecto. El enfoque Kanban ha permitido priorizar en todo momento las tareas importantes del proyecto gracias a su representación visual. En este caso, la proporción de tareas en la fase \textit{DOING} es mayor que la fase \textit{TO-DO}, lo que indica que conviene enfocarse en terminar algunas de las tareas en proceso antes de empezar con otras nuevas.

\begin{figure}[tb]
    \centering
    \includegraphics[width=0.75\linewidth]{imagenes/Trallao.png}
    \caption{Instantánea del tablero Kanban del proyecto}
    \label{fig:khan}
\end{figure}

En líneas generales, el seguimiento sufrió retrasos de varias semanas en algunas de las fases, que se pudieron compensar con la realización de otras tareas. Por lo que finalmente, se pudieron cumplir los objetivos en el plazo previsto. Algunos de los principales retrasos fueron:

\begin{itemize}
    \item \textbf{Instalación y ejecución del sistema de partida}: la instalación del sistema en la Jetson Xavier NX se retrasó una semana debido a la espera de una tarjeta microSD, necesaria para flashear el \acrshort{so} y que no venía con el dispositivo. Posteriormente fue necesario flashear varias veces el dispositivo, debido a problemas para redimensionar particiones. Para mitigar el tiempo perdido en las reinstalaciones, se trabajó en preparar un Dockerfile de despliegue con todo el software y a experimentar con el sistema en el portátil personal. Dicho sistema no funcionaba de primeras, por lo que fue necesario realizar modificaciones.
    \item \textbf{Pruebas de los modelos y del sistema}: fue la tarea que más retrasó el proyecto. Los modelos en un principio reportaban bajas precisiones al implementarlos en \acrshort{gpu}. La mala exportación de los modelos o el pipeline de procesado de las imágenes eran las principales causas.

          Se intentó convertir el modelo YuNet para utilizarse en el \gls{rt} de TensorRT pero, debido al complejo procedimiento del procesado de las entradas y salidas, se mantuvo el modelo en su versión de OpenCV, ya que de la otra forma no se lograron resultados equiparables.

          Debido a la falta de espacio en el almacenamiento local de los dispositivos para almacenar los archivos de las pruebas, se tuvo que montar una red con acceso al equipo donde residían dichos archivos y ejecutarlos remotamente. Debido a las limitaciones de la red (GigabitEthernet), se ideó un mecanismo para transmitir los datos comprimidos a través de la red y descomprimirlos en el destino.

          Aunque el sistema de partida permitía operar con más de 2 cámaras, este era muy sensible a los retardos de cualquiera de estas, además de que no estaba pensado para funcionar asumiendo cierto solape entre las cámaras. Por estos motivos, se tuvo que rediseñar parte de la lógica del sistema para lograr la escalabilidad. Después se realizaron las pruebas de escalabilidad con la nueva versión del sistema.
    \item \textbf{Migración a ROS 2}: la migración se pudo realizar de manera intuitiva, salvo por los problemas generados por un paquete, cuyo coste de migración era demasiado elevado para terminarlo. Finalmente, se migró el sistema prescindiendo de dicho paquete.
    \item \textbf{Diseño e implementación del sistema extendido}: en un principio fue complicado comprender todas las bases de la arquitectura de referencia \cite{Erik}. Tras empezar la dedicación a tiempo completo al proyecto en Octubre, se pudo cumplir con el objetivo el mes siguiente.
    \item \textbf{Pruebas del reconocimiento adaptativo}: esta tarea fue según el tiempo previsto, ya que el rendimiento del sistema evolucionaba rápidamente conforme se iban añadiendo mejoras. Aunque fue necesario contar con más personas, con un total de 10 fueron suficientes (4 más que en el \gls{dataset} de las pruebas) para realizar unas pruebas razonables del sistema.
    \item \textbf{Análisis de rendimiento}: esta tarea también llevo más tiempo de lo previsto, ya que tuvieron que rehacerse las pruebas de modelos en \textbf{todos los dipositivos}, debido a modificaciones en los modelos que reducían la latencia y el cambio de versión del modelo YOLO8n a YOLO11n.

          Debido a que las pruebas del sistema no se pudieron realizar en las Jetson hasta lograr la migración a \acrshort{ros} 2, dichas pruebas se realizaron durante el transcurso de esta tarea.
\end{itemize}

Gracias a la dedicación a tiempo completo desde octubre y que se contaba con un mes de holgura en la planificación, se pudieron completar todas las tareas antes de la deposición en febrero de este proyecto.

Las pruebas de los modelos y del sistema fueron las principales causantes de los retrasos. Tuvieron que realizarse múltiples modificaciones que no se esperaban, que implicaron rehacer casi todas las pruebas en 6 dispositivos diferentes.

\section{Recursos y costes del proyecto}

En cuanto a recursos materiales podemos hablar de software y hardware. En el primer caso, todo el software utilizado en el proyecto es de uso gratuito, por lo que no supuso ningún coste.

\begin{table}[tbp]
    \footnotesize
    \setlength{\tabcolsep}{4pt} % Default value: 6pt
    \centering
    \begin{tabular}{|p{4.5cm}|p{7.5cm}|>{\centering\arraybackslash}p{1.2cm}|}
        \hline
        \textbf{Hardware}                         & \textbf{Función}                                                                                     & \textbf{Coste} \\
        \hline
        \textit{ASUS VivoBook 16X K3605ZC-N1267W} & Portátil usado para las pruebas tanto en simulación como reales de la herramienta                    & 1000€          \\
        \hline
        \textit{2 cámaras kinect}                 & Necesarias para tomar capturas del entorno en los laterales del robot                                & 100€           \\
        \hline
        \textit{cámara Orbbec Astra}              & Para tomar capturas de la visión frontal del robot                                                   & 150€           \\
        \hline
        \textit{Logitech QuickCam Orbit}          & Cámara pan/tilt para grabar la prueba de interiores                                                  & ~70€           \\
        \hline
        \textit{Router cisco linksys e3000}       & Router para proporcionar conectividad en exteriores                                                  & 130€           \\
        \hline
        \textit{SAI Riello iPlug}                 & Dota al router de energía en exteriores gracias a su batería                                         & 100€           \\
        \hline
        \textit{USB WiFi 5}                       & Proporciona una interfaz inalámbrica al robot                                                        & ~15€           \\
        \hline
        \textit{tp-link Archer T3U AC1300}        & Proporciona una interfaz inalámbrica al robot                                                        & ~19€           \\
        \hline
        \textit{Adaptador USB Ethernet YUETUOL}   & Aumenta la velocidad de transferencia de datos del robot al portátil (útil para transferir imágenes) & ~11€           \\\hline
        \textbf{Total:}                           &                                                                                                      & 1.595€         \\\hline
    \end{tabular}
    \caption{Tabla con los recursos hardware del proyecto}
    \label{tab:resources}
\end{table}

En cuanto al hardware, salvo el Summit\textunderscore XL y la estación base del GPS que fueron facilitadas por el CITIC,  puede verse un desglose de los componentes usados y su precio en la tabla \ref{tab:resources}.

\begin{table}[tbp]
    \footnotesize
    \centering

    \begin{tabular}{|c|c|c|c|}
        \hline
        \textbf{Rol}              & \textbf{Dedicación} & \textbf{Coste €/h} & \textbf{Total} \\
        \hline
        \textit{Jefe de proyecto} & ~60                 & 40                 & ~2.400€        \\
        \hline
        \textit{Analista}         & 110                 & 30                 & ~3.300€        \\
        \hline
        \textit{Desarrollador}    & 258                 & 20                 & 5.160€         \\
        \hline
        \textit{Asesor externo}   & ~~22                & 40                 & ~~880€         \\
        \hline
        \multicolumn{4}{|c|}{\textbf{Total:} 11.740€}                                         \\\hline
    \end{tabular}
    \caption{Coste de los recursos humanos del proyecto}
    \label{tab:humans}
\end{table}

Los recursos humanos requeridos para el proyecto y sus costes se exponen en la tabla \ref{tab:humans}.

