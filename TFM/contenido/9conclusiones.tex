\chapter{Conclusiones y trabajo futuro}
\label{chap:conclusiones}

\section{Conclusiones}

Cuando tenga los valores concretos de los resultados.

\section{Trabajo Futuro}
El sistema \textit{Open-World} no se ha comportado según lo previsto. Se ha observado que dicho sistema es \textbf{muy sensible} a la hora de ajustar el parámetro del umbral de Weibull. Ajustando el umbral a un valor bajo, el número de comités nuevos por entidad se dispara, ya que añade incertidumbre al sistema rápidamente (puesto que no se generan casos extremos si la entidad en cuestión se encuentra en la distribución de non-match scores). Un valor alto del umbral mitiga este problema, a costa de comprometer la precisión. Se podría estudiar la implementación de un nuevo criterio para agrupar los comités y así solventar esta problemática.

El método de reidentificación de entidades mediante la distancia Euclidiana reporta problemas cuando las cámaras cambian de posición (ejemplo: el robot se desplaza), ya que se guardan las coordenadas \acrshort{2d} de una persona que pasan a ocupar el lugar de otro individuo. La imagen de distancia de las cámaras Kinect puede otorgar una posición en el espacio \acrshort{3d} de forma que la posición de la persona desaparecida se mantiene.

Implementar el modo secuencia y la adaptación en el sistema final.

En el anexo \ref{chap:coherence} se comenta la implementación de un módulo para la resolución de conflictos cuando dos cámaras \textbf{sin solape} predicen al mismo individuo. Debido a la nula mejora en el rendimiento por los motivos comentados, sería necesario solventar los problemas del método de clasificación implementado, o incluso eliminarlo y replantear un método nuevo.