\chapter{Conclusiones y trabajo futuro}
\label{chap:conclusiones}

\lettrine{Ú}{ltimo} capítulo en el que exponen las conclusiones de los resultados y los hitos alcanzados, así como los errores cometidos y las lecciones aprendidas. También se comentan algunas vías en las que se puede extender, con desarrollo e investigación, el trabajo de este proyecto.

\section{Conclusiones}

La ejecución de los modelos de redes neuronales y del sistema en múltiples arquitecturas era uno de los objetivos principales, que se ha cumplido con hasta 6 arquitecturas distintas. Los resultados arrojados muestran las similitudes y diferencias entre las arquitecturas de dichos dispositivos. Adicionalmente, se han realizado pruebas del sistema ante una carga de trabajo creciente, cuyo objetivo era acotar el número máximo de cámaras que el sistema puede soportar, todo operado en tiempo real.

Se ha exprimido la capacidad de las \acrshort{gpu}s de todos los dispositivos del proyecto. Los resultados del sistema obtenidos en las placas NVIDIA Jetson llegan incluso a superar al sistema de referencia, que dispone de una \acrshort{cpu} superior, demostrando así la capacidad de las \acrshort{gpu}s de hacer frente a cargas de trabajo relacionadas con la visión artificial.

La integración del sistema en contenedores de Docker fue clave para el éxito del análisis, ya que de otro modo no sería posible integrar el sistema en todas las arquitecturas, teniendo en cuenta que 3 de los dispositivos (Orin Nano, AGX Orin y AGX Thor) pasaron a estar disponibles con tan solo 3 meses de antelación (desde finales de Septiembre).

Por motivos relacionados con el fin de soporte de \acrshort{ros} 1 y de poder aplicar las últimas librerías disponibles para los distintos dispositivos, se ha realizado una migración del sistema a \acrshort{ros} 2.

Por otra parte, se ha incorporado un algoritmo de aprendizaje incremental que permite detectar y reconocer desconocidos, que son posteriormente integrados al sistema, de forma que se expande su conocimiento. Se han explorado diferentes métodos de inicialización, uno partiendo de un conjunto muy reducido de datos etiquetados y el otro que obtiene los datos de manera completamente autónoma. Se cree que este último método favorece el rápido despliegue del sistema en cualquier contexto.

Se modificó el funcionamiento del sistema base, de forma que pueda procesar secuencias de frames en vez de un solo frame.

Debido al contexto de las pruebas, que implican a múltiples individuos que se entrecruzan, se han tenido que desarrollar algoritmos que mantengan el rastro de dichos individuos, con capacidad de reidentificación ante la pérdida de visión los mismos.

Todos los nuevos componentes se han sometido a análisis, que exponen el comportamiento del sistema desde múltiples puntos de vista. TODO: resultados

\section{Trabajo Futuro}

Los nuevos componentes de la arquitectura extendida (sección \ref{sec:finalsys}) se han probado de forma aislada bajo un modelo de reconocimiento facial, debido a no disponer del tiempo suficiente. La integración de dichos componentes en el sistema implicaría muchos cambios en el código, aunque no debería de afectar significativamente a la lógica del sistema base ni a la interacción con \acrshort{ros}.

El método de reconocimiento adaptativo se ha probado únicamente con recortes faciales, como en \cite{Erik, CESAR}. Se cree que dicho método puede funcionar para recortes de cuerpos.

El modo \textit{Open-World} no se ha comportado según lo previsto. Se ha observado que dicho sistema es \textbf{muy sensible} a la hora de ajustar el parámetro del umbral de Weibull. Ajustando el umbral a un valor bajo, el número de comités nuevos por entidad se dispara, ya que añade incertidumbre al sistema rápidamente (puesto que no se generan casos extremos si se activa más de un comité). Un valor alto del umbral mitiga este problema, a costa de comprometer la precisión. Se podría estudiar la implementación de un nuevo criterio para agrupar los comités y así solventar esta problemática.

Como ya se ha visto, el reconocimiento adaptativo mejora conforme se añaden nuevos usuarios al sistema. Para lograr alcanzar el techo de rendimiento del sistema, sería necesario crear un \gls{dataset} más grande, con decenas de personas, en vez de las 6 del \gls{dataset} probado.

%Se ha comprobado que el rendimiento del reconocimiento adaptativo es similar entre 1 y 10 \acrshort{svm}s por comité. Este fenómeno puede justificarse por la alta especificidad .

TODO: Más allá del uso de TensorRT para acelerar la inferencia de los modelos aún existe espacio para aprovechar al máximo las prestaciones de la \acrshort{gpu}. Se podría explorar la cuantización de los modelos a INT8, aplicar multithreading para mejorar el rendimiento de TensorRT, etc.

El método de reidentificación de entidades mediante la distancia Euclidiana reporta problemas cuando las cámaras cambian de posición (ejemplo: el robot se desplaza), ya que se guardan las coordenadas \acrshort{2d} de una persona que pasan a ocupar el lugar de otro individuo. La imagen de distancia de las cámaras Kinect puede otorgar una posición en el espacio \acrshort{3d} de forma que la posición de la persona desaparecida se mantiene.

En el anexo \ref{chap:coherence} se comenta la implementación de un módulo para la resolución de conflictos cuando dos cámaras \textbf{sin solapamiento} predicen al mismo individuo. Debido a la nula mejora en el rendimiento por los motivos comentados, sería necesario solventar los problemas del método de clasificación implementado, o incluso eliminarlo y replantear un método nuevo.