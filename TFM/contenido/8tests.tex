\chapter{Pruebas y resultados del sistema completo}
\lettrine{E}{n} este capítulo se presentan los resultados de rendimiento del método con el modelo y junto con el sistema final, variando el número de cámaras. También se explica el proceso de creación de los \textit{\glspl{dataset}} utilizados para las pruebas.

\section{Escalabilidad de cámaras}
\subsection{Diseño de las pruebas}

El sistema final es una fusión de los modelos de la sección \ref{sec:modeleval} (en la sección \ref{sec:sysarch} se detalla su arquitectura), además de un modelo de seguimiento de personas a partir de las nubes de puntos otorgadas por un sensor \textbf{\acrshort{lidar}} instalado en el robot. En el inicio de este proyecto, el sistema estaba sustentado en ROS 1 Noetic (ver sección \ref{sec:sw}), compatible con el ROS 1 Melodic instalado en el Summit\_XL. Como se comentará en la sección \ref{subsec:ROS2}, se ha realizado una migración del sistema a ROS 2. El objetivo era migrar la versión de ROS del robot en consonancia, sin embargo, debido a que la migración no es trivial y el \acrshort{lidar} instalado no era compatible con ROS 2, se han migrado todos los componentes excepto el seguimiento con el \acrshort{lidar}.

Las métricas para evaluar el sistema final son las mismas que en \cite{andrew}:
\begin{itemize}
    \item \textit{\textbf{Accuracy\_id}}: es la métrica \textit{precision} de la sección \ref{subsec:recon}.
    \item \textbf{Desconocidos (\textit{unks})}: esta métrica indica a qué porcentaje de detecciones no se les ha podido asignar una identidad. El objetivo es poder comparar si las variaciones en la métrica \textit{Accuracy\_id} se producen a cambio de dejar de identificar a más gente.
    \item \textit{\textbf{Precision\_det}} : se refiere a la métrica \textit{precision} definida en la sección \ref{subsec:det}.
    \item \textit{\textbf{Recall\_det}} : es la métrica \textit{recall} definida en la sección \ref{subsec:det}.
\end{itemize}

\subsection{Resultados}

\subsubsection{Sistema en ROS 1}

Tabla \ref{tab:scaleperf}

\begin{table}[]
    \centering
    \begin{tabular}{cc|ccc}
        \hline
        \multicolumn{2}{c|}{\multirow{2}{*}{\textit{\begin{tabular}[c]{@{}c@{}}Number of\\ cameras\end{tabular}}}} & \multicolumn{3}{c}{\textit{Intel family}}                                                                                       \\
        \multicolumn{2}{c|}{}                                                                                      & \textbf{Lab PC}                           & \textbf{Laptop \footnotemark[1]} & \textbf{Reference \cite{andrew}}                 \\ \hline
        \multicolumn{1}{c|}{\multirow{4}{*}{\textit{\begin{tabular}[c]{@{}c@{}}2\\ cameras\end{tabular}}}}         & \textit{det\_p}                           & 87.2 ± 0.8                       & 87.0 ± 0.4                       & \textbf{91.2} \\ \cline{2-2}
        \multicolumn{1}{c|}{}                                                                                      & \textit{det\_r}                           & 51.7 ± 1.3                       & 52.0 ± 1.3                       & \textbf{52.8} \\ \cline{2-2}
        \multicolumn{1}{c|}{}                                                                                      & \textit{idf1}                             & 58.2 ± 4.5                       & 59.3 ± 5.4                       & \textbf{65.0} \\ \cline{2-2}
        \multicolumn{1}{c|}{}                                                                                      & \textit{idp}                              & 83.5 ± 6.6                       & 85.4 ± 7.9                       & \textbf{93.1} \\ \hline
        \multicolumn{1}{c|}{\multirow{4}{*}{\textit{\begin{tabular}[c]{@{}c@{}}5\\ cameras\end{tabular}}}}         & \textit{det\_p}                           & 85.7 ± 0.5                       & \textbf{86.1 ± 0.7}              & -             \\ \cline{2-2}
        \multicolumn{1}{c|}{}                                                                                      & \textit{det\_r}                           & 51.1 ± 1.3                       & \textbf{51.8 ± 1.9}              & -             \\ \cline{2-2}
        \multicolumn{1}{c|}{}                                                                                      & \textit{idf1}                             & 57.2 ± 5.2                       & \textbf{57.9 ± 4.4}              & -             \\ \cline{2-2}
        \multicolumn{1}{c|}{}                                                                                      & \textit{idp}                              & 82.6 ± 7.6                       & \textbf{82.7 ± 5.5}              & -             \\ \hline
        \multicolumn{1}{c|}{\multirow{4}{*}{\textit{\begin{tabular}[c]{@{}c@{}}10\\ cameras\end{tabular}}}}        & \textit{det\_p}                           & \textbf{87.8 ± 1.1}              & 84.9 ± 1.9                       & -             \\ \cline{2-2}
        \multicolumn{1}{c|}{}                                                                                      & \textit{det\_r}                           & 46.3 ± 2.7                       & \textbf{50.8 ± 2.1}              & -             \\ \cline{2-2}
        \multicolumn{1}{c|}{}                                                                                      & \textit{idf1}                             & 52.6 ± 2.8                       & \textbf{56.6 ± 3.0}              & -             \\ \cline{2-2}
        \multicolumn{1}{c|}{}                                                                                      & \textit{idp}                              & 80.7 ± 3.0                       & \textbf{81.8 ± 3.7}              & -             \\ \hline
    \end{tabular}
    \caption{Rendimiento del sistema en la familia Intel (\acrshort{ros} 1)}
    \label{tab:scaleperf}
\end{table}
%\end{landscape}
\footnotetext[1]{Se reciben los datos del archivo \gls{rosbag} desde una fuente externa}

Tabla \ref{tab:scaleresources}

% Please add the following required packages to your document preamble:
% \usepackage{multirow}
\begin{table}[]
    \begin{tabular}{cc|llll}
        \hline
        \multicolumn{2}{c|}{\multirow{3}{*}{\textit{\begin{tabular}[c]{@{}c@{}}Number of\\ cameras\end{tabular}}}} & \multicolumn{4}{c}{\textit{Desktop}}                                                                                                                        \\
        \multicolumn{2}{c|}{}                                                                                      & \multicolumn{2}{c}{\textbf{Lab PC}}  & \multicolumn{2}{c}{\textbf{Laptop (*)}}                                                                              \\ \cline{3-6}
        \multicolumn{2}{c|}{}                                                                                      & \multicolumn{1}{l|}{\textit{CPU}}    & \multicolumn{1}{l|}{\textit{GPU}}       & \multicolumn{1}{l|}{\textit{CPU}} & \multicolumn{1}{l|}{\textit{GPU}}      \\ \hline
        \multicolumn{1}{c|}{\multirow{2}{*}{\textit{\begin{tabular}[c]{@{}c@{}}2\\ cameras\end{tabular}}}}         & \textit{\%load}                      & 6.3                                     & \multicolumn{1}{l|}{30}           &                                   &    \\ \cline{2-2}
        \multicolumn{1}{c|}{}                                                                                      & \textit{\%mem}                       & 9.27                                    & \multicolumn{1}{l|}{21}           &                                   &    \\ \hline
        \multicolumn{1}{c|}{\multirow{2}{*}{\textit{\begin{tabular}[c]{@{}c@{}}5\\ cameras\end{tabular}}}}         & \textit{\%load}                      & 13.47                                   & \multicolumn{1}{l|}{48.46}        &                                   &    \\ \cline{2-2}
        \multicolumn{1}{c|}{}                                                                                      & \textit{\%mem}                       & 16.06                                   & \multicolumn{1}{l|}{56}           &                                   &    \\ \hline
        \multicolumn{1}{c|}{\multirow{2}{*}{\textit{\begin{tabular}[c]{@{}c@{}}10\\ cameras\end{tabular}}}}        & \textit{\%load}                      & 44                                      & \multicolumn{1}{l|}{88.22}        & 44.82                             & 64 \\ \cline{2-2}
        \multicolumn{1}{c|}{}                                                                                      & \textit{\%mem}                       & 27.87                                   & \multicolumn{1}{l|}{100}          & 73.1                              & 88 \\ \hline
    \end{tabular}
\end{table}
\captionof{table}{Recursos utilizados del sistema al escalar múltiples cámaras}
\label{tab:scaleresources}

\subsubsection{Sistema en ROS 2}

\begin{landscape}
    \centering
    % Please add the following required packages to your document preamble:
    % \usepackage{multirow}
    \begin{table}[]
        \begin{tabular}{cc|ll|l}
            \hline
            \multicolumn{2}{c|}{\multirow{2}{*}{\textit{\begin{tabular}[c]{@{}c@{}}Number of\\ cameras\end{tabular}}}} & \multicolumn{2}{c|}{Desktop}                                      & \textit{Jetson embedded systems}                                                                   \\
            \multicolumn{2}{c|}{}                                                                                      & \multicolumn{1}{c}{\textbf{Laptop (*)}}                           & \textbf{Reference \textbackslash{}cite\{andrew\}} & \multicolumn{1}{c}{\textbf{Orin Nano}}         \\ \hline
            \multicolumn{1}{c|}{\multirow{4}{*}{\textit{\begin{tabular}[c]{@{}c@{}}2\\ cameras\end{tabular}}}}         & \textit{\begin{tabular}[c]{@{}c@{}}Det\\ precision\end{tabular}}  &                                                   & 0.95                                   & 0.93  \\ \cline{2-2}
            \multicolumn{1}{c|}{}                                                                                      & \textit{\begin{tabular}[c]{@{}c@{}}Det\\ recall\end{tabular}}     &                                                   & 0.347                                  & 0.356 \\ \cline{2-2}
            \multicolumn{1}{c|}{}                                                                                      & \textit{\begin{tabular}[c]{@{}c@{}}Iden\\ F1 score\end{tabular}}  &                                                   &                                        & 0.464 \\ \cline{2-2}
            \multicolumn{1}{c|}{}                                                                                      & \textit{\begin{tabular}[c]{@{}c@{}}Iden\\ precision\end{tabular}} &                                                   & 0.931                                  & 0.936 \\ \hline
            \multicolumn{1}{c|}{\multirow{4}{*}{\textit{\begin{tabular}[c]{@{}c@{}}5\\ cameras\end{tabular}}}}         & \textit{\begin{tabular}[c]{@{}c@{}}Det\\ precision\end{tabular}}  &                                                   &                                        &       \\ \cline{2-2}
            \multicolumn{1}{c|}{}                                                                                      & \textit{\begin{tabular}[c]{@{}c@{}}Det\\ recall\end{tabular}}     &                                                   &                                        &       \\ \cline{2-2}
            \multicolumn{1}{c|}{}                                                                                      & \textit{\begin{tabular}[c]{@{}c@{}}Iden\\ F1 score\end{tabular}}  &                                                   &                                        &       \\ \cline{2-2}
            \multicolumn{1}{c|}{}                                                                                      & \textit{\begin{tabular}[c]{@{}c@{}}Iden\\ precision\end{tabular}} &                                                   &                                        &       \\ \hline
            \multicolumn{1}{c|}{\multirow{4}{*}{\textit{\begin{tabular}[c]{@{}c@{}}10\\ cameras\end{tabular}}}}        & \textit{\begin{tabular}[c]{@{}c@{}}Det\\ precision\end{tabular}}  &                                                   &                                        &       \\ \cline{2-2}
            \multicolumn{1}{c|}{}                                                                                      & \textit{\begin{tabular}[c]{@{}c@{}}Det\\ recall\end{tabular}}     &                                                   &                                        &       \\ \cline{2-2}
            \multicolumn{1}{c|}{}                                                                                      & \textit{\begin{tabular}[c]{@{}c@{}}Iden\\ F1 score\end{tabular}}  &                                                   &                                        &       \\ \cline{2-2}
            \multicolumn{1}{c|}{}                                                                                      & \textit{\begin{tabular}[c]{@{}c@{}}Iden\\ precision\end{tabular}} &                                                   &                                        &       \\ \hline
        \end{tabular}
    \end{table}
    \captionof{table}{Rendimiento del sistema en ROS 2}
    \label{tab:scaleros2}
\end{landscape}

Tabla \ref{tab:scaleros2}

\section{Tamaño del universo}
El tamaño del universo, es decir, del número de personas presentes en el sistema, es un factor crítico, ya que el sistema implementado se nutre de los datos de los individuos. Por un lado, tener más individuos se traduce en más comités, por lo tanto se dispone de más puntos para definir la distribución de Weibull, lo que refuerza las predicciones. Por otra parte, se dispone de un mayor número de muestras negativas, lo que permite definir mejor las fronteras de las \acrshort{svm} y así mejorar la capacidad de predicción.

(Pág 81, TODO:) La calidad de la función RDF viene determinada por el número de puntos que componen la función de Weibull, cuantos más puntos, mayor definición y por tanto reconocimientos más precisos. Con un universo de 20 individuos, sólo 5 puntos componen la función de Weibull (de las 20 entidades sólo 10 se registran en el sistema, a modo de probar el open-set, y de esos 10, la mitad definen la función), lo que lleva a un comportamiento más impreciso que con un universo mayor. El sistema inicial cuenta con un universo de \textbf{6 individuos}, lo que claramente \textbf{es insuficiente} para realizar una diferenciación precisa de los IoI respecto a lo desconocido. Para este proyecto, ha sido necesario ampliar el universo a \textbf{20 personas}, algunas procedentes del CITIC, que se han prestado voluntariamente, otras extraídas del YoutubeFaces \textit{\gls{dataset}}.

\subsection{Resultados}

\section{Ajuste de parámetros}

En esta sección se realiza un estudio acerca de la configuración óptima de parámetros en términos de rendimiento del sistema. A continuación se exponen los parámetros utilizados y su función, junto con el rango de valores para la prueba.

TODO: En la fase de inicialización de los modelos (o fase de entrenamiento), el sistema guarda la información estadística que representa a cierto modelo. Durante el funcionamiento del sistema, puede ocurrir que los datos que recibe de un individuo registrado varíen respecto a la información de entrenamiento, lo que se conoce como \textbf{concept drift}. Se debe de corregir el concept drift para evitar una pérdida en la precisión de reconocimiento.

TODO: actualización del modelo, añadir nuevas entidades puede corromper la información de las entidades iniciales (catastrophic forgetting).

TODO: él método tiene una precondición, \textbf{se necesitan de al menos 5 IoI para crear la distribución de Weibull}.

TODO: los criterios que conforman el módulo de limitación tienen en cuenta el signo de la puntuación de forma exclusiva, lo que puede no funcionar siempre, sobretodo si se dispone de pocos frames que definan al individuo.

\begin{itemize}
    \item \textbf{Umbral de Weibull (Tw)}: valor que determina si una entidad es desconocida o no. El rango probado es (0.05 0.1 0.25)
    \item \textbf{Percentil (percent)}: valor del percentil aplicado en la combinación de las puntuaciones generadas por las \acrshort{svm}. El rango probado es (0.25 0.5 0.75)
    \item \textbf{Número de \acrshort{svm} (max\_svm)}: valor que representa al tamaño máximo de un comité. El rango probado es (1 3 7 10)
    \item \item \textbf{Número de negativos (nneg)}: tamaño del conjunto de muestras negativas. El rango probado es (10 50)
    \item \textbf{Tamaño de la plantilla}: establece el número de frames positivos que define a un comité. Se va a partir siempre de un tamaño de 5 frames, el mínimo necesario acorde a \cite{Erik} para definir un nuevo individuo.
    \item \textbf{Tamaño de secuencia}: número de frames que forman una secuencia de video. Se ha fijado el valor en 15 frames, que equivalen a 1,5 segundos de video con cámaras funcionando a 10 Hz.
    \item \textbf{Solapamiento}: establece el salto en el número de frames entre secuencias. Se ha fijado el valor en 5, dando lugar a más secuencias para analizar al coincidir frames entre las mismas.
\end{itemize}

A continuación se exponen los resultados obtenidos para cada combinación en \textit{Open-Set} y \textit{Open-World} como una media de 5 ejecuciones. Se parte de \textbf{10 individuos registrados de antemano}, 6 de ellos presentes en el video de prueba. Se inicializa cada comité con 5 frames representativos.

\subsection{Modo \textit{Open-Set}}

\begin{table}[]
    \centering
    \begin{tabular}{c|c|c|c|c|c}
        \hline
        \textit{}          & \textit{Best result} & \textit{2º best result} & \textit{3º best result} & \textit{4º best result} & \textit{5º best result} \\ \hline
        \textit{Precision} & 77.8 ± 2.6           & 79.1 ± 1.7              & 76.0 ± 2.7              & 79.6 ± 2.4              & \textbf{81.9 ± 3.3}     \\
        \textit{Recall}    & 92.9 ± 1.4           & 89.7 ± 1.4              & \textbf{93.5 ± 1.1}     & 88.5 ± 1.3              & 85.1 ± 3.0              \\
        \textit{F1 score}  & \textbf{84.6 ± 1.0}  & 84.1 ± 1.4              & 83.8 ± 1.4              & 83.8 ± 1.6              & 83.4 ± 2.2              \\ \hline
        \textit{Tw}        & 0.25                 & 0.1                     & 0.25                    & 0.1                     & 0.05                    \\
        \textit{Percent}   & 25                   & 25                      & 25                      & 25                      & 25                      \\
        \textit{Max\_svm}  & 10                   & 7                       & 7                       & 10                      & 10                      \\
        \textit{Nneg}      & 50                   & 50                      & 50                      & 50                      & 50                      \\ \hline
    \end{tabular}
    \caption{Mejores resultados del ajuste de parámetros en \textit{Open-Set}.}
    \label{tab:osparams}
\end{table}

En la tabla \ref{tab:osparams} se muestra el \textit{F1 score} para 10 y 50 muestras que conforman el conjunto de negativos de entrenamiento (ratios 1:1 y 1:5 si el tamaño del template es 10). Claramente los resultados mejoran al escoger 50 negativos, esto se debe a que en 10 frames cabe un conjunto reducido de entidades a diferencia de 50 frames, sobretodo si se realiza una selección aleatoria como ocurre en este caso.

Tw: Si el valor del umbral de Weibull es bajo, la precisión se mantiene en valores altos, sacrificando el recall, mientras que un valor alto en dicho umbral se traduce en un elevado recall a costa de la precisión. Como ya se ha explicado, el umbral de Weibull representa la probabilidad máxima a la que el mejor score pertenece a la distribución de scores no coincidentes, cuanto más bajo sea el valor, más extrema debe ser la coincidencia para ser reconocida (la precisión aumenta), por lo que la función es más selectiva en el reconocimiento (devuelve un mayor número de desconocidos, reduciendo así el recall). Un valor alto representa justo lo contrario. TODO: parece que un TW alto es mejor siempre, pero tengo que poner los resultados de una prueba con un desconocido de verdad, así muestro el verdadero problema de elevarlo.

Percentiles: Al fusionar los scores de todas las SVM en todos los frames, se obtiene un único score acordado por la mayoría mediante la mediana, siendo esta equivalente al percentil 50. Al variar el percentil, el número de SVM participantes también varía. Un percentil de 100 implica obtener el mayor score, que se corresponde con la peor coincidencia (por ende, el valor de recall es paupérrimo, puesto que se necesita del consenso de todas las SVM del comité para devolver un match). Por otro lado, un percentil de 0 implica recuperar el score más bajo de entre todas las SVM en todos los frames. En este caso también podría darse un bajo recall ya que si por ejemplo se añade una SVM de otra entidad, dicha SVM devolvería un score negativo, por lo que Weibull etiquetaría el reconocimiento como desconocido al haber más de un comité activado (que devuelve un score bajo).

\subsection{Modo \textit{Open-World}}

% Please add the following required packages to your document preamble:
% \usepackage[table,xcdraw]{xcolor}
% Beamer presentation requires \usepackage{colortbl} instead of \usepackage[table,xcdraw]{xcolor}
\begin{table}[]
    \begin{tabular}{l|l|l|l|l}
        \hline
        \textit{}                                                              & \textit{Best result} & \textit{2º best result}     & \textit{3º best result}     & \textit{4º best result}     \\ \hline
        \textit{F1 score}                                                      & 78.1                 & {\color[HTML]{000000} 66.3} & {\color[HTML]{000000} 64.8} & {\color[HTML]{000000} 64.6} \\
        \textit{\begin{tabular}[c]{@{}l@{}}Number of\\ ensembles\end{tabular}} & 20                   & 33                          & 27                          & 56                          \\
        \textit{Tw}                                                            &                      &                             &                             &                             \\
        \textit{Percent}                                                       &                      &                             &                             &                             \\
        \textit{Max\_svm}                                                      &                      &                             &                             &                             \\
        \textit{Nneg}                                                          &                      &                             &                             &                             \\ \hline
    \end{tabular}
    \captionof{table}{Mejores resultados del ajuste de parámetros en \textit{Open-World}.}
    \label{tab:owparams}
\end{table}

La tabla \ref{tab:owparams}.

\subsection{Umbral de actualización?}

\textbf{Umbral de actualización}: valor que se compara frente a la fusión de scores de una secuencia. El rango probado es (0.1 0.3)



Uno de los objetivos del sistema es obtener la mayor diversidad intra-comité y la mayor especificidad inter-comité. Un \textbf{percentil alto} encajaría si las SVM son \textbf{específicas} entre sí (la mayoría de las SVM devuelven match). En cambio, un \textbf{bajo percentil} funcionaría con unas SVM \textbf{más diversas} (se espera que la minoría de las SVM den un resultado coincidente).

Tamaño del comité: En relación con el anterior punto, el número de SVM influye en la decisión de que percentil tomar. Si se utilizan 3 SVM por comité y la mediana, al menos 2 SVM deben de devolver un score bajo para dar un resultado coincidente. Por otro lado, si se utilizan 10 SVM y un percentil de 30, sólo 3 de las 10 SVM necesitan estar de acuerdo para devolver un match. En los resultados se muestra como un comité de 3 SVM y TW=0.05 destaca respecto a un comité de 1 SVM y TW=0.5. Los resultados utilizando 10 SVM por comité no llegan a mejorar notablemente el valor de 3 SVM, esto es de gran relevancia, puesto que se reduce el número de iteraciones de cada reconocimiento de 10*n a 3*n, siendo n el número de comités que existen y asumiendo que todos los comités han llegado al límite de SVM. Con estos resultados se ha demostrado que implementar varias SVM por comité mejora el recall manteniendo la precisión. 1 sóla SVM por comité es efectiva si dicha SVM es representativa del individuo, en caso contrario puede devolver respuestas coincidentes acerca de una entidad distinta del comité, generando ruido que perjudica a las predicciones de Weibull. Un mayor número de SVM combinado con la mediana devuelve un resultado conforme dicta una mayoría que omite los scores de SVM intrusas (de entidades distintas) y permite expulsarlas mediante los criterios ya comentados en la sección \ref{seq:limmod}.

Tamaño de la plantilla: El tamaño de plantilla es igual al tamaño del conjunto de positivos para una entidad, cuantas más muestras conformen dicho conjunto, mejores predicciones realizará la SVM asociada. En la tabla (TODO) se puede ver que el mejor resultado ha sido con un tamaño de plantilla de 10 frames. Los resultados con 5 frames son ligeramente peores (79.7 frente a 76.5), lo que demuestra la robustez del sistema ante dicho número escaso de muestras para un individuo.

Tamaño de secuencia: en la tabla (TODO) se muestran 3 valores diferentes de este parámetro. Como es de esperar, los mejores resultados coinciden con el mayor tamaño (25), puesto que se dispone de una mayor cantidad de información que apoye al reconocimiento. Los resultados con una secuencia de 15 frames caen levemente, un menor tamaño de secuencia hace que el sistema funcione a una mayor frecuencia (devuelva más reconocimientos en el mismo tiempo), concretamente a cada 1,5 segundos con dicho tamaño de secuencia asumiendo que las cámaras funcionan a 10 Hz (capturan una imagen cada 100 ms).

Solapamiento: el solapamiento entre secuencias puede causar que las SVM sean más específicas, ya que se da lugar a un mayor número de SVM parecidas entre sí (con un solapamiento de 5 y un tamaño de secuencia de 25, se comparten 20 frames entre secuencias y por ende entre las SVM). Con un número reducido o nulo de frames compartidos, el comité tiende a generalizar mejor, lo que es un factor crítico para la precisión y el recall.

Según los resultados arrojados (TODO), ninguno de los solapamientos utilizados indica una variación significativa en los resultados.

Criterio de selección de frames para nuevas SVM: también es muy importante la selección de las muestras que conforman las SVM, en la tabla (TODO) se muestran 2 criterios para la selección, uno consiste en seleccionar los frames con los scores más cercanos al cero, mientras que el otro es una selección aleatoria. El primer criterio demuestra un aumento en el F1 score, dado que los frames cercanos al cero son los más complicados de reconocer, de esta forma la SVM amplía sus fronteras respecto a lo que ya reconoce. A la hora de crear una SVM de un nuevo comité en el modo open-world, no queda otra alternativa que realizar una selección aleatoria.

Criterio de favorabilidad: en la tabla (TODO) se muestra el efecto de dicho criterio a la hora de eliminar las SVM respecto a un criterio de eliminación aleatorio. Este ...

En el modo \textit{Open-World}, el sistema crea un nuevo comité cuando se reconoce a una nueva entidad en el sistema. Sin embargo, debido a las predicciones incorrectas del método, puede ocurrir que se creen nuevos comités para un individuo ya registrado. Estos nuevos comités pueden desplazar a los comités originales adueñándose de su identidad.

TODO: Para concluir, cuando el sistema no es capaz de reconocer a un individuo con alta confianza y, en cambio, lo denota como desconocido, el mecanismo se vuelve \textbf{contraproducente}. Si una entidad ya conocida se reconoce como desconocida, se crea un nuevo comité de la misma persona, lo que incrementa la duda en el reconocimiento, ya que la distribución de weibull contendrá puntos que representen al propio individuo, indicando que el score de coincidencia no es un caso extremo cuando si debería serlo. La solución más rápida y efectiva es subir el valor de threshold de la probabilidad devuelta por la función de Weibull, aún así, es inevitable que se generen nuevos comités para un mismo individuo, contaminando el sistema rápidamente.

El número de comités redundantes creados varía conforme al umbral de Weibull. Cuanto menor sea el valor, más comités redundantes se crearán debido a que la incertidumbre del método aumenta.

\section{Pruebas de inicialización del sistema}

\begin{landscape}
    \begin{table}[]
        \begin{tabular}{l|llll|llll|llll}
            \hline
            \textit{}                                                              & \multicolumn{4}{c|}{\textit{Supervised}} & \multicolumn{4}{c|}{\textit{Semisupervised}} & \multicolumn{4}{c}{\textit{Non supervised}}                                                                                                                                                                                                                                                                                                                        \\ \cline{2-13}
                                                                                   & \multicolumn{2}{c|}{\textit{Open-Set}}   & \multicolumn{2}{c|}{\textit{Open-World}}     & \multicolumn{2}{c|}{\textit{Open-Set}}      & \multicolumn{2}{c|}{\textit{Open-World}}              & \multicolumn{2}{c|}{\textit{Open-Set}} & \multicolumn{2}{c}{\textit{Open-World}}                                                                                                                                                                             \\ \hline
            \textit{Tw}                                                            & \multicolumn{1}{c|}{0.25}                & \multicolumn{1}{c|}{0.1}                     & \multicolumn{1}{c|}{0.25}                   & \multicolumn{1}{c|}{0.1}                              & \multicolumn{1}{c|}{0.25}              & \multicolumn{1}{c|}{0.1}                & \multicolumn{1}{c|}{0.25} & \multicolumn{1}{c|}{0.1} & \multicolumn{1}{c|}{0.25} & \multicolumn{1}{c|}{0.1}         & \multicolumn{1}{c|}{0.25} & \multicolumn{1}{c}{0.1} \\ \hline
            \textit{Precision}                                                     & 79.1 ± 1.15                              & \multicolumn{1}{l|}{79.6 ± 3.2}              & 66 ± 3.95                                   & \begin{tabular}[c]{@{}l@{}}55.2 ±\\ 3.39\end{tabular} & 68.3 ± 1.89                            & \multicolumn{1}{l|}{72.9 ± 2.84}        & 45.4 ± 1.96               & 40.4 ± 6.84              & 48.5 ± 8.9                & \multicolumn{1}{l|}{64.6 ± 9.77} & 26.8 ± 6.66               & 30.6 ± 6.62             \\
            \textit{Recall}                                                        & 95.1 ± 1.34                              & \multicolumn{1}{l|}{89.6 ±1.72}              & 95.8 ± 0.82                                 & 90 ± 1.89                                             & 83.9 ± 2.66                            & \multicolumn{1}{l|}{78.2 ± 1.02}        & 93.2 ± 1.2                & 87.0 ± 3.51              & 80.1 ± 2.3                & \multicolumn{1}{l|}{75.5± 3.51}  & 88.6 ± 2.63               & 79.1 ± 10.22            \\
            \textit{F1\_score}                                                     & 86.3 ± 0.92                              & \multicolumn{1}{l|}{84.2 ± 2.22}             & 78.1 ± 2.9                                  & 68.4 ± 3.09                                           & 75.2 ± 0.7                             & \multicolumn{1}{l|}{75.5 ± 1.58}        & 61.1 ± 1.71               & 55 ± 7.01                & 60.1 ± 7.04               & \multicolumn{1}{l|}{69.3 ± 7.45} & 40.8 ± 7.78               & 44 ± 8.35               \\
            \textit{\begin{tabular}[c]{@{}l@{}}Number of\\ ensembles\end{tabular}} &                                          & \multicolumn{1}{l|}{}                        & 19.4 ± 2.24                                 & 32 ± 3                                                &                                        & \multicolumn{1}{l|}{}                   & 25 ± 2                    & 33 ± 4                   &                           & \multicolumn{1}{l|}{}            & 24 ± 1                    & 45 ± 17                 \\ \hline
        \end{tabular}
    \end{table}
    \captionof{table}{Resultados según la inicialización del sistema}
    \label{tab:inittab}
\end{landscape}

La selección de los frames que conforman un nuevo comité es un factor crítico para el correcto funcionamiento del sistema. En la tabla \ref{tab:inittab} se comparan 2 técnicas de inicialización, una \textbf{supervisada} (los frames se escogen manualmente) y la otra \textbf{no supervisada} (se escoge un momento del video en el que aparece un mínimo de personas simultáneamente y se aplica el tracking para agrupar sus frames y finalmente se inicializan los comités). En el modo no supervisado se debe de escoger un momento en el que aparezcan todos los individuos del video etiquetado, como dicha situación no se da en el video, se ejecuta el sistema para agregar a los desconocidos bajo las etiquetas de los voluntarios restantes.