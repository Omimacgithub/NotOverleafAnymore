\chapter{Introducción}
\label{chap:introduccion}
\lettrine{E}{n} este capítulo se expone la motivación y los objetivos de este proyecto. Adicionalmente, se comenta la estructura por capítulos de esta memoria y brevemente su contenido.

\section{Motivación}
Con la aparición del Deep Learning y las redes neuronales profundas, se han logrado notables avances en términos de precisión en áreas como la visión artificial. Sin embargo, debido a sus elevados requerimientos computacionales, los modelos de Deep Learning suelen ejecutarse en entornos en la nube \cite{MITTAL2019428}, lo que no es una solución viable en aplicaciones con ejecución en tiempo real debido a la latencia en las respuestas. En este contexto, surge el concepto de TinyML \cite{OLIVEIRA2024101153}, que consiste en la implementación de modelos compactos de forma eficiente en sistemas embebidos como microcontroladores (Low-end TinyML, ej: Arduino) u ordenadores de placa única (High-end TinyML, ej: NVIDIA Jetson), más potentes que los microcontroladores, pero menos económicos.

Dado que el caso de uso de este proyecto es el reconocimiento de personas en tiempo real a partir de los datos de múltiples cámaras, resultaría muy beneficioso aplicar la \textbf{computación en el borde o Edge Computing}. Este paradigma consiste en procesar los datos desde su fuente de origen o desde un dispositivo muy cercano, de esta forma, se reduce la latencia de transmisión y se reparte el volumen de datos entre múltiples dispositivos de bajo consumo \cite{EDCOM}. Este proyecto propone la integración de un sistema compuesto por varias redes neuronales \cite{andrew} en varios dispositivos Jetson y su análisis en un entorno de operación real con múltiples cámaras. El sistema emplea el framework ROS \cite{ROS}, estándar de facto en robótica, que permite la ejecución distribuida de procesos en diferentes máquinas y posee una amplia biblioteca de sensores y algoritmos.

Durante el desarrollo del sistema base, se contempló implementar la capacidad de detectar cuando un individuo es un desconocido y en ese caso agregarlo al sistema (\textit{Open-World})\cite{andrew}. Esta funcionalida es muy útil para implementar un robot móvil guía que asista a los usuarios proporcionando un trato cercano en todo momento (ejemplo: saludarlos por su nombre). Este proyecto propone la implementación de un método de aprendizaje máquina para detectar e incluir nuevos individuos basado en las ideas de una tesis doctoral \cite{Erik}. El método requiere de un conjunto reducido de datos para su inicialización (por ejemplo, los datos biométricos de 5 personas). Dado que la recopilación de imágenes faciales está fuertemente regulada por la ley de protección de datos, el uso de este enfoque resulta especialmente relevante para el presente proyecto \cite{Erik}.

\section{Objetivos}

El objetivo global del proyecto es la implementación y funcionamiento de un método de aprendizaje máquina bajo un sistema para el reconocimiento y seguimiento de personas en tiempo real en sistemas embebidos de altas prestaciones de la familia NVIDIA Jetson y dispositivos similares. En concreto, el proyecto pretende abordar los siguientes objetivos:

O1. Extender la funcionalidad de un sistema de reconocimiento y seguimiento de personas para robots móviles con el fin de detectar desconocidos e ingresarlos en el sistema.

O2. La ejecución eficiente del programa en sistemas embebidos y otros dispositivos.

TODO: comparan la AGX Xavier (+1000 leuros) con una GTX 1050 y los resultados son muy similares \cite{murthy2020investigations}

\section{Estructura de la memoria}

Esta memoria se divide en capítulos, cada uno con una explicación detallada del trabajo realizado. Se expone a continuación la estructura de esta memoria, explicando brevemente su contenido:

\begin{enumerate}
    \item Introducción: presente capítulo en el que se exponen los motivos de realización de este proyecto, los objetivos planteados y la estructura de la memoria resultante.
    \item Fundamentos teóricos y tecnológicos: capítulo en el que se presentan los conceptos clave del proyecto, así como las herramientas, tanto hardware como software utilizadas.
    \item Gestión del proyecto: TODO:
\end{enumerate}