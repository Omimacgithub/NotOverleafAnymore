\chapter{Arquitectura del sistema propuesto}
\label{chap:sysarch}
\lettrine{E}{n} este capítulo se expone la arquitectura planteada, así como los detalles de su implementación.

Este proyecto parte de un sistema que detecta y reconoce personas a partir de la información de diferentes sensores, de forma que se obtienen reconocimientos robustos y persistentes en el tiempo \cite{andrew}.

El funcionamiento se muestra en la figura \ref{fig:finalsys}. Se recibe la información de una o múltiples cámaras, que son procesadas por un nodo que devuelve una lista de predicciones de los individuos en escena(nodo cámara). Cada detección se etiqueta con su posición \acrshort{3d} a partir de la imagen de distancia de la cámara. Los resultados obtenidos de la anterior fase se contrastan con las detecciones realizadas por un modelo que procesa una \gls{pcl} otorgada por un sensor \acrshort{lidar} (nodo \acrshort{lidar}). Finalmente, las salidas de ambos nodos se fusionan para reforzar las predicciones (nodo integración de sensores).

Dentro del nodo cámara se agrupan los frames en secuencias (o videos) con el fin de explotar la coherencia espacio-temporal. Los modelos \acrshort{cnn} de detección reciben dicha secuencia de frames y devuelven los recortes (o \glspl{bbox}) con las detecciones, que son agrupados por individuo aplicando un método de seguimiento (tracking) en el caso de los rostros. Los recortes de caras y cuerpos se transforman en vectores de características a partir de los respectivos modelos de reconocimiento, que a su vez sirven de entrada para el módulo adaptativo, que toma la decisión de reconocimiento y actualiza la base de datos de personas registrada.

El módulo adaptativo es el encargado de otorgar el reconocimiento \textit{Open-Set} y/o \textit{Open-World}. El módulo evalúa si la persona detectada en la secuencia es un desconocido o pertenece a la base de datos. En ambos casos, el sistema se actualiza con la nueva información en forma de un nuevo registro en el sistema si es un desconocido o modificando el registro existente ante los cambios de la entidad (lo que se conoce como \textit{\textbf{concept-drift}}).

Toda la arquitectura vista se ejecuta en el \textit{framework} \acrshort{ros} \cite{ROS}. \acrshort{ros} se encarga de crear los procesos para cada nodo, comprobar su estado, regular la frecuencia a la que trabajan, crear la red en la que dichos procesos intercambian mensajes, entre otros muchos detalles que resultan transparentes para el programador.

\begin{figure}
    \centering
    \includegraphics[width=1\linewidth]{imagenes/FINALSYS.jpg}
    \caption{Arquitectura general del sistema}
    \label{fig:finalsys}
\end{figure}

\section{Procesamiento de video}
\label{sec:video}
El sistema de partida trabaja a nivel de frame \cite{andrew}, es decir, devuelve reconocimientos de los individuos presentes en una sola imagen. Esta aproximación permite trabajar a altas frecuencias, sin embargo, las predicciones dependen enteramente de la calidad del frame (ejemplo: nivel de borrosidad). En este proyecto se ha optado por trabajar con \textbf{secuencias de frames (o video)}, de esta forma, se devuelve un reconocimiento más robusto basado en el contexto espacio-temporal. Simplemente consiste en procesar un conjunto de frames en un intervalo de tiempo, dicho intervalo de tiempo es ajustable según las necesidades del operador (ejemplo: intervalo de 1 segundo). De los frames del intervalo, se extrae la información de los individuos presentes a partir de un modelo de detección de caras (en este caso YuNet) y/o de cuerpos (YOLO) y se agrupan por cada individuo a lo largo del intervalo, de modo que los frames que no contienen sujetos \textbf{se descartan}.

\subsubsection{Seguimiento de entidades}
En casos como los del \textit{\gls{dataset}} FACE COX \cite{cox}, donde en los videos siempre aparece una sola persona, la agrupación de las entidades es trivial. Sin embargo, en un video donde aparecen múltiples individuos que se entrecruzan, es necesario adoptar un método para seguir el rastro de cada individuo entre frames.

El \textbf{método húngaro} es un algoritmo que resuelve el problema de la asignación óptima. Dada una matriz de costes en el que cada fila establece una correlación con cada columna, dicho algoritmo buscará la \textbf{asignación óptima}, es decir, de menor coste entre elementos, en este caso, dos frames adyacentes \cite{Hungarian}. La figura \ref{fig:Seq} muestra un ejemplo de funcionamiento del método, que consiste en lo siguiente. Se aplica el modelo de detección pertinente y se generan las \glspl{bbox} del \textbf{frame actual y del siguiente}. Posteriormente, se genera la matriz de costes, donde las \glspl{bbox} del frame actual se encuentran en las filas y las \glspl{bbox} del frame posterior en las columnas. Para cada par de \glspl{bbox} de la matriz, se calcula el \textbf{\acrfull{iou}} \cite{iou}, esta métrica devuelve el porcentaje de solapamiento y similitud entre \glspl{bbox}, de forma que recortes de diferente tamaño (cuando más cerca esté el sujeto de la cámara, más grande será el recorte) den un valor bajo al solaparse (ejemplo: personas que coinciden en la imagen en diferentes distancias). Dadas dos \glspl{bbox} A y B, el \acrshort{iou} se calcula como sigue \ref{eq:iou}. Ya que el método húngaro busca las relaciones de menor coste, es necesario restar el valor obtenido por uno. Tras repetir el método en todos los frames de la secuencia, se agrupa la información de cada persona según el rastro generado por el algoritmo.

\begin{figure}
    \centering
    \includegraphics[width=0.5\linewidth]{imagenes/iou.png}
    \caption{Intersection over Union, imagen extraída de \cite{iou}}
    \label{fig:iou}
\end{figure}

\[ IoU = \frac{A \cap B}{A \cup B} \]
\captionof{figure}{Cálculo del \acrshort{iou}}
\label{eq:iou}

\subsubsection{Reidentificación de entidades}
Es posible que durante la detección, las \glspl{bbox} de una misma persona en frames consecutivos no se solapen debido a la velocidad de movimiento de la propia persona o a un movimiento de la cámara, lo que imposibilita la aplicación del método húngaro. Otro problema es el no seguimiento de la persona cuando esta se encuentra totalmente ocluida (ejemplo: se cruza un individuo justo delante) y reaparece o si simplemente la persona gira su cara fuera de la visión de la cámara y vuelve a detectarse después, en estos casos se crearía una nueva entidad para el mismo individuo, lo que no es un comportamiento deseable.

El \textbf{filtro de Kalman} predice la próxima posición del individuo a partir de la probabilidad Gausiana, de tal forma que puede mantenerse el rastro cuando la persona desaparece por unos frames. A pesar de ser una técnica efectiva, el filtro necesita de un mínimo de frames para converger cuando una nueva persona aparece \cite{MangoYOLO}, lo que no es beneficioso para intervalos de secuencia cortos. Finalmente se ha optado por un método más sencillo basado en la \textbf{distancia euclidiana}. En el caso de que haya un movimiento veloz del individuo o de la cámara, se calcula la distancia entre el \textbf{centro} de la \gls{bbox} actual con la \glspl{bbox} posterior que no tiene solape, si la distancia calculada es \textbf{inferior a un umbral}, \textbf{se valida la asociación}. En el caso de perder el rastro del individuo, se guarda la posición de la última aparición del mismo y se calcula la distancia con las \glspl{bbox} sin asociación en futuros frames, si la distancia no es inferior al umbral en un máximo de frames (prefijado por el operador), \textbf{se abandona el rastreo}, en caso contrario se restablece. El valor del umbral se fija de antemano \textbf{y se ajusta automáticamente en función de la resolución} de la cámara.

En la figura \ref{fig:eucls} se expone un ejemplo real, en el primer frame (figura \ref{fig:eucls}), se muestran dos entidades etiquetadas con un identificador (0 y 1) y como para la entidad 0 sus \glspl{bbox} no se solapan, en este caso la distancia euclidiana es capaz de mantener la identificación, también se muestra como la entidad 1 está a punto de ser ocluida, al no haber \glspl{bbox} en frames posteriores asociables a la entidad 1, la posición del sujeto se guarda. En el segundo frame (figura \ref{fig:eucls}), la entidad 1 se encuentra totalmente ocluida por la entidad 0, como en el instante posterior a dicho frame existe una \gls{bbox} sin ninguna asociación, se calcula la distancia euclidiana entre dicha \gls{bbox} respecto a la última detectada de la entidad 1. En el último frame (figura \ref{fig:eucls}) se muestra la entidad 1 reasignada.

\begin{figure}[hp!]
    \centering
    \begin{subfigure}[c]{0.2\textwidth}
        \includegraphics[width=\textwidth]{imagenes/eucl0.png}
    \end{subfigure}
    \begin{subfigure}[c]{0.2\textwidth}
        \includegraphics[width=\textwidth]{imagenes/eucl1.png}
    \end{subfigure}
    \begin{subfigure}[c]{0.2\textwidth}
        \includegraphics[width=\textwidth]{imagenes/eucl2.png}
    \end{subfigure}
    \caption{Reidentificación mediante la distancia euclidiana}
    \label{fig:eucls}
\end{figure}

\[ \Delta r_{euclid} = \sqrt{\Delta x^{2} + \Delta y^{2}} \]
\captionof{figure}{Cálculo de la distancia euclidiana en un sistema de coordenadas \acrshort{2d}, extraído de \cite{Hungarian}}
\label{eq:eucl}

\begin{figure}
    \centering
    \includegraphics[width=1\linewidth]{imagenes/Seq.jpg}
    \caption{Funcionamiento del método húngaro para el seguimiento de personas, como salida se obtiene una lista de recortes agrupados por entidad}
    \label{fig:Seq}
\end{figure}

\section{Módulo de adaptación}

Este módulo recibe la secuencia TODO: tiene a ArcFace contenido.

En la figura \ref{fig:FACESYS} se muestra la arquitectura del sistema con adaptación a los cambios. A continuación, se detallan todos sus componentes.

\begin{figure}
    \centering
    \includegraphics[width=1\linewidth]{imagenes/FACESYS.jpg}
    \caption{Diseño del sistema adaptativo}
    \label{fig:FACESYS}
\end{figure}

La mayoría de los componentes se han implementado basándose en \cite{Erik, CESAR}.

(TODO: Comités) Múltiples \acrshort{svm} simples (ejemplo: lineales) como conjunto (\textbf{comité}) \textbf{generalizan mejor} que una única \acrshort{svm} compleja (ejemplo: sigmoide) \cite{malisiewicz2011ensemble}. Otras ventajas de los comités son la incorporación de nueva información sin tener que reentrenar las \acrshort{svm}, lo que recorta una cantidad de tiempo significativa durante la operación del sistema, y la flexibilidad, que permite eliminar parte de la información que no es relevante borrando la \acrshort{svm} redundante. Para cada IoI se asigna un comité de \acrshort{svm}, cada \acrshort{svm} del comité se entrena con n muestras de la propia entidad (1 en la idea original \cite{malisiewicz2011ensemble}, 5 adaptado a este proyecto) y \textbf{m-n} muestras negativas (fotogramas de otras entidades), siendo \textbf{m} el número de IoI registrados en el sistema.