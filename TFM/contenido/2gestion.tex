\chapter{Gestión del proyecto}
\lettrine{E}{n} este capítulo se ahondará en la planificación del proyecto, su seguimiento y del plan de gestión de riesgos utilizado.

\section{Requisitos}
Los requisitos definen las pautas que el sistema desarrollado debe cumplir para alcanzar los objetivos establecidos. Los requisitos pueden dividirse en 2 tipos:

\subsection{Funcionales}
Definen las funcionalidades que el sistema debe de ofrecer. Para este proyecto, se busca ampliar el sistema base para que cumpla con los siguientes requisitos:

\begin{itemize}
    \item 
    \item Incorporación de nuevas cámaras: debe ser sencillo añadir nuevas cámaras en adición a las 2 iniciales.
    \item Pasar de reconocimientos frame a frame a ventanas de frames, aprovechando así la coherencia espacio-temporal.
    \item Funcionalidad Open-Set y Open-World: el sistema debe ser capaz de reconocer y agregar nuevas personas a su base de datos.
\end{itemize}
\subsection{No funcionales}
\begin{itemize}
    \item Que el sistema evolucione adecuadamente (ejemplo: evitar caídas drásticas en la precisión) a lo largo del tiempo.
    \item Que la instalación del sistema sea rápida y prácticamente automática.
    \item Operación en tiempo real: sujeto a la frecuencia del componente más "lento". En el caso de este proyecto, las cámaras RGBD otorgan una imagen RGB cada 100 ms (\textbf{10 Hz}).
    \item Portabilidad: la ejecución del sistema no debe de depender exclusivamente de la arquitectura o de las diferentes versiones de librerías.
    \item Escalabilidad: debe permitir incluir nuevos componentes (ejemplo: cámaras) sin comprometer significativamente al rendimiento.
\end{itemize}
\subsection{Actores}
A continuación se definen los actores (o roles) implicados y sus papeles:

\begin{itemize}
    \item Persona que se somete a las pruebas del sistema (ser detectada y reconocida).
    \item Operador del sistema: encargado de levantar y mantener el sistema.
\end{itemize}

\subsection{Casos de uso}

\section{Gestión de riesgos}

\section{Metodología Kanban}


Este enfoque basado en iteraciones ha permitido establecer pequeñas tareas concretas para así cumplir con los objetivos principales del proyecto.

\section{Planificación}

El trabajo correspondiente al presente proyecto se compone de las siguientes tareas, junto con sus hitos (“Hx”) y los entregables (“Ex”) correspondientes: 

\subsection{T1. Instalación del sistema en los diferentes dispositivo} 
\begin{itemize}
    \item \textbf{Descripción y metodología}: Primero de todo, se creará un Dockerfile de despliegue del sistema para cada dispositivo del apartado de infraestructuras. 
    \item \textbf{H1}. Sistema instalado y funcionando en los dispositivos. 
    \item \textbf{E1}. Archivos Dockerfile de despliegue para cada dispositivo. 
\end{itemize}

\subsection{T2. Realización de pruebas en los diferentes equipos} 
\begin{itemize}
    \item \textbf{Descripción y metodología}: Se realizarán diferentes pruebas para medir el rendimiento del sistema en los equipos del apartado de infraestructuras. Los análisis pondrán a prueba los componentes desarrollados frente a múltiples situaciones del sistema (ejemplo: variación del número de cámaras y/o número de personas en escena).
    \item \textbf{H2}. Resultados de rendimiento, de forma que se puedan utilizar para realizar comparaciones con otros dispositivos.
\end{itemize}


\textbf{TODO: incluir las otras 2 Jetson}

\subsection{TODO: Ti. Creación de nuevos datasets }
\begin{itemize}
    \item \textbf{Descripción y metodología}: Se crearán nuevos datasets a partir de los datos de 2 cámaras RGBD con información de nuevas entidades, para medir la capacidad de aprendizaje Open-World del método a implementar. 
    \item \textbf{H3}. Nuevos datasets creados útiles para las pruebas. 
    \item \textbf{E3}. Datasets con información de entidades desconocidas por el sistema. 
\end{itemize}

\subsection{TODO: Ti. Nuevos sistemas embebidos: AGX Jetson y Thor}

\subsection{(TODO: no estaba previsto inicialmente) T3. Migración ROS 2 y actualización de librerías} 

\subsection{T4. Estudio del sistema Open-World}

\begin{itemize}
    \item \textbf{Descripción y metodología}: Se estudiará qué partes del método Open-World escogido \cite{Erik} se pueden adaptar al sistema propuesto. También se estudiarán los conceptos teóricos que los sustentan.
    \item \textbf{H4}. Obtener los conocimientos para aplicar la funcionalidad al sistema.
\end{itemize}

\subsection{T5. Implementación del sistema Open-World}

\begin{itemize}
    \item \textbf{Descripción y metodología}: En base al conocimiento adquirido, se implementarán los componentes que mejor se adaptan al sistema y que permiten la funcionalidad Open-World.
    \item \textbf{H5}. Implementar el reconocimiento de desconocidos en el sistema.
    \item \textbf{E5}. Software con dicha implementación.
\end{itemize}

TODO: separar T4, fusionar T5 y T6.


\begin{itemize}
    \item \textbf{Descripción y metodología}: Se realizarán múltiples análisis de la nueva funcionalidad en los equipos descritos en el apartado de infraestructuras. 
    \item \textbf{H5}. Registrar varias pruebas empíricas del sistema.
\end{itemize}

\subsection{T6. La ejecución eficiente en sistemas embebidos} 
\begin{itemize}
    \item \textbf{Descripción y metodología}: Se aplicarán mejoras a modo de reducir la latencia sin comprometer a la precisión de los reconocimientos.
    \item \textbf{H6}. Obtener una mejora en el rendimiento.
    \item \textbf{E6}. Nuevos resultados acerca del rendimiento, junto con las optimizaciones realizadas.
\end{itemize}

\subsection{T7. Documentación del proyecto }
\begin{itemize}
    \item \textbf{Descripción y metodología}: Los avances del proyecto se documentarán de forma continua en una memoria, con el objetivo de asegurar la reproducibilidad del trabajo.
    \item \textbf{H7}. Registro de todo el trabajo realizado.
    \item \textbf{E7}. Memoria con todo el trabajo registrado, de forma que pueda ser reproducido.
\end{itemize}

\section{Seguimiento}
\subsection{Análisis de desviaciones temporales}